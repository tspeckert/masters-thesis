%Choice of research method:
%
%Description: Requirement for 1 point: that the choice of a research strategy and research methods is clearly motivated and described, that alternative research strategies and methods that could be used to solve the research question are discussed, as well as that relevant ethical considerations are discussed. For 2 points the following is also required: that alternative, applicable research strategies and methods are comprehensively discussed and that a profound reasoning about the chosen strategies and methods is made, where the motives for choices made are clearly evident. For 3 points the following is also required: that the choice of method is discussed in relation to the research strategies and methods that are used in current, related research studies that can be regarded as state-of-the-art.
%
%Instructions: It should be evident how the thesis relates to empirical and design research. If the thesis relates to design research then some method framework should be discussed, see for example A Design Science Primer. Research strategies, data collection methods, and data analysis methods should be described. See for example The Good Research Guide for the differences between theses. The description should include references to method literature. There should also be references to literature regarding ethical aspects, for example Appendix 1 in The Good Research Guide. The discussion should be closely tied to the research question of the thesis. There should not be long, general descriptions of research strategies and methods that are only summations from method literature without connections to the thesis topic.
%
%
%
%Application of research method:
%
%Description: Requirement for 1 point: that the application of the chosen scientific strategies and methods are clearly described and that relevant ethical aspects are discussed. For 2 points the following is also required: that the application of research strategies and methods are done in accordance to the demands of said methods and strategies and that a clear argumentation exists for this. For 3 points the following is also required: that there is a real depth to the data analysis
%
%Instructions: How the chosen research strategies and methods (both data collection and data analysis methods) were applied should be clearly described. If the thesis uses design research it should explain how the chosen method framework for design research has been applied. The description should include references to method literature. There should even be references to literature regarding ethical aspects, for example Appendix 1 in The Good Research Guide.



\section{Choice of Research Method}

Empirical research "aims at describing, explaining and predicting the world"~\cite{johannessonPerjons2012}. In comparison, design research additionally wants to improve upon the world through the development of artifacts. This thesis project is interested in improving on EA by through the development of an EA frameowrk that is suitable for decentralized organizations. As a result, a design research approach, specifically design science, will be utilized. The remainder of this section seeks to further demonstrate the suitability of design science and outline the specific research strategies and methods chosen. 

\subsection{Design Science and its Relevance to this Thesis Work}

Design science is concerned with the development and application of \textit{artifacts} aimed at solving some practical problem~\cite{hevner2004,johannessonPerjons2012} in a manner that is of general interest~\cite{johannessonPerjons2012}. 

In order to be relevant, Design Science must exist in some context. Johannesson and Perjons~\cite{johannessonPerjons2012} define a generic context for design science in terms of people, practices and problems. A practice is a set of related activities performed regularly by people. In the performance of a practice, people encounter practical problems. Two general kinds of problems exist; one where the current state of affairs problematic and the desirable state is neutral, and a second where the current state is neutral and the desirable state is a big, surprising improvement. The artifacts created through design science can be used by people to solve these practical problems. These concepts are easily related to this thesis work: Enterprise Architecture is a practice with a problem of the first type.  The current state of EA is problematic when applied to decentralized organizations. This thesis argues that the existing artifacts (i.e. EA frameworks) are insufficient these types of organizations, and consequently, new artifacts (or a revision to the existing ones) need to be developed in order to solve this problem.

According to Johannesson and Perjons~\cite{johannessonPerjons2012}, artifacts themselves have an ``inner construction'', exist in an environment, and have a function. The inner construction refers to the inner components of the artifact and the relations between them. The environment refers to the artifacts practice, the people using it, and anything else in its surrounds that have an affect on it. Lastly, the function of an artifact is the result of using it in its practice. This definition of an artifact also relates well to this thesis work: the inner construction of our artifact (an EA framework for decentralized organizations) is, at a high level, composed of an EA method, EA description, and EA engine. The relations for these components are outlined in figure~\ref{TODO_IRINAS_DIAGRAM}. The environment of our artifact includes the practice of EA and all affected components of the decentralized organization using it, such as involved stakeholders and implementers. The function of our artifact is, on a high level, to bring the benefits of traditional EA to decentralized organizations (e.g. business-IT alignment).


There exist four different types of artifacts: constructs, models, methods, and instantiations~\cite{hevner2004,johannessonPerjons2012}. Enterprise architecture is concerned with the first three of those types: constructs, models and methods. Constructs are ways to describe some phenomenon. They give a common language for talking about something, but do not make any assertions about reality. For example, the EA description phase of an EA framework provides a common taxonomy for the different parts of an organization covered by EA. Models represent other objects. EA makes use of descriptive models and prescriptive models~\cite{johannessonPerjons2012}. Descriptive models are used to represent a current situation and its challenges, such the "as-is" architecture from the EA description phase. Prescriptive models represent potential future solutions, such as the "to-be" architecture, also from the EA description phase. Methods define "guidelines and processes for how to solve problems and achieve goals"~\cite{johannessonPerjons2012}. The EA method and EA engine phases are primarily methods, the former to construct the EA description, the latter to ensure its proper use throughout its lifecycle. 

\subsection{A Design Science Method Framework}
\label{sec:framework}

Having established the relevance of design science to this thesis project, this thesis will therefore follow the framework for a design science method presented by Johannesson and Perjons in ~\cite{johannessonPerjons2012}. This method is composed of five activities with input-output relationships: Explicate Problem, Outline Artifact and Define Requirements, Design and Develop Artifact, and Evaluate Artifact. Each activity has an output which serves as an input to the next activity (e.g. an explicated problem is the input to the Outline Artifact and Define Requirements phase). These activities are carried out in an iterative manner, meaning that the practitioner will move back and forth between them as opposed to working in a sequential manner. 

The Explicate Problem activity is concerned with outlining the problem addressed by the research work in detail. To this end, the problem's significance needs to be clearly stated and its underlying causes can be possibly identified and analyzed. The output of this phase an explicated problem. 

The Outline Artifact and Define Requirements activity is where the explicated problem is transformed into the requirements for a solution to said problem. The output of this phase is the set of requirements for the artifact. 

The Design and Develop Artifact activity is where the artifact itself is built based on the requirements for the artifact. The output of this phase is the artifact itself.

The Demonstrate Artifact activity takes the developed artifact and implements it in either a real or illustrative case in order to demonstrate its viability. The output of this phase is the demonstrated artifact. 

The Evaluate Artifact activity is to demonstrate the artifact's fulfillment of the requirements and the degree to which it solves the problem. The output here is an evaluated artifact. 

\subsection{Choice of Research Strategy}

Alternative strategies exist for undertaking research in the field of design science. A number of common strategies will be briefly outlined in order to discuss their suitability for this thesis. 

Surveys aim to take a comprehensive look at something by gathering data from a large number of different sources. This data is then analyzed in some manner. Surveys offer a wide view~\cite{denscombe2010good,johannessonPerjons2012}, and as such, are not well suited for a depth view of something. This does not fit in with this project which takes an in-depth look at EA frameworks. 

Experiments employ a controlled and artificial environment in order to isolate a small number of specific factors to study them in detail. The effects of manipulating variables in the environment needs to be precisely measured~\cite{denscombe2010good}. This poses a problem for EA as organizations are highly complex entities where it would be exceptionally difficult to exert precise control and precisely measure the effects. For this reason, experiments are not a suitable strategy for this project. 

In action research, the researcher is an active participant in affecting the environment they are researching. Here, the research is done as part of the practice, as opposed to it being a separate activity~\cite{denscombe2010good}. This could be a highly effective strategy for this thesis topic as it would allow the researcher to experience the problems of decentralization first-hand. Furthermore, action research is a cyclical process, meaning the researcher could repeatedly try out different solutions and evaluate their effects in order to come to a good solution. This would allow for a researcher in a decentralized organization the flexibility to find a solution that works. Despite this fit to the thesis topic, the practical issue of finding a decentralized organization that is willing to go through this process is a significant one. As a result, action research is not used in this project. 

Ethnography is similar to action research in that the researcher becomes a member of the environment being researched. The difference lies in that they are there to integrate themselves into it, rather than affect change~\cite{denscombe2010good}. This could be an applicable research strategy for understanding problems from the perspective of stakeholders, however finding a decentralized organization with some sort of EA (or at least an interest in it) is quite the challenge in itself. Additionally, ethnography requires a large time investment in order to integrate adequately into the environment of study, which is not feasible for a Master's project. For these reasons, ethnography is not used in this project. 

Case studies take an in depth view of a single instance of the practice where the problem of interest exists. Case studies are ideal when "a researcher wants to investigate an issue in depth and provide an explanation that can cope with the complexity and subtlety of real life situations"~\cite{denscombe2010good}. This project is interested in an in depth view of the problem of suitability of EA for decentralized organizations and decentralized organizations are real life entities that are highly complex. Furthermore, in contrast with ethnography and action research, conducting a single case study fits in well with the scope of a Master's project; it is not necessary for the subject organization to invest large amounts of resources into the project and time investment of a case study fits in with a short-term project. For these reasons, this thesis project will employ a case study research strategy.  

\subsection{Choice of Research Methods}

Research strategies do not prescribe any concrete ways to generate and analyze data. Specific research methods for data generation and data analysis are needed. 


\paragraph*{Data generation methods}

%%data generation

This thesis employs the use of interviews and document studies for data generation.  Document studies are used as a large amount of data on the structure of the organization being studied is available. Documents are a good source of authoritative, objective, and factual data~\cite{denscombe2010good}, which therefore gives a solid foundation on understanding the studied organization. Interviews were chosen in order to supplement this data with information from stakeholders about the organization. Interviews are suited for gaining insight into complex phenomena, which is supportive of our need for an in-depth view of a complex entity that is an organization. Furthermore, interviews are practical for this project as; a) the organization being studied does not need to invest large amounts of time and b) I have physical access to potential interviewees. 

Other common data generation methods are questionnaires and observations. Questionnaires are not particularly suitable for this project as they are most useful when used for specific, straightforward information~\cite{denscombe2010good}. This project, on the other hand, is interested in the complexities of an organization. An observation study would require spending time in an organization in order to directly observe its operations. As this thesis is conducted as an individual project, this is not a feasible activity, due to the size and complexity of an organization.

\paragraph*{Data analysis methods}

After the data has been obtained, it is necessary to analyze it in order to understand the object being studied. Data can be analyzed in either a quantitative or a qualitative manner. Quantitative data analysis is concerned with numeric data, whereas qualitative deals with words and visuals. According to Denscombe ~\cite{denscombe2010good}, some other differences between the approaches are; quantitative research is generally associated with large-scale studies whereas qualitative research is concerned with small-scale studies, and quantitative research is concerned with "specific variables" while qualitative research takes a "holistic perspective". This project follows a qualitative approach because; a) the data being analyzed will composed of words coming from interviews and document studies, b) this is a small-scale study, and c) this project is interested in a holistic perspective on our case study subject.

\section{Application of Method}

This thesis work follows the framework by Johannesson and Perjons~\cite{johannessonPerjons2012} presented above in Section \ref{sec:framework}. 

\subsection{Explicate Problem}

formulate precisely the initial problem
motivate its importance
investigate its underlying causes

\subsection{Outline Artifact and Define Requirements}



\subsection{Design and Develop Artifact}



\subsection{Demonstrate Artifact}



\subsection{Evaluate Artifact}



\section{Ethical Considerations}
