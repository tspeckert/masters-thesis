%Choice of research method:
%
%Description: Requirement for 1 point: that the choice of a research strategy and research methods is clearly motivated and described, that alternative research strategies and methods that could be used to solve the research question are discussed, as well as that relevant ethical considerations are discussed. For 2 points the following is also required: that alternative, applicable research strategies and methods are comprehensively discussed and that a profound reasoning about the chosen strategies and methods is made, where the motives for choices made are clearly evident. For 3 points the following is also required: that the choice of method is discussed in relation to the research strategies and methods that are used in current, related research studies that can be regarded as state-of-the-art.
%
%Instructions: It should be evident how the thesis relates to empirical and design research. If the thesis relates to design research then some method framework should be discussed, see for example A Design Science Primer. Research strategies, data collection methods, and data analysis methods should be described. See for example The Good Research Guide for the differences between theses. The description should include references to method literature. There should also be references to literature regarding ethical aspects, for example Appendix 1 in The Good Research Guide. The discussion should be closely tied to the research question of the thesis. There should not be long, general descriptions of research strategies and methods that are only summations from method literature without connections to the thesis topic.
%
%
%
%Application of research method:
%
%Description: Requirement for 1 point: that the application of the chosen scientific strategies and methods are clearly described and that relevant ethical aspects are discussed. For 2 points the following is also required: that the application of research strategies and methods are done in accordance to the demands of said methods and strategies and that a clear argumentation exists for this. For 3 points the following is also required: that there is a real depth to the data analysis
%
%Instructions: How the chosen research strategies and methods (both data collection and data analysis methods) were applied should be clearly described. If the thesis uses design research it should explain how the chosen method framework for design research has been applied. The description should include references to method literature. There should even be references to literature regarding ethical aspects, for example Appendix 1 in The Good Research Guide.
%

\subsection{Relevant Research Methods}

Design science



\subsection{Choice of Research Method}

Design Science
% Maybe doesn't need whole section

\subsection{Design Science}

Definitions: (paul)
empirical --> describe
design research --> change world

design research --> develop artifacts

design science aims to create artefacts that address
problems experienced by people

A practice is a set of human activities performed regularly and seen
as meaningfully related to each other by the people participating in
them. --> practice of EA

First, there are problems in which the current state is
viewed as truly unsatisfying and the desirable state is seen as neutral,
e.g., having a toothache or a flat tire. Secondly, there are problems
where the current state is seen as neutral and the desirable
state is regarded as a potentially huge and surprising improvement. --> first: currently EA is unsatisfying for DECEN.

Summarising,
the term "problem" is used here to denote troublesome situations as
well as promising opportunities.

In many cases, practical problems can be solved by means of artefacts.
An artefact is defined here as an object made by humans with the intention to be used for addressing a practical problem. Some
artefacts are physical objects, like hammers, cars and hip replacements. Other artefacts take the form of drawings or blueprints, e.g.
an architect's plan for a building. Methods and guidelines can also be
artefacts, e.g. a method for designing databases. Common to all these
artefacts is that they support people when they encounter problems
in some practice. --> our artifact is an EA Framework (that is suitable for D)

The relationships between people, practices, problems and artefacts
are summarized in figure 1.1. People engage in practices in
which they may perceive problems that can be addressed by means
of artefacts. Thus, artefacts do not exist in isolation but are always
embedded in a larger context

Artifact anatomy

More precisely, every artefact has an inner construction,
is situated in an environment, and offers a function for its
intended practice. The intended practice is here defined as the practice
that holds the practical problem, which the artefact is to address.

The construction of an artefact is about its inner workings, the
components it consists of, how these are related, and how they interact
with each other. An artefact is always constructed from smaller
parts that are assembled in such a way that they can interact with
each other and produce some behaviour. --> EA Method, EA Description, EA Engine


The environment of an artefact is about the external surroundings
and conditions in which it will operate. The environment of an
artefact always includes its intended practice as well as the people
and other objects participating in that practice. The environment
may also include other practices that are affected by the use of the
artefact. Furthermore, the environment may contain various objects
that are not related to any specific practice. --> Decentralized Organizations


The function of an artefact is the intended result of using it in its
intended practice, i.e. those benefits the artefact is expected to bring
to its users. --> Benefits of EA for DO




\subsection{Application of Method}

\subsection{Ethical Considerations}
