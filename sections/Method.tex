%Choice of research method:
%
%Description: Requirement for 1 point: that the choice of a research strategy and research methods is clearly motivated and described, that alternative research strategies and methods that could be used to solve the research question are discussed, as well as that relevant ethical considerations are discussed. For 2 points the following is also required: that alternative, applicable research strategies and methods are comprehensively discussed and that a profound reasoning about the chosen strategies and methods is made, where the motives for choices made are clearly evident. For 3 points the following is also required: that the choice of method is discussed in relation to the research strategies and methods that are used in current, related research studies that can be regarded as state-of-the-art.
%
%Instructions: It should be evident how the thesis relates to empirical and design research. If the thesis relates to design research then some method framework should be discussed, see for example A Design Science Primer. Research strategies, data collection methods, and data analysis methods should be described. See for example The Good Research Guide for the differences between theses. The description should include references to method literature. There should also be references to literature regarding ethical aspects, for example Appendix 1 in The Good Research Guide. The discussion should be closely tied to the research question of the thesis. There should not be long, general descriptions of research strategies and methods that are only summations from method literature without connections to the thesis topic.
%
%
%
%Application of research method:
%
%Description: Requirement for 1 point: that the application of the chosen scientific strategies and methods are clearly described and that relevant ethical aspects are discussed. For 2 points the following is also required: that the application of research strategies and methods are done in accordance to the demands of said methods and strategies and that a clear argumentation exists for this. For 3 points the following is also required: that there is a real depth to the data analysis
%
%Instructions: How the chosen research strategies and methods (both data collection and data analysis methods) were applied should be clearly described. If the thesis uses design research it should explain how the chosen method framework for design research has been applied. The description should include references to method literature. There should even be references to literature regarding ethical aspects, for example Appendix 1 in The Good Research Guide.
%

\subsection{Relevant Research Methods}

Design science

Artefacts are studied in different fields of science, including formal
sciences, behavioural sciences and social sciences. For example, a
study in theoretical computer science (formal science) could determine
the complexity properties of a new algorithm for traversing a
social graph. A study in psychology (behavioural science) could investigate
how photo sharing on social networks influences stress
levels. A study in business administration (social science) could examine
how the adoption of ERP systems in companies affects their
internal communication.



\subsection{Choice of Research Method}

Design Science
% Maybe doesn't need whole section

\subsection{Design Science}

Definitions: (paul)

"Design science is the scientific study and creation of artefacts
as they are developed and used by people with the
goal of solving practical problems of general interest"

While design is a process
for developing a working solution to a problem that may be
relevant only for a single actor, design science aims at producing and
communicating knowledge that is of general interest.

empirical --> describe
design research --> change world

design research --> develop artifacts

design science aims to create artefacts that address
problems experienced by people

A practice is a set of human activities performed regularly and seen
as meaningfully related to each other by the people participating in
them. --> practice of EA

First, there are problems in which the current state is
viewed as truly unsatisfying and the desirable state is seen as neutral,
e.g., having a toothache or a flat tire. Secondly, there are problems
where the current state is seen as neutral and the desirable
state is regarded as a potentially huge and surprising improvement. --> first: currently EA is unsatisfying for DECEN.

Summarising,
the term "problem" is used here to denote troublesome situations as
well as promising opportunities.

In many cases, practical problems can be solved by means of artefacts.
An artefact is defined here as an object made by humans with the intention to be used for addressing a practical problem. Some
artefacts are physical objects, like hammers, cars and hip replacements. Other artefacts take the form of drawings or blueprints, e.g.
an architect's plan for a building. Methods and guidelines can also be
artefacts, e.g. a method for designing databases. Common to all these
artefacts is that they support people when they encounter problems
in some practice. --> our artifact is an EA Framework (that is suitable for D)

The relationships between people, practices, problems and artefacts
are summarized in figure 1.1. People engage in practices in
which they may perceive problems that can be addressed by means
of artefacts. Thus, artefacts do not exist in isolation but are always
embedded in a larger context

Artifact anatomy

More precisely, every artefact has an inner construction,
is situated in an environment, and offers a function for its
intended practice. The intended practice is here defined as the practice
that holds the practical problem, which the artefact is to address.

The construction of an artefact is about its inner workings, the
components it consists of, how these are related, and how they interact
with each other. An artefact is always constructed from smaller
parts that are assembled in such a way that they can interact with
each other and produce some behaviour. --> EA Method, EA Description, EA Engine


The environment of an artefact is about the external surroundings
and conditions in which it will operate. The environment of an
artefact always includes its intended practice as well as the people
and other objects participating in that practice. The environment
may also include other practices that are affected by the use of the
artefact. Furthermore, the environment may contain various objects
that are not related to any specific practice. --> Decentralized Organizations


The function of an artefact is the intended result of using it in its
intended practice, i.e. those benefits the artefact is expected to bring
to its users. --> Benefits of EA for DO

When designing an artefact, a designer often starts by creating a
specification that defines the functional requirements for the artefact,
i.e. which functions the artefact should offer. Two requirements
for a watch are that it should be usable as a stopwatch and as an
alarm clock.

Design Science - the Study of Artefacts


** knowledge **

Scientific research produces different kinds of knowledge and there
have been numerous attempts to classify these

Another aspect of knowledge is the form in which it can be materialized.
This is particularly relevant for design science, as the
knowledge it produces is not always explicit but sometimes embedded
in artefacts.


Definitional knowledge consists of concepts, constructs, terminologies,
definitions, vocabularies, classifications, taxonomies and other
kinds of conceptual knowledge. It may be formal and precise, such as
the basic concepts of relational database theory, which includes relations,
attributes, functional dependencies, and multi-valued dependencies.
Definitional knowledge may also include vague and informal
concepts like the HCI notions of usability, affordance, and situatedness.
Definitional knowledge does not include statements about reality
that are claimed to be true. --> maybe?

does not [describe if something] really exist; such claims are made by descriptive knowledge.


Descriptive knowledge describes and analyses an existing or past
reality. This type of knowledge typically describes, summarises, generalises
and classifies observations of phenomena or events. For
example, descriptive knowledge can claim that a majority of the ERP
users in a company are dissatisfied. --> me?


Explanatory knowledge provides answers to how and why questions,
explaining how objects behave and why events occur.


Predictive knowledge offers black-box predictions, i.e. it predicts
outcomes based on underlying factors but without explaining causal
or other relationships between them. The goal of predictive
knowledge is accurate prediction, not understanding.


Explanatory and predictive knowledge offers both explanations and
predictions. It predicts outcomes and explains how these are related,
often through causal relationships, to underlying mechanisms and
factors. This kind of knowledge is considered the most common in
natural science.


Prescriptive knowledge consists of methods and prescriptive models
that help solve practical problems. Prescriptive models can be seen
as blueprints for developing artefacts, while methods are guidelines
and procedures that help people to work in systematic ways when
solving problems. Methods often, but not always, prescribe how to
construct artefacts. --> me



Knowledge forms are important for design
science work, as it creates not only knowledge explicitly codi-fied in documents but also embedded in artefacts. The following subsections
will introduce the three main knowledge forms and their
roles in design science.  

Explicit Knowledge
Explicit knowledge is articulated, expressed and recorded in media
such as text, numbers, codes, formulas, musical notations, and video
tracks. Typical containers of explicit knowledge are encyclopaedias,
textbooks, and manuals. A main strength of explicit knowledge is
that it can be easily transferred between individuals, as it is stored in
media that can be moved and decoded in a convenient way. However,
not all knowledge can be articulated in an explicit way,



4 types of artifacts:

Based on the knowledge types and forms introduced, it is possible to
distinguish between different artefact types. Within the IT area, it
has become common to identify four such types: constructs, models,
methods, and instantiations.

Constructs are terms, notations, definitions, and concepts that are
needed for formulating problems and their possible solutions. Constructs
do not make any statements about the world, but they make
it possible to speak about it, so it can be understood and changed.
Thus, constructs are definitional knowledge

Models are used to depict or represent other objects. A model can
represent an existing situation, which can be used for describing and
analyzing problem situations. Such a descriptive model may work as a
pedagogical tool for representing a current situation and explaining
why it is challenging. However, models can also be used to describe
potential solutions to practical problems, e.g. a drawing for a new
type of vehicle or a proposal for an architecture of a mobile operating
system.

Methods express prescriptive knowledge by defining guidelines
and processes for how to solve problems and achieve goals. In particular,
they can prescribe how to create artefacts. Methods can be
highly formalized like algorithms, but they can also be informal such
as rules of thumb or best practices.

Instantiations are working systems that can be used in a practice.
Instantiations can always embed knowledge, e.g. a database can embed
a database model. Some examples of instantiations are a Java
program realising a search algorithm, a database for electronic medical
records, or a new planet in the computer game Entropia.

%research strategies and methods
%%data generation

This chapter offers an overview
of a number of well-established research strategies and methods for
empirical research, particularly within the social sciences. These
strategies and methods are useful also when doing design science, in
particular when investigating practical problems, defining requirements
and evaluating artefacts.








\subsection{Application of Method}

\subsection{Ethical Considerations}
