%Choice of research method:
%
%Description: Requirement for 1 point: that the choice of a research strategy and research methods is clearly motivated and described, that alternative research strategies and methods that could be used to solve the research question are discussed, as well as that relevant ethical considerations are discussed. For 2 points the following is also required: that alternative, applicable research strategies and methods are comprehensively discussed and that a profound reasoning about the chosen strategies and methods is made, where the motives for choices made are clearly evident. For 3 points the following is also required: that the choice of method is discussed in relation to the research strategies and methods that are used in current, related research studies that can be regarded as state-of-the-art.
%
%Instructions: It should be evident how the thesis relates to empirical and design research. If the thesis relates to design research then some method framework should be discussed, see for example A Design Science Primer. Research strategies, data collection methods, and data analysis methods should be described. See for example The Good Research Guide for the differences between theses. The description should include references to method literature. There should also be references to literature regarding ethical aspects, for example Appendix 1 in The Good Research Guide. The discussion should be closely tied to the research question of the thesis. There should not be long, general descriptions of research strategies and methods that are only summations from method literature without connections to the thesis topic.
%
%
%
%Application of research method:
%
%Description: Requirement for 1 point: that the application of the chosen scientific strategies and methods are clearly described and that relevant ethical aspects are discussed. For 2 points the following is also required: that the application of research strategies and methods are done in accordance to the demands of said methods and strategies and that a clear argumentation exists for this. For 3 points the following is also required: that there is a real depth to the data analysis
%
%Instructions: How the chosen research strategies and methods (both data collection and data analysis methods) were applied should be clearly described. If the thesis uses design research it should explain how the chosen method framework for design research has been applied. The description should include references to method literature. There should even be references to literature regarding ethical aspects, for example Appendix 1 in The Good Research Guide.


% Maybe doesn't need whole section


from hevner {
In the design-science
paradigm, knowledge and understanding of a
problem domain and its solution are achieved in
the building and application of the designed artifact.
Three recent exemplars in the research
literature are used to demonstrate the application
of these guidelines. We conclude with an analysis
of the challenges of performing high-quality
design-science research in the context of the
broader IS community.

Guideline 1: Design as an Artifact 
Design-science research must produce a viable artifact in the
form of a construct, a model, a method, or an instantiation.

Guideline 2: Problem Relevance 
The objective of design-science research is to develop
technology-based solutions to important and relevant
business problems.

Guideline 3: Design Evaluation 
The utility, quality, and efficacy of a design artifact must be
rigorously demonstrated via well-executed evaluation
methods.

Guideline 4: Research Contributions
Effective design-science research must provide clear and
verifiable contributions in the areas of the design artifact,
design foundations, and/or design methodologies.

Guideline 5: Research Rigor 
Design-science research relies upon the application of
rigorous methods in both the construction and evaluation of
the design artifact.

Guideline 6: Design as a Search Process
The search for an effective artifact requires utilizing available
means to reach desired ends while satisfying laws in the
problem environment.

Guideline 7: Communication of Research
Design-science research must be presented effectively both
to technology-oriented as well as management-oriented
audiences.
}

Empirical research aims at describing, explaining and predicting the
world. The goal of empirical research, at least in the natural sciences, is to faithfully
describe and explain the world as it exists regardless of human
interests and biases. The world is out there, and can be explained by
science so that people have a common understanding of it, irrespective
of their backgrounds, traditions and values.

In contrast to empirical research, design research is not content
only to describe, explain and predict. It also wants to change the
world, to improve it and to create new worlds. Design research does
this by developing artefacts that can help people fulfil their needs,
overcome their problems, and grasp new opportunities. In this endeavour,
design research not only creates novel artefacts but also
knowledge about them, their use, and their environment.

Design science

Artefacts are studied in different fields of science, including formal
sciences, behavioural sciences and social sciences. For example, a
study in theoretical computer science (formal science) could determine
the complexity properties of a new algorithm for traversing a
social graph. A study in psychology (behavioural science) could investigate
how photo sharing on social networks influences stress
levels. A study in business administration (social science) could examine
how the adoption of ERP systems in companies affects their
internal communication.



empirical --> describe
design research --> change world

design research --> develop artifacts

design science aims to create artefacts that address
problems experienced by people



\subsection{Choice of Research Method}

Briefly explain my choice of Design Science (e.g. goal of an artifact) over Empirical 

\subsubsection{Design Science and its Relevance to this Thesis Work}


Design science is concerned with the development and application of \textit{artifacts} aimed at solving some practical problem~\cite{hevner2004,johannessonPerjons2012} in a manner that is of general interest~\cite{johannessonPerjons2012}. 

In order to be relevant, Design Science must exist in some context. Johannesson and Perjons~\cite{johannessonPerjons2012} define a generic context for design science in terms of people, practices and problems. A practice is a set of related activities performed regularly by people. In the performance of a practice, people encounter practical problems. Two general kinds of problems exist; one where the current state of affairs problematic and the desirable state is neutral, and a second where the current state is neutral and the desirable state is a big, surprising improvement. The artifacts created through design science can be used by people to solve these practical problems. These concepts are easily related to this thesis work: Enterprise Architecture is a practice with a problem of the first type.  The current state of EA is problematic when applied to decentralized organizations. This thesis argues that the existing artifacts (i.e. EA frameworks) are insufficient these types of organizations, and consequently, new artifacts (or a revision to the existing ones) need to be developed in order to solve this problem.

According to Johannesson and Perjons~\cite{johannessonPerjons2012}, artifacts themselves have an ``inner construction'', exist in an environment, and have a function. The inner construction refers to the inner components of the artifact and the relations between them. The environment refers to the artifacts practice, the people using it, and anything else in its surrounds that have an affect on it. Lastly, the function of an artifact is the result of using it in its practice. This definition of an artifact also relates well to this thesis work: the inner construction of our artifact (an EA framework for decentralized organizations) is, at a high level, composed of an EA method, EA description, and EA engine. The relations for these components are outlined in figure~\ref{TODO_IRINAS_DIAGRAM}. The environment of our artifact includes the practice of EA and all affected components of the decentralized organization using it, such as involved stakeholders and implementers. The function of our artifact is, on a high level, to bring the benefits of traditional EA to decentralized organizations (e.g. business-IT alignment).


There exist four different types of artifacts: constructs, models, methods, and instantiations~\cite{hevner2004,johannessonPerjons2012}. Enterprise architecture is concerned with the first three of those types: constructs, models and methods. Constructs are ways to describe some phenomenon. They give a common language for talking about something, but do not make any assertions about reality. For example, the EA description phase of an EA framework provides a common taxonomy for the different parts of an organization covered by EA. Models represent other objects. EA makes use of descriptive models and prescriptive models~\cite{johannessonPerjons2012}. Descriptive models are used to represent a current situation and its challenges, such the "as-is" architecture from the EA description phase. Prescriptive models represent potential future solutions, such as the "to-be" architecture, also from the EA description phase. Methods define "guidelines and processes for how to solve problems and achieve goals"~\cite{johannessonPerjons2012}. The EA method and EA engine phases are primarily methods, the former to construct the EA description, the latter to ensure its proper use throughout its lifecycle. 

\subsubsection{A Design Science Method Framework}

Having established the relevance of design science to this thesis project, this thesis will therefore follow the framework for a design science method presented by Johannesson and Perjons in ~\cite{johannessonPerjons2012}. This method is composed of five activities with input-output relationships: Explicate Problem, Outline Artifact and Define Requirements, Design and Develop Artifact, and Evaluate Artifact. Each activity has an output which serves as an input to the next activity (e.g. an explicated problem is the input to the Outline Artifact and Define Requirements phase). These activities are carried out in an iterative manner, meaning that the practitioner will move back and forth between them as opposed to working in a sequential manner. 

The Explicate Problem activity is concerned with outlining the problem addressed by the research work in detail. To this end, the problem's significance needs to be clearly stated and its underlying causes can be possibly identified and analyzed. The output of this phase an explicated problem. 

The Outline Artifact and Define Requirements activity is where the explicated problem is transformed into the requirements for a solution to said problem. The output of this phase is the set of requirements for the artifact. 

The Design and Develop Artifact activity is where the artifact itself is built based on the requirements for the artifact. The output of this phase is the artifact itself.

The Demonstrate Artifact activity takes the developed artifact and implements it in either a real or illustrative case in order to demonstrate its viability. The output of this phase is the demonstrated artifact. 

The Evaluate Artifact activity is to formally(?) demonstrate the artifact's fulfillment of the requirements and the degree to which it solves the problem. The output here is an evaluated artifact. 

\subsubsection{Choice of Research Strategies}

A number of different research strategies exist

surveys
  - view something comprehensively and in detail
experiments
  - empirical investigation under controlled conditions
case studies
action research
  - actively participating in introducing change into an environment while doing research
grounded theory
  - about generating theories based on analysis of empirical data
  - not appropriate as we are starting out with a theory
ethnography
  - spending time in a culture
phenomenology
  - 


Data generation and analysis methods are never used in isolation but
always in the context of a research strategy, i.e. an overall approach
to answering a research question. In addition to data generation and
analysis methods, a research strategy includes the general set-up of
the context in which the research is undertaken.

Surveys
A survey starts by generating data from a large group of objects
(people, organisations, systems, etc.) often through use of questionnaires
or document studies. The data are then analysed, typically by
a quantitative method, in order to find generalisations.

Experiments
In an experiment, a researcher creates an artificial environment that
will make it possible to isolate and study a small number of objects,
thereby preventing other objects from influencing those under investigation.
The researcher will then observe (a data generation
method) the behaviour and relationships of the objects in the environment
and analyse the generated data (using a data analysis
method), often with the purpose of establishing cause and effect relationships.

Case studies --> me
A case study investigates in detail one specific case of the general
phenomenon under investigation, e.g. one organisation, one systems
development project, or one mobile application. The purpose of a
case study is to paint a rich picture of a single object or situation as a
basis for obtaining a deep and comprehensive understanding of
some general phenomenon

Action research
In action research, a researcher introduces some change into a real
environment.

Grounded theory
In grounded theory, a researcher generates theory or hypotheses
from data. She gathers data from a practice and identifies patterns or
structures in the collected data, which will become the base for creating
the theory or hypotheses. A main thrust in grounded theory is
that the researcher should try to avoid applying her preunderstanding
or theoretical knowledge about the data. Instead, the
data should "speak for itself' to the researcher and be the sole, or at
least primary, basis for creating the theory.

Ethnography
In ethnographic studies, the purpose is to understand the culture
and perspectives of some group of people, e.g., systems administrators
in a large company. The researcher is not a detached observer
but tries to become a part of the studied group by participating in
their daily activities. Thereby, she gains a deep and comprehensive
understanding of how some groups view their practice.

\subsubsection{Choice of Research Methods}

This chapter offers an overview
of a number of well-established research strategies and methods for
empirical research, particularly within the social sciences. These
strategies and methods are useful also when doing design science, in
particular when investigating practical problems, defining requirements
and evaluating artefacts.

An early activity in any empirical research study is to collect data
about the phenomenon under investigation. For this purpose, data
generation methods are used.

%%data generation

Interviews --> me
An interview is a communication session between a researcher and a
respondent, where the researcher controls the agenda by asking
questions of the respondent. An interview can be structured meaning
that it strictly follows a predefined protocol. Alternatively, it can
be semi-structured or unstructured, providing opportunities for digressing
from a protocol and allowing the respondent to take initiatives.

Group discussions
A group discussion is a communication session in which a researcher
and a group of respondents interact under the guidance of the researcher
and the participants influence each other in order to generate
information.

Questionnaires
A questionnaire is a written document consisting of questions distributed
to a number of respondents. A questionnaire can include
questions with predefined answers, as well as open questions where
the respondent can answer more freely.

Observation studies
In an observation study, objects are observed in their natural environments.
The researcher often tries to be as invisible as possible in
order not to interfere with the natural dynamics of the situation under
investigation.

Document studies --> me
A document study is a special kind of observation study where existing
documents are examined, including policy documents, user manuals,
video recordings, system specifications, system logs and websites.

When data have been generated, they need to be analysed. For this
purpose, there exist both quantitative and qualitative data analysis
methods. Quantitative data analysis focuses on numeric (quantitative)
data and uses mathematical approaches, in particular statistical
ones, in order to investigate, interpret and structure data. Qualitative
data analysis is about interpreting data of any form (both qualitative
and quantitative) by discovering themes and patterns in them,
thereby using the personal skills and understanding of the researcher
who undertakes the analysis. Examples of qualitative analysis
methods are content analysis and narrative analysis.


%%Research Strategies
Data generation and analysis methods are never used in isolation but
always in the context of a research strategy, i.e. an overall approach
to answering a research question. In addition to data generation and
analysis methods, a research strategy includes the general set-up of
the context in which the research is undertaken.

Surveys
A survey starts by generating data from a large group of objects
(people, organisations, systems, etc.) often through use of questionnaires
or document studies. The data are then analysed, typically by
a quantitative method, in order to find generalisations.

Experiments
In an experiment, a researcher creates an artificial environment that
will make it possible to isolate and study a small number of objects,
thereby preventing other objects from influencing those under investigation.
The researcher will then observe (a data generation
method) the behaviour and relationships of the objects in the environment
and analyse the generated data (using a data analysis
method), often with the purpose of establishing cause and effect relationships.

Case studies --> me
A case study investigates in detail one specific case of the general
phenomenon under investigation, e.g. one organisation, one systems
development project, or one mobile application. The purpose of a
case study is to paint a rich picture of a single object or situation as a
basis for obtaining a deep and comprehensive understanding of
some general phenomenon

Action research
In action research, a researcher introduces some change into a real
environment.

Grounded theory
In grounded theory, a researcher generates theory or hypotheses
from data. She gathers data from a practice and identifies patterns or
structures in the collected data, which will become the base for creating
the theory or hypotheses. A main thrust in grounded theory is
that the researcher should try to avoid applying her preunderstanding
or theoretical knowledge about the data. Instead, the
data should "speak for itself' to the researcher and be the sole, or at
least primary, basis for creating the theory.

Ethnography
In ethnographic studies, the purpose is to understand the culture
and perspectives of some group of people, e.g., systems administrators
in a large company. The researcher is not a detached observer
but tries to become a part of the studied group by participating in
their daily activities. Thereby, she gains a deep and comprehensive
understanding of how some groups view their practice.

***
Can social science be used when doing design science?
Short answer: Yes
***

Figure 3.1 shows the relationships between research strategies, data
generation methods, and data analysis methods. It also reads from
top to bottom, as a simple guide to the initial decisions in a research
project. First, the researcher will select one or more research strategies
to be applied. For each research strategy, she will select one or
more data generation methods to use as well as one or more data
analysis methods. In practice, these decisions will not be as serial as
suggested here but will rather develop in iterations during the entire
research project.




***
design science vs. action research

The purpose and activities of design science are close to
those of action research. Both aim to change and improve human practices
and thus can be viewed as practice research, as pointed out by
Goldkuhl {2012). Furthermore, both include problem solving and evaluation.
However, there are also differences. Design science addresses
practical problems through the design and positioning of artefacts,
while action research does not require an artefact to be part of the solution
addressing the practical problems. Instead, action research addresses
problems through psychological, social and organizational
change. Furthermore, design science does not require a practical problem
from a specific organisation as the starting point for the research,
while this is required in action research. Finally, action research is a
single research strategy, while design science can make use of many
different research strategies,
***

%strategies and methods in DS 
any research strategy
can be used in any activity.

%relate to knoeledge base

The results from a design science project should be original and well
founded. Therefore, it is required to relate both design science activities
and their results to an existing knowledge base. Such a
knowledge base may include models and theories from several fields
of science. In the IT and information systems area, theories from
social and behavioural sciences are particularly relevant but also
models from formal sciences



%visualize

The proposed design science method can be visualized using IDEFOJ
which is a technique for describing systems as a number of interrelated
activities, graphically represented as boxes,

Four kinds of channels are used in IDEFO: input, output, control and
resources.
- Input (arrow from left) -describes what Imowledge or object
is the input to an activity
- Output (arrow to right)- describes what knowledge or object
is the output from an activity
- Controls (arrow from above) - describe what knowledge is
used for governing an activity, including research strategies,
research methods, and creative methods
- Resources (arrow from below)- describe what knowledge is
used as the basis of an activity, i.e. the knowledge base including
models and theories


%Must a design science project always create a new artefact?
%Short answer: No

The project may develop a
new artefact from scratch or refine an existing one.

%canvas

However, it is sometimes desirable to get an
even more compact description [than the idef0] and for this purpose, the Design Science
Canvas can be used. The Canvas offers a concise, easily understandable
and visually appealing overview of the key components of
a design science project. The Canvas consists of a rectangle divided
into a number of fields that describe the artefact under consideration,
the problem it addresses, the knowledge base used, the requirements
on the artefact, the research strategies and methods
used in the project, and the results of the project.


%ch5-9 descbie activites AND ETHICS

%communicate artefact knowledge

% DS method + systems dev

The answers to these questions [why ds over sd?] depend on the different purposes of
design and design science

Design science, in contrast, aims at producing and
communicating new knowledge that is relevant for a global practice

3 additional reqs on ds:

1 the purpose of creating new and generalizable knowledge
requires design science projects to make use of rigorous research
strategies and methods.

2 the results produced must be related to an existing knowledge base,
thereby ensuring that they are both well founded and original. It is
not sufficient just to produce some knowledge; it also has to be integrated
with previous knowledge in the area and shown to provide
novel insights.

3 the new knowledge should be communicated to both
practitioners and researchers. This requirement on communication
does not exist in most systems development methods, where instead
there are activities devoted to deployment and maintenance.


% template for a thesis
Part I-Introduction

Background
The background section is to provide concise information required
for understanding the problem and the goal of the thesis. The background
typically includes a description of the practice in which the
problem arises.

Problem
The problem section is to describe the practical problem that motivates
the thesis. A practical problem is often a situation that involves
or causes significant difficulties, disadvantages or risks to people or
organizations, such as people being exposed to health hazards, businesses
losing money or citizens receiving poor service from government
agencies. A practical problem can also be about new possibilities,
e.g. how tablets could be used in health care. A problem should
be of general interest. The problem should be described in such a
way that the reader becomes convinced that it is important to address
it.

Goal
The goal should relate to the
problem, i.e. the problem should be addressed by achieving the goal.
A goal should be of general interest, i.e. it should be interesting not
only for a local practice. The section should make clear what type of
artefact is to be developed, such as a model or a method.

Part II Extended Background
This part should include two sub-parts, The first sub-part should
include a description of what knowledge from the knowledge base
that has been used as a basis for the research carried out. The second
sub-part should include a description and discussion of research
that is related to the research presented in the thesis. Differences
between the work of the thesis and related work should be
described. Part II should give the reader basic background information
for understanding the research presented in the thesis as
well as a base for validating the originality of the presented research.

Part III Method
This part is to first explain why design science is an appropriate approach
for the thesis. It should then provide a description of the
methods used, particularly the research strategies and research
methods chosen. If creative methods or systems development methods
have been used, these should also be described. All choices shall
be justified, and the advantages and disadvantages of the choices
shall be described. The part is also to discuss alternatives and why
these were not used. Relevant ethical considerations are to be made.

Part IV Explicate Problem
This part is to precisely formulate the initial problem that was introduced
in Part I, explain its significance and possibly analyze its underlying
causes. This part can sometimes be omitted if the initial
problem is sufficiently well described and analyzed in Part I. However,
there are also theses in which problem explication is a large and
important part.

Research question
This section is to formulate the research question that forms the
basis of the problem explication. This can be expressed as a variation
of the following question: "What is the problem experienced by some
stakeholders of a practice and why is it important?"

Method Application
If the problem explication was based on a literature study, this section
should specify the databases and search queries that were used.
If an empirical study was undertaken, the section should describe
how the chosen research strategies and methods were applied. For
example, if interviews were used for data generation, the design of
the interview questions should be discussed as well as the set-up of
the interview situation. Furthermore, the section is to show that the
method application is effective, i.e. that it helps answer the research
question of this part.

Results and analysis
This section is to describe and analyze the results of the problem
explication. Conclusions are drawn that form the basis for answering
the research question of this part. The certainty of the conclusions is
to be discussed, for example in terms of validity, reliability, and
transferability.

Part V: Outline Artefact and Define Requirements
This part is to outline a solution to the explicated problem in the
form of an artefact and identify the requirements for this artefact.

Research question
This section shall formulate the research question that forms the
basis for the requirements definition. This can be expressed as a
variation on the following question: "What artefact can be a solution
for the explicated problem and which requirements on this artefact
are important for the stakeholders?"

Method Application
See Part IV above.

Results and analysis
See Part IV above.

Part VI Design and Develop Artefact
This section is to describe both the developed artefact and the process
followed in its development. If the thesis only evaluates an artefact,
this part only includes a description of it.

Development process
This section is to describe how the development of the artefact was
carried out, particularly the methods used. The use of systems development
methods, creative methods as well as research strategies
and research methods is to be described here. Design decisions and
the reasons for these (design rationale) are to be discussed.

Artefact Description
This section is to describe the developed artefact. The functionality
and construction of the artefact should be described.

Part VII Demonstration
This part is to show how the developed artefact can be used in one
case. The part is omitted if no demonstration was carried out.

Research question
This section shall formulate the research question that forms the
basis for the demonstration. This can be expressed as a variation on 
the following question: "How can the developed artefact be used to
address the explicated problem in one case?"

Method Application
This section is to describe how the chosen research strategies and
research methods were applied. Furthermore, the section is to show
that the method application is effective, i.e. that it helps answer the
research question of this part.

Results and analysis
See Part IV.

Part VIII Evaluation
This part is to evaluate the developed artefact, addressing both the
defined requirements and the explicated problem.

Research question
This section is to formulate the research question that forms the
basis for the evaluation. This can be expressed as a variation on the
following question: "How well does the artefact solve the explicated
problem and fulfil the defined requirements?"

Method Application
If an empirical study was undertaken, the section should describe
how the chosen research strategies and methods were applied. Furthermore,
the section is to show that the method application is effective,
i.e. that it helps answer the research question of this part.

Results and analysis
See Part IV.

Part IX Discussion
This section is to summarize the findings and conclusions of the preceding
parts. A discussion of future research is also to be included in
this part. Significance and originality is to be discussed. Ethical and
societal implications of using the artefact should be discussed.


\subsection{Application of Method}


%Typical Cases of Design Science Projects
five typical cases of design science projects.

Problem focused design science
Some design science projects focus on problem explication. They
carefully investigate a problem situation and divide it into subproblems,
and may also carry out a root cause analysis. They thereby
employ research strategies and methods in a rigorous way and typically
include comprehensive empirical studies. These projects also
define requirements on an artefact based on more or less careful
investigations. The design of the artefact is only outlined and neither
demonstration nor evaluation is carried out.

Requirements focused design science
Other design science projects focus on requirements definition.
These projects start with an existing problem that is simply accepted
as is, or slightly explicated. Careful and rigorous investigations are
carried out in order to collect requirements, which will involve literature
studies, as well as interaction with relevant stakeholders. The
design of the artefact is only outlined and neither demonstration nor
evaluation is carried out.

Requirements and development focused design science
Many design science projects focus on a combination of requirements
definition and artefact development. Such projects will not
perform any problem explication but move directly to requirements
definition, which is carried out in a rigorous way. The artefact is then
developed using research as well as creative methods. The viability
of the artefact is demonstrated or a lightweight evaluation is performed.

Development and evaluation focused design science
It is also common that design science projects focus on development
and evaluation. Such projects will neither perform problem explication
nor requirements definition but start from an existing requirements
specification. The focus will be the design and development of
an artefact using both research and creative methods. There will also
be a demonstration and a thorough evaluation by means of experiments,
case studies, or other research strategies.

Evaluation focused design science
Some design science projects focus on the evaluation activity. Thus,
no artefact is developed, nor even outlined and therefore, it could be
questioned whether such a project should count as design science,
see the box below for a discussion. Evaluation projects often include
careful requirements definition, which results in a requirements
specification used as a basis for the evaluation. The evaluation is
carried out in a rigorous way using adequate research strategies and
methods

\subsection{Ethical Considerations}
