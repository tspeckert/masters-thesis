\subsection{Iteration 1}

\subsubsection*{Generate}

The purpose of this ``Generate'' sub-section is to generate alternative principals that could be used for a solution artifact. One of these alternative principals are used in next sub-section, ``Search and Select'', in order to build the fully developed solution artifact. 

The challenge of decentralization is not a new one; other efforts have been able to address their view on it with success. The specifics of the challenge varies between domains, however there may exist general principles that can be taken and applied to EA. 

One such effort is peer-to-peer architecture. According to Saroiu, Gummadi and Gribble, peer-to-peer systems ``...typically lack dedicated, centralized infrastructure, but rather depend on the voluntary participation of peers to contribute resources out of which the infrastructure is constructed. Membership in a peer-to-peer system is ad-hoc and dynamic...''~\cite{saroiu2001measurement}. 

We argue that peer-to-peer is a relevant concept to decentralization in EA for two reasons. First, individuals in highly decentralized organization are able to contribute to the enterprise in a manner that is completely up to them. This is similar to peers in a peer-to-peer system, where the peers participate in a completely voluntary manner. Second, the challenge that peer-to-peer systems overcome is similar to the main challenge faced by decentralized organizations. Saroiu et al. state that the challenge of peer-to-peer systems is to ``to figure out a mechanism and architecture for organizing the peers in such a way so that they can cooperate to provide a useful service to the community of users''~\cite{saroiu2001measurement}. This is similar to the main challenge facing decentralized organizations--a lack of interaction and communication, or in other words, cooperation--which was identified in Section \ref{sec:challenge}. 

With EA being a potential solution to this challenge of decentralization in organizations and the parallels between the domains of peer-to-peer systems and decentralized organizations, we propose that peer-to-peer may be a potential source of principles that could form the basis for evolving current centralization-focused EA frameworks into ones that are supportive of decentralization. This section outlines solutions based on relevant principles from peer-to-peer that have been generated through brainstorming.

\paragraph*{Peer production}

Benkler defines peer production as ``...production systems that depend on individual action that is self-selected and decentralized, rather than hierarchically assigned''~\cite{benkler2006wealth}. Here, individuals act according to their own will rather than being directed by a central figure. Peer production works on the idea of the individuals willingly coordinating with one another by expressing their own views while understanding the views of others.

Peer production takes many different forms. One example are user-driven media sites such as Reddit\footnote{~\url{www.reddit.com}} and Slashdot\footnote{~\url{www.slashdot.org}}, which follow a peer-production model for producing ``relevance/accreditation''~\cite{benkler2006wealth} on user-submitted content. On these sites, the users have the ability to vote on the submitted content in order to decide on the content's relevance or credibility. Another example of relevance production are crowdfunding sites such as Kickstarter\footnote{~\url{www.kickstarter.com}} where individuals decide on the funding of user-submitted projects by giving their own money. Peer production is also used to produce content, such as in the case of Wikipedia\footnote{~\url{www.wikipedia.org}}, an online encyclopedia which provides a platform for user-driven content submission and change management of that content.

If we view enterprises as being composed of peers (a peer could be individual or an organizational unit), the idea of peer production becomes useful for EA. For example, the EA  Engine of TOGAF relies on an Architecture Board responsible high-level decisions and governance. Instead of a central board responsible for making decisions, a model based on the principle of peer production for relevance/accreditation could be used instead. This would better support decentralization as decision making would then be distributed amongst the peers that make the organization.

\paragraph*{Trust management in peer-to-peer}

Due to the fact that peers in peer-to-peer systems are able to operate in a completely independent manner, there exists the problem of knowing whether or not the contribution made by a peer is trustworthy or not. Consequently, some researchers have proposed various methods for determining trust in a peer-to-peer environment.

For example, Aberer and Despotovic~\cite{aberer2001managing} have proposed determining whether a peer is trustworthy or not based on a peers history of interactions with other peers in the system. This assessment is performed by the individual peers, and as such, is appropriate for a peer-to-peer environment.

TOGAF employs the idea of an approval process grounded on the presence of centralized authority. This is to ensure that the presented architectural material is in fact valid for the enterprise. In a decentralized environment, this central authority is not likely to exist. Peer-to-peer trust management may offer a solution here. Instead of being give an explicit stamp of approval, the acceptance of a peer's contribution to EA by other peers can be based on a peer's level of trustworthiness. 

\subsection{Iteration 2}
\label{sec:design_iteration2}

\subsubsection*{Search and Select}

As the problem outlined in ``Iteration 2'' Section~\ref{sec:exproblem} relates primarily to the decision making aspect of the university's EA engine, \textbf{the principle of peer production -- which relates to the governance of a decentralized group of people striving towards a common goal -- will be used to create the solution artifact}.  The principle of trust management in peer-to-peer would be useful in determining whether some produced content (i.e. an EA artifact) is of sufficient quality to be included in the overall architecture, but is not as applicable to the problem of a mismatch between decision making structures. 

As highlighted in table~\ref{table:summary}, EA frameworks often assume vertical coordination, for example the Architecture Board of TOGAF. Here, some central authority is responsible for the overall decision making with respect to the architecture. This thesis instead proposes a solution artifact that can provide support for an EA that supports the horizontal coordination that is characteristic of decentralized organizations.

\paragraph*{The Solution Artifact}

The solution artifact is composed of three peer production based guidelines aimed at providing increased support for decentralization in EA:

\begin{description}
\item[1. willful coordination] \hfill \\ Instead of strictly defined formal compliance measures and roles, operate under the principle that the architecture requires willful coordination by individual entities. This is based on the idea from peer production that individuals willingly coordinate with one another for the purpose of mutual benefit.

\item[2. Decentralized authority structure] \hfill \\ Allow for operational departments to make decisions for themselves about their relevant areas, thus granting them freedom to do as they see best. With respect to IT, for example, an individual business unit should have the authority to decide which systems they should develop themselves, and when they should collaborate with the rest of the organization. 

\item[3. Peer decision making] \hfill \\ Decision making should be distributed to the affected individuals as opposed to residing with an top-level entity. This can be done by giving everyone the right to vote on a decision. Since it is however not feasible to vote on all decisions, such voting should be reserved more for major decisions. 
\end{description}