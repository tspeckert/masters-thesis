\subsection{Iteration 1}

\subsubsection*{Generate}

The challenge of decentralization is not a new one; other efforts have been able to address their view on it with success. The specifics of the challenge varies between domains, however there may exist general principles that can be taken and applied to EA. 

One such effort is peer-to-peer architecture. According to Saroiu, Gummadi and Gribble, peer-to-peer systems ``...typically lack dedicated, centralized infrastructure, but rather depend on the voluntary participation of peers to contribute resources out of which the infrastructure is constructed. Membership in a peer-to-peer system is ad-hoc and dynamic...''~\cite{saroiu2001measurement}. 

%Multiple parallels can be drawn between the problem decentralization for peer-to-peer systems and the problem of decentralization for EA.

We argue that peer-to-peer is a relevant concept to decentralization in EA for two reasons. First, individuals in highly decentralized organization are able to contribute to the enterprise in a manner that is completely up to them. This is similar to peers in a peer-to-peer system, where the peers participate in a completely voluntary manner. Second, the challenge that peer-to-peer systems overcome is similar to the main challenge faced by decentralized organizations. Saroiu et al. state that the challenge of peer-to-peer systems is to ``to figure out a mechanism and architecture for organizing the peers in such a way so that they can cooperate to provide a useful service to the community of users''~\cite{saroiu2001measurement}. This is similar to the main challenge facing decentralized organizations--a lack of interaction and communication, or in other words, cooperation--which was identified in Section \ref{sec:challenge}. 

With EA being a potential solution to this challenge of decentralization in organizations and the parallels between the domains of peer-to-peer systems and decentralized organizations, we propose that peer-to-peer may be a potential source of principles that could form the basis for evolving current centralization-focused EA frameworks into ones that are supportive of decentralization. This section outlines solutions based on relevant principles from peer-to-peer that have been generated through brainstorming.

%
% Theory: 
%

\paragraph*{Peer production}

Benkler defines peer production as ``...production systems that depend on individual action that is self-selected and decentralized, rather than hierarchically assigned''~\cite{benkler2006wealth}. Here, individuals act according to their own will rather than being directed by a central figure. Peer production works on the idea of the individuals willingly coordinating with one another by expressing their own views while understanding the views of others.

%
%% Examples:
%

Peer production takes many different forms. One example are user-driven media sites such as Reddit\footnote{~\url{www.reddit.com}} and Slashdot\footnote{~\url{www.slashdot.org}}, which follow a peer-production model for producing ``relevance/accreditation''~\cite{benkler2006wealth} on user-submitted content. On these sites, the users have the ability to vote on the submitted content in order to decide on the content's relevance or credibility. Another example of relevance production are crowdfunding sites such as Kickstarter\footnote{~\url{www.kickstarter.com}} where individuals decide on the funding of user-submitted projects by giving their own money. Peer production is also used to produce content, such as in the case of Wikipedia\footnote{~\url{www.wikipedia.org}}, an online encyclopedia which provides a platform for user-driven content submission and change management of that content.

%
%% Why Appropriate:
%

If we view enterprises as being composed of peers (a peer could be individual or an organizational unit), the idea of peer production becomes useful for EA. For example, the EA  Engine of TOGAF relies on an Architecture Board responsible high-level decisions and governance. Instead of a central board responsible for making decisions, a model based on the principle of peer production for relevance/accreditation could be used instead. This would better support decentralization as decision making would then be distributed amongst the peers that make the organization.

%%
%% Theory: 
%%

\paragraph*{Trust management in peer-to-peer}
%
%% Definition:
%
Due to the fact that peers in peer-to-peer systems are able to operate in a completely independent manner, there exists the problem of knowing whether or not the contribution made by a peer is trustworthy or not. Consequently, some researchers have proposed various methods for determining trust in a peer-to-peer environment.
%
%% Examples:
%
For example, Aberer and Despotovic~\cite{aberer2001managing} have proposed determining whether a peer is trustworthy or not based on a peers history of interactions with other peers in the system. This assessment is performed by the individual peers, and as such, is appropriate for a peer-to-peer environment.
%
%% Why Appropriate:
%
TOGAF employs the idea of an approval process grounded on the presence of centralized authority. This is to ensure that the presented architectural material is in fact valid for the enterprise. In a decentralized environment, this central authority is not likely to exist. Peer-to-peer trust management may offer a solution here. Instead of being give an explicit stamp of approval, the acceptance of a peer's contribution to EA by other peers can be based on a peer's level of trustworthiness. 
%
%




% ----


%
%In TOGAF, the concept of an Architecture Repository [CROSS-REFERENCE?] exists as part of the EA Engine in order to manage the storage of all architecture-related information in a central repository throughout the architecture's life cycle. While this has the benefit of making this information easy to retrieve, it is not supportive of a organizational structures on the decentralized end of the spectrum. 
%
%Instead of a central repository of EA artifacts, the Architecture Repository could be treated as a distributed resource that is developed and stored in a completely distributed manner, as is done with Git or other distributed revision control systems. In this way, the decentralized EA contributors could submit their own updates to EA without having to go through a bureaucratic process (e.g. gaining approval) to have their updates included in the overall EA. 
%
%This could also be a relevant principle for FEA in supporting further decentralization in that individual organizational units could contribute to the overall distributed resource of EA as opposed to developing within the confines of set standards. 
%
%

%
% Theory: 
%

%\paragraph*{Distributed Revision Control//distributed collaboration}
%
%
%% Definition:
%
%
%
%
%% Examples:
%
%Revision control systems enable multiple people to contribute to the development of a single project, for example a computer program, by managing any changes made. They generally offer such features as history tracking and the ability for different people to work without interrupting the work of others. A new type of revision control system has emerged, called distributed revision control, which is based on peer-to-peer concepts~\cite{O'Sullivan2009}. Distributed revision control systems, such as Git~\cite{git}, offer the same key features of a traditional revision control system, with the key difference being that each contributing party maintains a complete copy of the project~\cite{O'Sullivan2009}; meaning that there is no need for a central server. 
%
%% Why Appropriate:
%
%
%
%%
%% Theory:
%%
%
%\paragraph*{Distributed Decision Making}
%
%% Definition:
%
%? - definition from theory
%
%% Examples:
%
%%Social media is becoming an increasingly popular trend, especially on the internet as evidenced by the popularity of such sites as Reddit and Kickstarter. Reddit functions on the concepts of user voting on content in order to determine what is the most important~\cite{RedditInc.}. Kickstarter functions in a somewhat similar manner, though here the voting is done with money. In particular, Kickstarter revolves around users submitting projects which are then funded by individuals contributing whatever they want~\cite{KickstarterInc.}.
%%
%%% Why Appropriate:
%
%
%
% 
%%Theory:
%\paragraph*{Co-Design}
%
%%Definition:
%"collective creativity as it is applied across the whole span of a design process" ~\cite{sanders2008co}
%
%%Example: 
%
%Smart Cities is a collaborative project between a number of governments and universities seeking to achieve excellence in the field of e-services~\cite{Cities}. One of their publications explored the concept of co-design and how it contributes to the delivery of e-services.  The general idea of co-design, according to Smart Cities, is take into account the perspectives of all stakeholders (including end-users). 
%
%%Why Appropriate: 
%
%The Zachman Framework, on the other hand, has a limited number of perspectives; none of which are end-users. Integrating the concept of co-design into the Zachman Framework in order to expand on its perspectives might allow it to better support decentralization. 
%

\subsection{Iteration 2}

\subsubsection*{Search and Select}

As the main identified issue in case is with decision making, two governance framework's  addressing this issue have been developed; a decentralized framework based on peer production is shown in tables \ref{table:peerGeneralGovernance} and \ref{table:peerITGovernance}, and a centralized framework is shown in tables \ref{table:centralGeneralGovernance} and \ref{table:centralITGovernance}. To give context to this solution, the current or ``as-is'' governance framework is outlined in tables \ref{table:as-isGeneralGovernance} and \ref{table:as-isITGovernance}.  

\begin{table}[H]
\caption{As-is Governance Framework: General Governance}
\label{table:as-isGeneralGovernance}
%%\begin{tabular}{l p{0.28\textwidth}}
\begin{tabular}{ | p{0.15\textwidth} | p{0.18\textwidth}| p{0.18\textwidth} | p{0.38\textwidth}|}
%
\hline
%
\textbf{Name} & 
\textbf{Relevant Organizational Property} &
\textbf{Centralization} &  
\textbf{Description} \\
%
\hline
%
 Allocation of decision rights & 
 Coordination &
 Centralized & 
 Decision rights are granted by the Swedish government to the University Board and Vice-Chancellor. From here, they are either kept or delegated to the lower levels as depicted in figure \ref{fig:decision}. \\
%
\hline
%
 Decision rights in practice (BETTER WORDING?) & 
 Coordination &
 Decentralized & 
 The university operates under the principle that decision rights are pushed down as close to the operational level as possible. \\
%
\hline
%
%
% &
% &
% &
% Supportive quality check system (though SU doesn't really use it...) \\
%%
%\hline
%%
%
 University board &
 Coordination &
 Centralized &
 Top-level board responsible for university-wide strategic direction setting and overall control economic control. \\
%
\hline
%
 
 Faculty board &
 Coordination &
 Centralized &
 Faculty-level board responsible for faculty-wide strategic direction setting and economic control. \\
%
\hline
%
%
 Department board &
 Coordination &
 Centralized &
 Department-level board responsible for department-wide strategic direction setting and economic control. \\
%
\hline
%
 Budgeting &
 Coordination &
 Centralized &
 Faculty sets funding for department, and the department board controls allocation within the department. If the department is under, leftover funds goes to other departments who go over. If the department is over, they are not guaranteed to be covered for it. \\
%
\hline
%
%
 Performance measurements &
 Coordination &
 Decentralized &
 The university does not employ formal centralized performance measurements for its employees.  \\
%
\hline
%
 Advisory group &
 Communication patterns &
 Decentralized &
 An advisory group composed of members from all aspects of a department give suggestions to the department head. \\
%
\hline
%
%
 Department operating principles &
 Coordination &
 Centralized &
 The department sets specific operating principles  yearly that need to be approved by the Faculty. \\
%
\hline
%
%
 Department strategy &
 Coordination &
 Centralized &
 The department sets general strategy and vision set every 3 years and needs to be approved by the Faculty \\
%
\hline
%%
%%
% &
% &
% &
%  DSV does things because they think they are are interesting (PROTECT THIS) \\
%%
%\hline
%%

\end{tabular}
\end{table}


\begin{table}[H]
\caption{As-is Governance Framework: Information Technology}
\label{table:as-isITGovernance}
%%\begin{tabular}{l p{0.28\textwidth}}
\begin{tabular}{ | p{0.15\textwidth} | p{0.18\textwidth}| p{0.18\textwidth} | p{0.38\textwidth}|}
%
\hline
%
\textbf{Name} & 
\textbf{Relevant Organizational Property} &
\textbf{Centralization} &  
\textbf{Description} \\
%
\hline
%
 Authority structure & 
 Coordination &
 Decentralized  &
 The department and the university have separate IT and the departmental IT does not report to the university. \\% done purely because they wa together frequently (e.g. university-wide domain) as it is mutually beneficial.  \\
%
\hline
%
 IT adoption (department IT)& 
 Coordination &
 Decentralized & 
 Department IT does not dictate all IT used in the department, research projects and centers, for example, can develop and use their own IT systems should they desire. \\

%
\hline
%
%
 Approval (department IT) &
 Coordination &
 Mixed &
 IT projects are run by independently by IT, though they sometimes need approval from the department if they are expensive.  \\
%
%%
\hline
%
 IT collaboration & 
 Coordination &
 Decentralized  &
 Any decision to cooperate with other departments or with the university IT is made by the departmental IT itself and is based on the cooperation resulting in mutual benefit.\\% done purely because they wa together frequently (e.g. university-wide domain) as it is mutually beneficial.  \\
%
%
\hline
%
 Management of essential central systems &
 Coordination &
 Centralized &
 Essential central systems (eg. administrative systems such as HR) for the whole university are controlled by the university board. \\
%
\hline
%
 Management of non-essential central systems &
 Coordination &
 Mixed &
 The department is required to pay for these central IT systems but is not required to use them.  \\
%
\hline
%
%
\end{tabular}
\end{table}

%%%%%%%%%%%%%%%%%%%%%%%
%%%%%%%%%%%%%%%%%%%%%%%
%%%%%%%%%%%%%%%%%%%%%%%
%%%%%%%%%%%%%%%%%%%%%%%
%%%%%%%%%%%%%%%%%%%%%%%
%%%%%%%%%%%%%%%%%%%%%%%
%%%%%%%%%%%%%%%%%%%%%%%
%%%%%%%%%%%%%%%%%%%%%%%

\begin{table}[H]
\caption{Centralized Governance Framework: General Governance}
\label{table:centralGeneralGovernance}
%%\begin{tabular}{l p{0.28\textwidth}}
\begin{tabular}{ | p{0.15\textwidth} | p{0.18\textwidth}| p{0.18\textwidth} | p{0.38\textwidth}|}
%
\hline
%
\textbf{Name} & 
\textbf{Relevant Organizational Property} &
\textbf{Centralization} &  
\textbf{Description} \\
%
\hline
%
 Allocation of decision rights & 
 Coordination &
 Centralized & 
 Decision rights are granted by the Swedish government to the University Board and Vice-Chancellor. From here, they are either kept or delegated to the lower levels as depicted in figure \ref{fig:decision}. \\
%
\hline
%
 \textit{Decision rights in practice (BETTER WORDING?)} & 
 \textit{Coordination} &
 \textit{Centralized} & 
 \textit{A minimum of decision rights are pushed down to the operational level.} \\
%
\hline
%
%
% &
% &
% &
% Supportive quality check system (though SU doesn't really use it...) \\
%%
%\hline
%%
%
 University board &
 Coordination &
 Centralized &
 Top-level board responsible for university-wide strategic direction setting and overall control economic control. \\
%
\hline
%
 
 Faculty board &
 Coordination &
 Centralized &
 Faculty-level board responsible for faculty-wide strategic direction setting and economic control. \\
%
\hline
%
%
 Department board &
 Coordination &
 Centralized &
 Department-level board responsible for department-wide strategic direction setting and economic control. \\
%
\hline
%
 Budgeting &
 Coordination &
 Centralized &
 Faculty sets funding for department, and the department board controls allocation within the department. If the department is under, leftover funds goes to other departments who go over. If the department is over, they are not guaranteed to be covered for it. \\
%
\hline
%
%
 Performance measurements &
 Coordination &
 Decentralized &
 The university does not employ formal centralized performance measurements for its employees.  \\
%
\hline
%
 \textit{Advisory group} &
 \textit{Communication patterns} &
 \textit{Centralized} &
 \textit{An group composed of upper level management from the different areas (e.g. the head of undergraduate studies) of the department advise the department head.} \\
%
\hline
%
%
 Department operating principles &
 Coordination &
 Centralized &
 The department sets specific operating principles  yearly that need to be approved by the Faculty. \\
%
\hline
%
%
 Department strategy &
 Coordination &
 Centralized &
 The department sets general strategy and vision set every 3 years and needs to be approved by the Faculty \\
%
\hline
%%
%%
% &
% &
% &
%  DSV does things because they think they are are interesting (PROTECT THIS) \\
%%
%\hline
%%

\end{tabular}
\end{table}

\begin{table}[H]
\caption{Centralized Governance Framework: Information Technology}
\label{table:centralITGovernance}
%%\begin{tabular}{l p{0.28\textwidth}}
\begin{tabular}{ | p{0.15\textwidth} | p{0.18\textwidth}| p{0.18\textwidth} | p{0.38\textwidth}|}
%
\hline
%
\textbf{Name} & 
\textbf{Relevant Organizational Property} &
\textbf{Centralization} &  
\textbf{Description} \\
%
\hline
%
 \textit{Authority structure} & 
 \textit{Coordination} &
 \textit{Centralized}  &
 \textit{The department IT is a subordinate entity to the university IT.} \\% done purely because they wa together frequently (e.g. university-wide domain) as it is mutually beneficial.  \\
%
\hline
%
 \textit{IT adoption (department IT)}& 
 \textit{Coordination} &
 \textit{Centralized} & 
 \textit{All IT systems used in the department are controlled by the department's IT department.} \\

%
\hline
%
%
 \textit{Approval (department IT)} &
 \textit{Coordination} &
 \textit{Centralized} &
 \textit{Any IT projects need to be approved by the university IT.}  \\
%
\hline
%
%
%
 \textit{IT collaboration} & 
 \textit{Coordination} &
 \textit{Decentralized}  &
 \textit{All cooperation is controlled and managed by the university IT.} \\% Any decision to cooperate with other departments or with the university IT is made by the departmental IT itself and is based on the cooperation resulting in mutual benefit.\\% done purely because they wa together frequently (e.g. university-wide domain) as it is mutually beneficial.  \\
%
\hline
%
 Management of essential central systems &
 Coordination &
 Centralized &
 Essential central systems (e.g. administrative systems such as HR) for the whole university are controlled by the university board. The department is required to pay for and use or interface with these systems. \\
%
\hline
%
 \textit{Management of non-essential central systems} &
 \textit{Coordination} &
 \textit{Centralized} &
 \textit{The university decides whether or not the department is required to pay for and use these central IT systems.}  \\
%
\hline
%
%
\end{tabular}
\end{table}

%%%%%%%%%%%%%%%%%%%%%%%
%%%%%%%%%%%%%%%%%%%%%%%
%%%%%%%%%%%%%%%%%%%%%%%
%%%%%%%%%%%%%%%%%%%%%%%
%%%%%%%%%%%%%%%%%%%%%%%
%%%%%%%%%%%%%%%%%%%%%%%
%%%%%%%%%%%%%%%%%%%%%%%
%%%%%%%%%%%%%%%%%%%%%%%

\begin{table}[H]
\caption{Peer Production Based Governance Framework: General Governance}
\label{table:peerGeneralGovernance}
%%\begin{tabular}{l p{0.28\textwidth}}
\begin{tabular}{ | p{0.15\textwidth} | p{0.18\textwidth}| p{0.18\textwidth} | p{0.38\textwidth}|}
%
\hline
%
\textbf{Name} & 
\textbf{Relevant Organizational Property} &
\textbf{Centralization} &  
\textbf{Description} \\
%
\hline
%
 Allocation of decision rights & 
 Coordination &
 Centralized & 
 Decision rights are granted by the Swedish government to the University Board and Vice-Chancellor. From here, they are either kept or delegated to the lower levels as depicted in figure \ref{fig:decision}. \\
%
\hline
%
 Decision rights in practice & 
 Coordination &
 Decentralized & 
 The university operates under the principle that decision rights are pushed down as close to the operational level as possible. \\
%
\hline
%
%
% &
% &
% &
% Supportive quality check system (though SU doesn't really use it...) \\
%%
%\hline
%%
%
 University board &
 Coordination &
 Centralized &
 Top-level board responsible for university-wide strategic direction setting and overall control economic control. \\
%
\hline
%
 
 Faculty board &
 Coordination &
 Centralized &
 Faculty-level board responsible for faculty-wide strategic direction setting and economic control. \\
%
\hline
%
%
 \textit{Department board} &
 \textit{Coordination} &
 \textit{Centralized} &
 \textit{Department-level board responsible for department-wide strategic direction setting and economic control.} \\
%
\hline
%
 \textit{Budgeting} &
 \textit{Coordination} &
 \textit{Mixed} &
 \textit{Faculty sets funding for department, and the department board controls allocation within the department. The department has complete control over their allocated funds. If the department is under, leftover funds goes to other departments who go over. If the department is over, they are not guaranteed to be covered for it.} \\
%
\hline
%
%
 Performance measurements &
 Coordination &
 Decentralized &
 The university does not employ formal centralized performance measurements for its employees.  \\
%
\hline
%
 Advisory group &
 Communication patterns &
 Decentralized &
 An advisory group composed of members from all aspects of a department give suggestions to the department head. \\
%
\hline
%
%
 \textit{Department operating principles} &
 \textit{Coordination} &
 \textit{Mixed} &
 \textit{Department members collaboratively set specific operating principles for the department on a yearly basis that need to be approved by the Faculty.} \\
%
\hline
%
%
 \textit{Department strategy} &
 \textit{Coordination} &
 \textit{Mixed} &
 \textit{Department members collaboratively set general strategy and vision for the department every three years that need to be approved by the Faculty.} \\
%
\hline
%%
%%
% &
% &
% &
%  DSV does things because they think they are are interesting (PROTECT THIS) \\
%%
%\hline
%%

\end{tabular}
\end{table}


\begin{table}[H]
\caption{As-is Governance Framework: Information Technology}
\label{table:peerITGovernance}
%%\begin{tabular}{l p{0.28\textwidth}}
\begin{tabular}{ | p{0.15\textwidth} | p{0.18\textwidth}| p{0.18\textwidth} | p{0.38\textwidth}|}
%
\hline
%
\textbf{Name} & 
\textbf{Relevant Organizational Property} &
\textbf{Centralization} &  
\textbf{Description} \\
%
\hline
%
 Authority structure & 
 Coordination &
 Decentralized  &
 The department and the university have separate IT and the departmental IT does not report to the university. \\% done purely because they wa together frequently (e.g. university-wide domain) as it is mutually beneficial.  \\
%
\hline
%
 IT adoption (department IT)& 
 Coordination &
 Decentralized & 
 Department IT does not dictate all IT used in the department, research projects and centers, for example, can develop and use their own IT systems should they desire. \\

%
\hline
%
%
 \textit{Approval (department IT)} &
 \textit{Coordination} &
 \textit{Decentralized} &
 \textit{IT projects are run by independently by IT, though they sometimes need approval from the department if they are expensive. This approval is granted collaboratively by the department members.}  \\
%
%%
\hline
%
 IT collaboration & 
 Coordination &
 Decentralized  &
 Any decision to cooperate with other departments or with the university IT is made by the departmental IT itself and is based on the cooperation resulting in mutual benefit.\\ % done purely because they wa together frequently (e.g. university-wide domain) as it is mutually beneficial.  \\
%
\hline
%
 \textit{Management of essential central systems} &
 \textit{Coordination} &
 \textit{Centralized} &
 \textit{Essential central systems (eg. administrative systems such as HR) for the whole university are controlled by the university board.} \\
%
\hline
%
\textit{Management of non-essential central systems} &
 \textit{Coordination} &
 \textit{Mixed} &
 \textit{The department is required to pay for these central IT systems but is not required to use them.}  \\
%
\hline
%
\end{tabular}
\end{table}
%  
%
%
%
%
%
%
%
%
%
%
%

%\subparagraph*{Undergraduate Education}
%
%\begin{table}
%\caption{As-is Governance Framework: Undgraduate Studies}
%\label{table:summary}
%%%\begin{tabular}{l p{0.28\textwidth}}
%\begin{tabular}{ | p{0.15\textwidth} | p{0.18\textwidth}| p{0.18\textwidth} | p{0.38\textwidth}|}
%%
%\hline
%%
%\textbf{Name} & 
%\textbf{Relevant Organizational Property} &
%\textbf{Centralization} &  
%\textbf{Description} \\
%%
%\hline
%%
% & 
% &
%  & 
% Collaboration with outside institutions managed by the department  itself, no SU involvement \\
%
%%
%\hline
%%
% & 
% &
%  &
% Collaboration with other departments/faculties managed entirely by the department \\
%%
%\hline
%%
%%
% &
% &
% &
% Current change structure for studies (come from anywhere -> Director investigates feasibility -> group (made up of relevant ppl) sets up plan); eg. Bologna \\
%%
%\hline
%%
%%
% &
% &
% &
% New programs require approval from faculty; they can't infringe on other faculties \\
%%
%\hline
%%
%%
% &
% &
% &
% Evaluation is generally informal (eg. teacher's get feedback from students), except for state level requirements (e.g. grade distributions) \\
%%
%\hline
%%
%%
% &
% &
% &
% Funding for courses is generally the same as the previous year, but is based around enrollment levels. \\
%%
%\hline
%%
%%
% &
% &
% &
% Collaboration outside of the department is encouraged as it makes the size of the pie bigger. (maybe actually points to decentralized). \\
%%
%\hline
%%
%%
% &
% &
% &
% Performance measures \\
%%
%\hline
%%
%\end{tabular}
%\end{table}
%
%\subparagraph*{Graduate Education}
%
%\begin{table}
%\caption{As-is Governance Framework: Graduate Studies}
%\label{table:summary}
%%%\begin{tabular}{l p{0.28\textwidth}}
%\begin{tabular}{ | p{0.15\textwidth} | p{0.18\textwidth}| p{0.18\textwidth} | p{0.38\textwidth}|}
%%
%\hline
%%
%\textbf{Name} & 
%\textbf{Relevant Organizational Property} &
%\textbf{Centralization} &  
%\textbf{Description} \\
%%
%\hline
%%
% & 
% &
%  & 
% Basic guidelines come from the university (basic course requirements and the number of PhDs per professor), but the rest is managed by the department.  \\
%
%%
%\hline
%%
% & 
% &
%  &
% Funding a combination of internal (regulated by the university) and external (mostly self-managed, though the department maintains some economic control) \\
%%
%\hline
%%
%%
% &
% &
% &
% The department is responsible for quality assurance.  \\
%%
%\hline
%%
%%
% &
% &
% &
% Research projects are self-managed (some economic controls exist) \\
%%
%\hline
%%
%%
% &
% &
% &
% DSV does things because they think they are are interesting (PROTECT THIS) \\
%%
%\hline
%%
%%
% &
% &
% &
% Supervisors are responsible for their PhD students \\
%%
%\hline
%%
%%
% &
% &
% &
% PhD projects rely on a formal plan that is approved by the faculty. Supervisor and student should be solely responsible for the plan, but in reality the Director of Doctoral Studies needs to give feedback.  \\
%%
%\hline
%%
%%
% &
% &
% &
% The department prescribes yearly progress checks where the plan is updated and approved.  \\
%%c  
%\hline
%%
%%
% &
% &
% &
% Department tracks costs of all projects using a central system. \\
%%
%\hline
%%
%%
% &
% &
% &
% Department defines the responsibilities of the various actors (e.g. supervisor) itself. \\
%%
%%
% &
% &
% &
%   \\
%%
%\hline
%%
%\end{tabular}
%\end{table}