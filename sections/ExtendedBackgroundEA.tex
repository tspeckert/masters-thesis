% The thesis should have an introduction to the research area to which the thesis belongs. The thesis should have the author’s own discussion about previous solutions to the problem that is being addressed. The thesis should also have the author’s own discussion about which scientific articles that the thesis builds upon and a motivation as to why these were chosen. Relevant theories should be discussed for the problem and research question, but or a Bachelor’s thesis it is not necessary that it is based on such theories.

% Requirement for 1 point: that the thesis provides a base for the topic of the thesis based on previous scientific research. The area within computer and system sciences to which the thesis contributes should also be named by presenting the scientific research to which the thesis refers. 

% For 2 points the following is also required: that a deep and critical discussion is made about how the thesis builds upon previous scientific research. 

% For 3 points the following is also required: that a systematic and comprehensive literature study is presented that is the basis for placing and evaluating the research contribution of the thesis in a scientific context.


\subsection{Overview}
According to Sessions~\cite{sessions2007}, the field of Enterprise Architecture (EA) emerged in order to combat two increasingly prevalent problems facing enterprises: system complexity and business-IT alignment. As enterprises rely more and more on information systems of increasing complexity, these problems become even more important. The field of EA views the solution to these problems to be one of concurrent design. It is not enough simply try and fit IT to the business; business and IT aspects should be designed concurrently. 

While there is no singular agreed-upon definition for EA, different definitions[gartner, jelena, MIT, TOGAF, lankhorst] do have much in common. EA is a discipline that takes a holistic approach to transforming high-level business vision and goals into the integration of an enterprise's organizational structure, business processes, and information systems. This transformation involves identifying and implementing the necessary change for this to occur. The various EA frameworks aim to accomplish this through some or all of the following three phases [ACTUALLY THIS IS HOW WE ARE ANALYZING THEM, NOT NECESSARILY HOW THEY ARE ORGANIZED]: the preliminary phase, the EA description phase, and the EA engine phase. 

The preliminary phase aims to lay the groundwork for the EA process. Typically, this involves setting up teams, ownership, responsibilities and gaining commitment. The EA description phase is involved with creating the actual architecture artifacts (eg. models, principles) themselves. This generally involves outlining an enterprises EA principles, its as-is EA description,  the planned to-be architecture, an analysis of the gap between the as-is and to-be architectures and a plan to cross that gap. The engine  phase involves setting up a support structure for ensuring the ongoing adoption of the to-be EA description. This can involve gaining commitment from stakeholders, setting up some compliance checking procedures, and deciding upon a prioritization of tasks to be completed. The remainder of this section will look at 4 [NEED THE REAL NUMBER HERE] different EA frameworks from the perspective of of these three phases. 



% *Definition*
% 
% (1) http://www.gartner.com/it-glossary/enterprise-architecture-ea/
% Enterprise architecture (EA) is a discipline for proactively and holistically leading enterprise responses to disruptive forces by identifying and analyzing the execution of change toward desired business vision and outcomes. EA delivers value by presenting business and IT leaders with signature-ready recommendations for adjusting policies and projects to achieve target business outcomes that capitalize on relevant business disruptions. EA is used to steer decision making toward the evolution of the future state architecture.
% 
% (2) Nameless Book from Jelena
% Enterprise architecture (EA) is the process of translating business vision and strategy into effective enterprise change by creating, communicating and improving the key requirements, principles and models that describe the enterprise's future state and enable its evolution.
% 
% (3) MIT Center for Information Systems Research (MIT CISR) [but from Jelena's book]
% Enterprise architecture is the organizing logic for business processes and IT infrastructure reflecting the integration and standardization requirements of the company's operating model. The operating model is the desired state of business process integration and business process standardization for delivering goods and services to customers.

% (4) TOGAF
% The purpose of enterprise architecture is to optimize across the enterprise the often fragmented legacy of processes (both manual and automated) into an integrated environment that is responsive to change and supportive of the delivery of the business strategy.
% 
% (5)~\cite{lankhorst2009}
% a coherent whole of principles, methods and models that are used in the design and realisation of an enterprise's organisational structure, business processes, information systems, and infrastructure

% (6) 
% Op't Land, M.; Proper, E.; Waage, M.; Cloo, J.; Steghuis, C. (2009): Enterprise Architecture: Creating Value by Informed Governance. 1 ed. Springer, Berlin.
% “a continuous process involving the creation, modification, enforcement, application, and dissemination of different results. This process should be in sync with developments in the environment of the enterprise as well as developments internal to the enterprise, including both its strategy and its operational processes” [18].

\subsection{The Open Group Architecture Framework (TOGAF)}
The Open Group Architecture Framework, more commonly know as TOGAF, is a freely available EA framework created by The Open Group~\cite{togaf9.1}, a consortium of IT organizations. [SOMETHING ABOUT WHETHER A CONSORTIUM OF IT COMPANIES IS GOOD FOR INNOVATION, PERHAPS MORE FOR STANDARDS]. 

TOGAF is comprised of a number of differen aspects, mainly: the Architecture Development Method (ADM), "a method for developing and managing the lifecycle of an enterprise
architecture"~cite{togaf9.1}; the Architecture Content Framework, the companion to the ADM which describes the content of the architecture description; and the Enterprise Continuum, which shows that architectures should be developed across a continuum that ranges from generic to organisation-specific. 

[SOMETHING ABOUT FLEXIBILITY]

ADM, Continuum,  \cite{lankhorst2009}

Process
- ADM phases

architecture definition
- ADM phases (maybe not actually)
- 4 types of architectures
- Continuum
- Repository (?)
- 

engine
- ADM phases
- Repository (?)


\subsection{The Zachman Framework}
Intro
- only taxonomy
- started in 1987 (?), first EA

architecture definition
- entirety if ZF

\subsection{Federal Enterprise Architecture (FEA)}
FEA

\subsection{A Reference Architecture for Collaborative Networks (ARCON)}
