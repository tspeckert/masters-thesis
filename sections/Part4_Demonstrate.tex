\subsection{Iteration 1}

The ideas presented in this thesis work were adapted into two short papers that were submitted to two conferences, PoEM\footnote{\url{http://poem2013.rtu.lv}} 2013 and the TEAR workshop at IEEE EDOC\footnote{\url{http://planet-sl.org/edoc2013/}} 2013: \cite{speckert2013} and \cite{rychkova2013}. The main feedback from these conferences agreed with the problem explication. There was an general agreement that the difference in communication and decision making management between centralized and decentralized organizations required some new governance models in the field of EA. 

%- after I presented some "false" decisions at DSV, some from TEAR; and
%some recent, an agreement was that decentralized environments require
%adequate communication and decision making management, which is not the
%same as for centralized organizations.
%- a set of patterns for IT governance could be established, while then
%there is a need to measure relevant parameters to be able to switch from
%one IT governance pattern to another
%- Back to PoEM, in the
%audience we have kind of strong researchers in EM from Norway, Sweden,
%etc. - nobody neglected the need for "new ways" of governance to fit to
%DEA. One relevant person asked me "will this be a new EA model", and I
%said that intention was to complement exiting ones.
%

\subsection{Iteration 2}

In order to apply peer production to the case, it is important to understand the current or ``as-is'' situation with the case and to have some context in which to compare the presented solution. This has been accomplished by first outlining the current governance framework (focusing on IT) and by then developing a centralized governance framework. A governance framework based on peer production is then outlined and compared to the as-is and centralized frameworks. 

\subsubsection*{As-Is Governance Framework}

As the main identified issue in case is with decision making, two governance framework's  addressing this issue have been developed; a decentralized framework based on peer production is shown in tables \ref{table:peerGeneralGovernance} and \ref{table:peerITGovernance}, and a centralized framework is shown in tables \ref{table:centralGeneralGovernance} and \ref{table:centralITGovernance}. To give context to this solution, the current or ``as-is'' governance framework is outlined in tables \ref{table:as-isGeneralGovernance} and \ref{table:as-isITGovernance}.  

\begin{center}
%%\begin{longtable}{l p{0.28\textwidth}}
\begin{longtable}{ | p{0.15\textwidth} | p{0.18\textwidth}| p{0.18\textwidth} | p{0.38\textwidth}|}
\caption{As-Is Governance Framework: General Governance} \label{table:as-isGeneralGovernance} \\
%
\hline
\textbf{Name} & 
\textbf{Relevant Organizational Property} &
\textbf{Centralization} &  
\textbf{Description} \\ \hline
\endfirsthead
%
\multicolumn{4}{c}{\textit{Table \ref{table:as-isGeneralGovernance} -- continued from previous page}} \\  
\hline
\textbf{Name} & 
\textbf{Relevant Organizational Property} &
\textbf{Centralization} &  
\textbf{Description} \\ \hline
\endhead
%
 Allocation of decision rights & 
 Coordination &
 Centralized & 
 Decision rights are granted by the Swedish government to the University Board and Vice-Chancellor. From here, they are either kept or delegated to the lower levels as depicted in Figure~\ref{fig:dsv_decisionRights}\footref{fn:document}\footref{fn:interviewHead}. \\
%
\hline
%
 Decision rights in practice & 
 Coordination &
 Decentralized & 
 The university operates under the principle that decision rights are pushed down as close to the operational level as possible\footref{fn:document}\footref{fn:interviewHead}. \\
%
\hline
%
%
% &
% &
% &
% Supportive quality check system (though SU doesn't really use it...) \\
%%
%\hline
%%
%
 University board &
 Coordination &
 Centralized &
 Top-level board responsible for university-wide strategic direction setting and overall control economic control\footref{fn:document}. \\
%
\hline
%
 
 Faculty board &
 Coordination &
 Centralized &
 Faculty-level board responsible for faculty-wide strategic direction setting and economic control\footref{fn:document}. \\
%
\hline
%
%
 Department board &
 Coordination &
 Centralized &
 Department-level board responsible for department-wide strategic direction setting and economic control\footref{fn:document}. \\
%
\hline
%
 Budgeting &
 Coordination &
 Centralized &
 Faculty sets funding for department, and the department board controls allocation within the department. If the department is under, leftover funds goes to other departments who go over. If the department is over, they are not guaranteed to be covered for it\footref{fn:interviewHead}\footref{fn:document}. \\
%
\hline
%
%
 Performance measurements &
 Coordination &
 Decentralized &
 The university does not employ formal centralized performance measurements for its projects\footref{fn:interviewHead}\footref{fn:interviewIT}.  \\
%
\hline
%
 Advisory group &
 Communication patterns &
 Decentralized &
 An advisory group composed of members from all aspects of a department give suggestions to the department head\footref{fn:interviewHead}\footref{fn:document}. \\
%
\hline
%
%
 Department operating principles &
 Coordination &
 Centralized &
 The department sets specific operating principles  yearly that need to be approved by the Faculty\footref{fn:interviewHead}. \\
%
\hline
%
%
 Department strategy &
 Coordination &
 Centralized &
 The department sets general strategy and vision set every 3 years and needs to be approved by the Faculty\footref{fn:interviewHead}. \\
%
\hline
%%
%%
% &
% &
% &
%  DSV does things because they think they are are interesting (PROTECT THIS) \\
%%
%\hline
%%

\end{longtable}
\end{center}


\begin{center}
%%\begin{longtable}{l p{0.28\textwidth}}
\begin{longtable}{ | p{0.15\textwidth} | p{0.18\textwidth}| p{0.18\textwidth} | p{0.38\textwidth}|}
%\begin{longtable}{ | p{0.15\textwidth} | p{0.18\textwidth}| p{0.18\textwidth} | p{0.38\textwidth}|}
\caption{As-is Governance Framework: Information Technology} \label{table:as-isITGovernance} \\
%
\hline
\textbf{Name} & 
\textbf{Relevant Organizational Property} &
\textbf{Centralization} &  
\textbf{Description} \\ \hline
\endfirsthead
%
\multicolumn{4}{c}{\textit{Table \ref{table:as-isITGovernance} -- continued from previous page}} \\  
\hline
\textbf{Name} & 
\textbf{Relevant Organizational Property} &
\textbf{Centralization} &  
\textbf{Description} \\ \hline
\endhead
%
 Authority structure & 
 Coordination &
 Decentralized  &
 The department and the university have separate IT and the departmental IT does not report to the university\footref{fn:interviewHead}\footref{fn:interviewIT}. \\% done purely because they wa together frequently (e.g. university-wide domain) as it is mutually beneficial.  \\
%
\hline
%
 IT adoption (department IT)& 
 Coordination &
 Decentralized & 
 Department IT does not dictate all IT used in the department, research projects and centers, for example, can develop and use their own IT systems should they desire \footref{fn:interviewPHD}\footref{fn:interviewIT}. \\
%
\hline
%
%
 Approval (department IT) &
 Coordination &
 Mixed &
 IT projects are run by independently by IT, though they sometimes need approval from the department if they are expensive\footref{fn:interviewIT}.  \\
%
%%
\hline
%
 IT collaboration & 
 Coordination &
 Decentralized  &
 Any decision to cooperate with other departments or with the university IT is made by the departmental IT itself and is based on the cooperation resulting in mutual benefit\footref{fn:interviewIT}.\\% done purely because they wa together frequently (e.g. university-wide domain) as it is mutually beneficial.  \\
%
%
\hline
%
 Management of essential central systems &
 Coordination &
 Centralized &
 Essential central systems (eg. administrative systems such as HR) for the whole university are controlled by the university board\footref{fn:interviewIT}. \\
%
\hline
%
 Management of non-essential central systems &
 Coordination &
 Mixed &
 The department is required to pay for these central IT systems but is not required to use them\footref{fn:interviewIT}.  \\
%
\hline
%
%
\end{longtable}
\end{center}

%%%%%%%%%%%%%%%%%%%%%%%
%%%%%%%%%%%%%%%%%%%%%%%
%%%%%%%%%%%%%%%%%%%%%%%
%%%%%%%%%%%%%%%%%%%%%%%
%%%%%%%%%%%%%%%%%%%%%%%
%%%%%%%%%%%%%%%%%%%%%%%
%%%%%%%%%%%%%%%%%%%%%%%
%%%%%%%%%%%%%%%%%%%%%%%

\subsubsection*{Centralized Governance Framework}


\begin{center}
%%\begin{longtable}{l p{0.28\textwidth}}
\begin{longtable}{ | p{0.15\textwidth} | p{0.18\textwidth}| p{0.18\textwidth} | p{0.38\textwidth}|}
\caption{Centralized Governance Framework: General Governance} \label{table:centralGeneralGovernance} \\
%
\hline
\textbf{Name} & 
\textbf{Relevant Organizational Property} &
\textbf{Centralization} &  
\textbf{Description} \\ \hline
\endfirsthead
%
\multicolumn{4}{c}{\textit{Table \ref{table:centralGeneralGovernance} -- continued from previous page}} \\  
\hline
\textbf{Name} & 
\textbf{Relevant Organizational Property} &
\textbf{Centralization} &  
\textbf{Description} \\ \hline
\endhead
%
 Allocation of decision rights & 
 Coordination &
 Centralized & 
 Decision rights are granted by the Swedish government to the University Board and Vice-Chancellor. From here, they are either kept or delegated to the lower levels as depicted in figure \ref{fig:decision}. \\
%
\hline
%
 \textit{Decision rights in practice} & 
 \textit{Coordination} &
 \textit{Centralized} & 
 \textit{A minimum of decision rights are pushed down to the operational level.} \\
%
\hline
%
%
% &
% &
% &
% Supportive quality check system (though SU doesn't really use it...) \\
%%
%\hline
%%
%
 University board &
 Coordination &
 Centralized &
 Top-level board responsible for university-wide strategic direction setting and overall control economic control. \\
%
\hline
%
 
 Faculty board &
 Coordination &
 Centralized &
 Faculty-level board responsible for faculty-wide strategic direction setting and economic control. \\
%
\hline
%
%
 Department board &
 Coordination &
 Centralized &
 Department-level board responsible for department-wide strategic direction setting and economic control. \\
%
\hline
%
 Budgeting &
 Coordination &
 Centralized &
 Faculty sets funding for department, and the department board controls allocation within the department. If the department is under, leftover funds goes to other departments who go over. If the department is over, they are not guaranteed to be covered for it. \\
%
\hline
%
%
 Performance measurements &
 Coordination &
 Decentralized &
 The university does not employ formal centralized performance measurements for its projects.  \\
%
\hline
%
 \textit{Advisory group} &
 \textit{Communication patterns} &
 \textit{Centralized} &
 \textit{A group composed of upper level management from the different areas (e.g. the head of undergraduate studies) of the department advise the department head.} \\
%
\hline
%
%
 Department operating principles &
 Coordination &
 Centralized &
 The department sets specific operating principles  yearly that need to be approved by the Faculty. \\
%
\hline
%
%
 Department strategy &
 Coordination &
 Centralized &
 The department sets general strategy and vision set every 3 years and needs to be approved by the Faculty \\
%
\hline
%%
%%
% &
% &
% &
%  DSV does things because they think they are are interesting (PROTECT THIS) \\
%%
%\hline
%%

\end{longtable}
\end{center}

\begin{center}
%%\begin{longtable}{l p{0.28\textwidth}}
\begin{longtable}{ | p{0.15\textwidth} | p{0.18\textwidth}| p{0.18\textwidth} | p{0.38\textwidth}|}
\caption{Centralized Governance Framework: Information Technology} \label{table:centralITGovernance}\\
%
\hline
\textbf{Name} & 
\textbf{Relevant Organizational Property} &
\textbf{Centralization} &  
\textbf{Description} \\ \hline
\endfirsthead
%
\multicolumn{4}{c}{\textit{Table \ref{table:centralITGovernance} -- continued from previous page}} \\  
\hline
\textbf{Name} & 
\textbf{Relevant Organizational Property} &
\textbf{Centralization} &  
\textbf{Description} \\ \hline
\endhead
%
 \textit{Authority structure} & 
 \textit{Coordination} &
 \textit{Centralized}  &
 \textit{The department IT is a subordinate entity to the university IT.} \\% done purely because they wa together frequently (e.g. university-wide domain) as it is mutually beneficial.  \\
%
\hline
%
 \textit{IT adoption (department IT)}& 
 \textit{Coordination} &
 \textit{Centralized} & 
 \textit{All IT systems used in the department are controlled by the department's IT department.} \\

%
\hline
%
%
 \textit{Approval (department IT)} &
 \textit{Coordination} &
 \textit{Centralized} &
 \textit{Any IT projects need to be approved by the university IT.}  \\
%
\hline
%
%
%
 \textit{IT collaboration} & 
 \textit{Coordination} &
 \textit{Decentralized}  &
 \textit{All cooperation is controlled and managed by the university IT.} \\% Any decision to cooperate with other departments or with the university IT is made by the departmental IT itself and is based on the cooperation resulting in mutual benefit.\\% done purely because they wa together frequently (e.g. university-wide domain) as it is mutually beneficial.  \\
%
\hline
%
 Management of essential central systems &
 Coordination &
 Centralized &
 Essential central systems (e.g. administrative systems such as HR) for the whole university are controlled by the university board. The department is required to pay for and use or interface with these systems. \\
%
\hline
%
 \textit{Management of non-essential central systems} &
 \textit{Coordination} &
 \textit{Centralized} &
 \textit{The university decides whether or not the department is required to pay for and use these central IT systems.}  \\
%
\hline
%
%
\end{longtable}
\end{center}

%%%%%%%%%%%%%%%%%%%%%%%
%%%%%%%%%%%%%%%%%%%%%%%
%%%%%%%%%%%%%%%%%%%%%%%
%%%%%%%%%%%%%%%%%%%%%%%
%%%%%%%%%%%%%%%%%%%%%%%
%%%%%%%%%%%%%%%%%%%%%%%
%%%%%%%%%%%%%%%%%%%%%%%
%%%%%%%%%%%%%%%%%%%%%%%

\subsubsection*{Peer Production Based Governance Framework}

\begin{center}
\begin{longtable}{ | p{0.15\textwidth} | p{0.18\textwidth}| p{0.18\textwidth} | p{0.38\textwidth}|}
\caption{Peer Production Based Governance Framework: General Governance} \label{table:peerGeneralGovernance} \\
%
\hline
\textbf{Name} & 
\textbf{Relevant Organizational Property} &
\textbf{Centralization} &  
\textbf{Description} \\ \hline
\endfirsthead
%
\multicolumn{4}{c}{\textit{Table \ref{table:peerGeneralGovernance} -- continued from previous page}} \\  
\hline
\textbf{Name} & 
\textbf{Relevant Organizational Property} &
\textbf{Centralization} &  
\textbf{Description} \\ \hline
\endhead
%
 Allocation of decision rights & 
 Coordination &
 Centralized & 
 Decision rights are granted by the Swedish government to the University Board and Vice-Chancellor. From here, they are either kept or delegated to the lower levels as depicted in figure \ref{fig:decision}. \\
%
\hline
%
 Decision rights in practice & 
 Coordination &
 Decentralized & 
 The university operates under the principle that decision rights are pushed down as close to the operational level as possible. \\
%
\hline
%
%
% &
% &
% &
% Supportive quality check system (though SU doesn't really use it...) \\
%%
%\hline
%%
%
 University board &
 Coordination &
 Centralized &
 Top-level board responsible for university-wide strategic direction setting and overall control economic control. \\
%
\hline
%
 
 Faculty board &
 Coordination &
 Centralized &
 Faculty-level board responsible for faculty-wide strategic direction setting and economic control. \\
%
\hline
%
%
 \textit{Department board} &
 \textit{Coordination} &
 \textit{Centralized} &
 \textit{Department-level board responsible for department-wide budgeting.} \\
%
\hline
%
 \textit{Budgeting} &
 \textit{Coordination} &
 \textit{Mixed} &
 \textit{Faculty sets funding for department, and the department board controls allocation within the department. The department has complete control over their allocated funds. If the department is under, leftover funds goes to other departments who go over. If the department is over, they are not guaranteed to be covered for it.} \\
%
\hline
%
%
 Performance measurements &
 Coordination &
 Decentralized &
 The university does not employ formal centralized performance measurements for its projects.  \\
%
\hline
%
 Advisory group &
 Communication patterns &
 Decentralized &
 An advisory group composed of members from all aspects of a department give suggestions to the department head. \\
%
\hline
%
%
 \textit{Department operating principles} &
 \textit{Coordination} &
 \textit{Mixed} &
 \textit{Department members collaboratively set specific operating principles for the department on a yearly basis that need to be approved by the Faculty.} \\
%
\hline
%
%
 \textit{Department strategy} &
 \textit{Coordination} &
 \textit{Mixed} &
 \textit{Department members collaboratively set general strategy and vision for the department every three years that need to be approved by the Faculty.} \\
%
\hline
%%
%%
% &
% &
% &
%  DSV does things because they think they are are interesting (PROTECT THIS) \\
%%
%\hline
%%

\end{longtable}
\end{center}


\begin{center}
%%\begin{longtable}{l p{0.28\textwidth}}
\begin{longtable}{ | p{0.15\textwidth} | p{0.18\textwidth}| p{0.18\textwidth} | p{0.38\textwidth}|}
\caption{Peer Production Based Governance Framework: Information Technology} \label{table:peerITGovernance} \\
%
\hline
\textbf{Name} & 
\textbf{Relevant Organizational Property} &
\textbf{Centralization} &  
\textbf{Description} \\ \hline
\endfirsthead
%
\multicolumn{4}{c}{\textit{Table \ref{table:peerITGovernance} -- continued from previous page}} \\  
\hline
\textbf{Name} & 
\textbf{Relevant Organizational Property} &
\textbf{Centralization} &  
\textbf{Description} \\ \hline
\endhead
%
 Authority structure & 
 Coordination &
 Decentralized  &
 The department and the university have separate IT and the departmental IT does not report to the university. \\% done purely because they wa together frequently (e.g. university-wide domain) as it is mutually beneficial.  \\
%
\hline
%
 IT adoption (department IT)& 
 Coordination &
 Decentralized & 
 Department IT does not dictate all IT used in the department, research projects and centers, for example, can develop and use their own IT systems should they desire. \\

%
\hline
%
%
 \textit{Approval (department IT)} &
 \textit{Coordination} &
 \textit{Decentralized} &
 \textit{IT projects are run by independently by IT, though they sometimes need approval from the department if they are expensive. This approval is granted collaboratively by the department members.}  \\
%
%%
\hline
%
 IT collaboration & 
 Coordination &
 Decentralized  &
 Any decision to cooperate with other departments or with the university IT is made by the departmental IT itself and is based on the cooperation resulting in mutual benefit.\\ % done purely because they wa together frequently (e.g. university-wide domain) as it is mutually beneficial.  \\
%
\hline
%
 \textit{Management of essential central systems} &
 \textit{Coordination} &
 \textit{Mixed} &
 \textit{Classification of what is an essential central system is determined collaboratively by the university departments, with each department getting a vote. For systems deemed to be essential, the departments are required to either use the system or interface with it.} \\ %TODO Discuss why %Essential central systems (e.g. administrative systems such as HR) for the whole university are controlled by the university board.
%
\hline
%
\textit{Management of non-essential central systems} &
 \textit{Coordination} &
 \textit{Mixed} &
 \textit{Departments have the choice to opt-in to these systems. Those who choose not to opt-in do not pay for them.}  \\
%
\hline
%
\end{longtable}
\end{center}
%  
%
%
%
%
%
%
%
%
%
%
%

%\subparagraph*{Undergraduate Education}
%
%\begin{center}
%\caption{As-is Governance Framework: Undgraduate Studies}
%\label{table:summary}
%%%\begin{longtable}{l p{0.28\textwidth}}
%\begin{longtable}{ | p{0.15\textwidth} | p{0.18\textwidth}| p{0.18\textwidth} | p{0.38\textwidth}|}
%%
%\hline
%%
%\textbf{Name} & 
%\textbf{Relevant Organizational Property} &
%\textbf{Centralization} &  
%\textbf{Description} \\
%%
%\hline
%%
% & 
% &
%  & 
% Collaboration with outside institutions managed by the department  itself, no SU involvement \\
%
%%
%\hline
%%
% & 
% &
%  &
% Collaboration with other departments/faculties managed entirely by the department \\
%%
%\hline
%%
%%
% &
% &
% &
% Current change structure for studies (come from anywhere -> Director investigates feasibility -> group (made up of relevant ppl) sets up plan); eg. Bologna \\
%%
%\hline
%%
%%
% &
% &
% &
% New programs require approval from faculty; they can't infringe on other faculties \\
%%
%\hline
%%
%%
% &
% &
% &
% Evaluation is generally informal (eg. teacher's get feedback from students), except for state level requirements (e.g. grade distributions) \\
%%
%\hline
%%
%%
% &
% &
% &
% Funding for courses is generally the same as the previous year, but is based around enrollment levels. \\
%%
%\hline
%%
%%
% &
% &
% &
% Collaboration outside of the department is encouraged as it makes the size of the pie bigger. (maybe actually points to decentralized). \\
%%
%\hline
%%
%%
% &
% &
% &
% Performance measures \\
%%
%\hline
%%
%\end{longtable}
%\end{center}
%
%\subparagraph*{Graduate Education}
%
%\begin{center}
%\caption{As-is Governance Framework: Graduate Studies}
%\label{table:summary}
%%%\begin{longtable}{l p{0.28\textwidth}}
%\begin{longtable}{ | p{0.15\textwidth} | p{0.18\textwidth}| p{0.18\textwidth} | p{0.38\textwidth}|}
%%
%\hline
%%
%\textbf{Name} & 
%\textbf{Relevant Organizational Property} &
%\textbf{Centralization} &  
%\textbf{Description} \\
%%
%\hline
%%
% & 
% &
%  & 
% Basic guidelines come from the university (basic course requirements and the number of PhDs per professor), but the rest is managed by the department.  \\
%
%%
%\hline
%%
% & 
% &
%  &
% Funding a combination of internal (regulated by the university) and external (mostly self-managed, though the department maintains some economic control) \\
%%
%\hline
%%
%%
% &
% &
% &
% The department is responsible for quality assurance.  \\
%%
%\hline
%%
%%
% &
% &
% &
% Research projects are self-managed (some economic controls exist) \\
%%
%\hline
%%
%%
% &
% &
% &
% DSV does things because they think they are are interesting (PROTECT THIS) \\
%%
%\hline
%%
%%
% &
% &
% &
% Supervisors are responsible for their PhD students \\
%%
%\hline
%%
%%
% &
% &
% &
% PhD projects rely on a formal plan that is approved by the faculty. Supervisor and student should be solely responsible for the plan, but in reality the Director of Doctoral Studies needs to give feedback.  \\
%%
%\hline
%%
%%
% &
% &
% &
% The department prescribes yearly progress checks where the plan is updated and approved.  \\
%%c  
%\hline
%%
%%
% &
% &
% &
% Department tracks costs of all projects using a central system. \\
%%
%\hline
%%
%%
% &
% &
% &
% Department defines the responsibilities of the various actors (e.g. supervisor) itself. \\
%%
%%
% &
% &
% &
%   \\
%%
%\hline
%%
%\end{longtable}
%\end{center}