Other domains (e.g. technical) have tackled their version of the problem of decentralization
    \begin{itemize}
    \item Borrow principles from them
    \end{itemize}
    
    Organizations have been operating in a decentralized manner without a concrete "EA"
    \begin{itemize}
    \item Borrow principles from them
    \end{itemize}
    
    "Social Peer-to-Peer"; Kickstarter/Crowdfunding, Reddit/Voting. This is related to some EDOC topics:
    \begin{itemize}
    \item Cross-enterprise collaboration in a world of cloud, \textbf{social} and big data
    \item Social information and innovation networks, social media impact on the enterprise
    \end{itemize}

\subsection{Principles from Peer-to-Peer}

Why P2P

Treat EA description as a distribute resource

Distributed responsibility 

\subsection{Principles from Existing Decentralized Organizations}

Co-design, from SmartCities

Social networks

\subsection{Principles from Organizational Science}

Co-opetition


% Principles:

Distributed resource -> repository
Social media -> Board
Principles -> ?
Co-Design -> Zachman?


Revision control systems enable multiple people to contribute to the development of a single project, for example a computer application. They generally offer such features as history tracking and the ability for different people to work without interrupting the work of others. A new type of revision control system has emerged, called distributed revision control, which is based on peer-to-peer concepts~\cite{O'Sullivan2009}. Distributed revision control systems, such as Git~\cite{SoftwareFreedomConservancy}, offer the same key features of a traditional revision control system, with the key difference being that each contributing party maintains a complete copy of the project~\cite{O'Sullivan2009}; meaning that there is no need for a central server. The relevant principle to take from this is one of a distributed resource that is managed, developed, and stored in a completely distribute manner. This could relate back the concept of an Architecture Repository as stated by TOGAF. Instead of a central repository of EA artifacts, it could be treated as a distributed resource that is managed, developed, and stored in a completely distributed manner. This could also be a relevant principle for FEA in supporting further decentralization in that individual organizational units could contribute to the overall distributed resource of EA as opposed to developing within the confines of set standards. 

Social media is becoming an increasingly popular trend, especially on the internet as evidenced by the popularity of such sites as Reddit and Kickstarter. Reddit functions on the concepts of user voting on content in order to determine what is the most important~\cite{RedditInc.}. Kickstarter functions in a somewhat similar manner, though here the voting is done with money. In particular, Kickstarter revolves around users submitting projects which are then funded by individuals contributing whatever they want~\cite{KickstarterInc.}. This principle of social voting could be relevant to the centralized style of Architecture Governance in TOGAF. Instead of governance based on a top-level board, a more social approach where all relevant individuals have a vote is more supportive of decentralization. 

Smart Cities is a 

