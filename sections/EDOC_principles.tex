Other domains (e.g. technical) offer solutions to utilize the benefits of that domain's version of decentralization. As a result, perhaps they can provide some principles or insights that can be mapped to the domain of EA. 

%
% Theory: 
%

\subsection{Distributed Revision Control}


% Definition:




% Examples:

Revision control systems enable multiple people to contribute to the development of a single project, for example a computer program, by managing any changes made. They generally offer such features as history tracking and the ability for different people to work without interrupting the work of others. A new type of revision control system has emerged, called distributed revision control, which is based on peer-to-peer concepts~\cite{O'Sullivan2009}. Distributed revision control systems, such as Git~\cite{git}, offer the same key features of a traditional revision control system, with the key difference being that each contributing party maintains a complete copy of the project~\cite{O'Sullivan2009}; meaning that there is no need for a central server. 

% Why Appropriate:

Distributed revision control is based around the management of a resource that can be developed, and stored in a completely distribute manner. In TOGAF, the concept of an Architecture Repository [CROSS-REFERENCE?] exists as part of the EA Engine in order to manage the storage of all architecture-related information in a central repository throughout the architecture's life cycle. While this has the benefit of making this information easy to retrieve, it is not supportive of a organizational structures on the decentralized end of the spectrum. 

Instead of a central repository of EA artifacts, the Architecture Repository could be treated as a distributed resource that is developed and stored in a completely distributed manner, as is done with Git or other distributed revision control systems. In this way, the decentralized EA contributors could submit their own updates to EA without having to go through a bureaucratic process (e.g. gaining approval) to have their updates included in the overall EA. 

This could also be a relevant principle for FEA in supporting further decentralization in that individual organizational units could contribute to the overall distributed resource of EA as opposed to developing within the confines of set standards. 

%
% Theory:
%

\subsection{Social media/p2p}

% Definition:

? - definition from theory

% Examples:

Social media is becoming an increasingly popular trend, especially on the internet as evidenced by the popularity of such sites as Reddit and Kickstarter. Reddit functions on the concepts of user voting on content in order to determine what is the most important~\cite{RedditInc.}. Kickstarter functions in a somewhat similar manner, though here the voting is done with money. In particular, Kickstarter revolves around users submitting projects which are then funded by individuals contributing whatever they want~\cite{KickstarterInc.}.

% Why Appropriate:

This principle of social voting could be relevant to the centralized style of Architecture Governance in TOGAF. Instead of governance based on a top-level board, a more social approach where all relevant individuals have a vote is more supportive of decentralization. 


%Theory:
\subsection{Co-Design}

%Definition:
"collective creativity as it is applied across the whole span of a design process" ~\cite{sanders2008co}

%Example: 

Smart Cities is a collaborative project between a number of governments and universities seeking to achieve excellence in the field of e-services~\cite{Cities}. One of their publications explored the concept of co-design and how it contributes to the delivery of e-services.  The general idea of co-design, according to Smart Cities, is take into account the perspectives of all stakeholders (including end-users). 

%Why Appropriate: 

The Zachman Framework, on the other hand, has a limited number of perspectives; none of which are end-users. Integrating the concept of co-design into the Zachman Framework in order to expand on its perspectives might allow it to better support decentralization. 


