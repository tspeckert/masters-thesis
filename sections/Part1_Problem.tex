The problem of business-IT alignment is relevant for all types of organizations, regardless of whether they have a centralized or decentralized structure. Solving it allows all components of an enterprise to operate together in a collaborative manner for the purpose of maximizing overall enterprise efficiency. Enterprise Architecture frameworks outline a formal way in which to solve this problem. This thesis argues, however, that modern EA frameworks are primarily supportive of centralized organizations, and as such, have some shortcomings when applied in decentralized organizations. 

Many modern organizations are becoming increasingly decentralized in order to deal with increasingly dynamic business environments~\cite{fulk1995}. Modern EA frameworks and methodologies need to be able to handle these environments, and while they often can, rapidly changing business conditions have been identified as a critical problem in EA~\cite{kaisler2005ea,lucke2010critical}. In \cite[Ch. 1]{Bente2012}, situations were identified where ``EA fails to keep pace with the speed of change in modern business'' and ``companies that, despite having a fully institutionalized EA in place, were in a state close to paralysis''. In \cite{najafi2010kasra}, Najafi and Baraani state that a main challenge of existing EA frameworks is that they are ``inflexible to perceive business changes or opportunities and then change appropriately to adapt and adopt these changes''.

Furthermore, current EA frameworks rely on some organizational properties that are becoming obsolete with progressive decentralization. For example, TOGAF~\cite[Ch. 47]{togaf9.1} suggests an approval process based on hierarchy, with an Architecture Board responsible for decision making; and FEA sets enterprise-wide standards to be followed by all through a set of common reference models~\cite{sessions2007}. For these reasons, ensuring the suitability of modern EA frameworks for  decentralized business environments--which are highly dynamic--is becoming increasingly relevant. 

This thesis addresses the problem of a suitable EA for decentralized organizations. The goal is not to create a new EA framework, but rather to enhance existing ones through guidelines or principles that increase their support for decentralized business environments.

An important part of this thesis project is to analyze the suitability of existing EA frameworks for decentralization. Therefore, the problem will be fully explicated--including a precise definition, problem motivation, and root cause analysis--in the results chapter, Section \ref{sec:exproblem}.

%%%%%%%%%%%%
%%%%%%%%%%%%
%~\cite{kaisler2005ea}
%rapidly changing business conditions have been identified as a critical problem in EA
%
%
%Rapidly changing conditions
%%%%%%%%%%%%%
%%%%%%%%%%%%%
%~\cite{}
%The embracing nature of an EA coupled with the constantly changing environment its management takes place in, gives rise to a number of severe challenges.
%%%%%%%%%%%%%
%%%%%%%%%%%%%
%~\cite{Bente2012}
%On the other hand, we encountered companies that, despite having a fully institutionalized EA in place, were in a state close to paralysis. The conglomerate of business, organizational, and technical dependencies had grown to a muddle that made reasonable changes impossible. 
%
%EA is supposed to prevent such failures— so why do they keep happening? 
%
%EA fails to keep pace with the speed of change in modern business.
%
%
%Sometimes the EA organization has an overly rigid approach to enforcing its own standards. Instead of discussing a sensible level of standardization with the IT crowd on the ground, enterprise architects invest their energy in political fights for IT standards that are irrelevant at best and that stall creativity at worst.
%
%In addition, if enterprise architects claim to be the only decision making body in technical matters, there is a huge risk that they create a bottleneck. If decisions take ages due to pending strategic issues, imminent changes in the business model, and so forth, IT projects can be seriously delayed. The practical consequence is that projects deliberately circumvent the enterprise architects— for example, by choosing less suitable technologies not managed by the EA group.
%
%An IT landscape requires continuous evolution. Concentrating only on the operational aspects of IT will not be enough to meet future challenges. Both the innovations on the business side— new products, markets, and business models— and on the technology side, require a constant renewal. It is one of the core tasks of EA to plan and monitor this evolution.
%
%Our analysis of EA failures and shortcomings has revealed that EA needs to find its proper position in a complex and chaotic environment. The solution is not about eliminating chaos altogether. That would overstretch the meager powers of the EA or IT organization, or it could turn EA into an overly rigid control freak— just the other side of the coin.
%
%As we will see, the real issue is to balance chaos and order properly. The question is how to leverage chaos to increase orderliness.
%
%
%
%%%%%%%%%%%%%