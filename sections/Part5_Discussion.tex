% \subsection{Limitations of the Case}
%
% Limitations should be discussed in terms of reproducibility, validity, reliability, generalizability, extensibility, credibility, etc.
%
%
% reproducibility --> not relevant??
%
%\subsubsection*{Extensibility}
%Move to Future?
%
%
%\subsubsection*{Generalizability}
The three guidelines presented in section \ref{sec:design_iteration2}; willful coordination, decentralized authority structure, peer decision making, are, on their own, highly generalizable. These three guidelines should be applicable to organizations other than the case, however no research on this topic was conducted.

% validity, credibility, reliability
%\subsubsection*{Credibility}
There are two main limitations of the thesis which negatively affect the overall credibility of the results and conclusions: The first is a lack of explicit EA in the case, and the second is that the evaluation of the artifact was limited to an argumentative evaluation. These limitations are primarily due to the difficulty in finding a case organization that is both decentralized and implements an explicit EA. 

The EA of the case organization is implicit rather than explicit, i.e. they do not implement an accepted EA framework but they do organize their activities in some manner: their ``implicit EA''. As an implicit EA does not explicitly make use of the concepts EA frameworks, it becomes difficult to draw parallels -- which are therefore very open to interpretation -- between the case's EA and the EA frameworks of TOGAF, Zachman, and FEA.  Furthermore, the validity of the interpretations, and thus the results, may be somewhat negatively affected by this.

The second main limitation is that there was no opportunity to apply the presented artifact in practice, thus preventing an ex post evaluation of the artifact. The demonstration shows that the artifact is plausible, however the use of an informed argument form of evaluation based on the demonstration means that the artifact should still be considered as immature.

An additional limitation that bears mentioning is that only three EA frameworks (TOGAF, Zachman Framework, FEA) were described and analyzed in this thesis due to the time constraints which exist on a Master's thesis project. The field of EA is quite large, however, and there are other EA frameworks and methodologies. This said, TOGAF, Zachman Framework and FEA are quite influential and extensively used in the field of EA, and thus are able to together provide a reasonably accurate representation of the state of the art of Enterprise Architecture. 

%Despite these limitations, the findings of this thesis are still interesting as this thesis is part of a research work in its early phases, and as such is exploratory in nature. The ideas presented here have been used to form the basis for two papers that were accepted in two conferences, PoEM\footnote{\url{http://poem2013.rtu.lv}} 2013 and the TEAR workshop at IEEE EDOC\footnote{\url{http://planet-sl.org/edoc2013/}} 2013: \cite{speckert2013} and \cite{rychkova2013}.

  