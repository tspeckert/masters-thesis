\subsection{Fulfilment of initial requirements}

In keeping with the principles of design science, it is important that the artifact described in section \ref{sec:design_iteration2} and demonstrated in section \ref{sec:demo} fulfils the requirements outlined in table \ref{table:requirements}. 

\paragraph{Lateral Coordination}

The first requirement is that the artifact supports the lateral coordination characteristic of a decentralized organization, and the decentralized authority structures and heterogeneous goals that go along with it. The three guidelines of the artifact, being based on peer production, inherently support lateral coordination over vertical coordination. The core characteristic of peer production is that a number of peers work together without a central coordinating authority (as in vertical coordination), which supports lateral coordination. The artifact supports decentralized authority structures and decision making (requirement 1.1) by suggesting that individual operational units are responsible for their own decision making. Furthermore, it suggests that the individuals in such a department make decisions collaboratively by giving each individual a vote for major decisions. This further supports heterogeneous goals (requirement 1.2) as each individual (and their respective views and goals) is able to contribute to making decisions.

Requirement 2? --> wilful coordination + voting for approval?

%TODO reflect decisions in design "peer decision making"
%TODO maybe also suggest discussion groups/meetings instead of just voting
%TODO fix requirements table

Goals -> everyone has a voice

demonstrated in section \ref{sec:demo} addresses this requirement by proposing six changes to the coordination practices of the case organization. The first suggestion addresses how 
  
The   model  shall  support  lateral coordination
1.1 The model shall support decentralized authority structures
1.2 The model shall support heterogeneous goals

The model shall support governance activities
2.1 The model shall address architecture interoperability issues
2.2 The model shall support approving architectural artifacts for implementation
2.3 The model shall support accountability


\subsection{Advantages of the new EA compared to the old and Centralized ones}

In order for the artifact to be beneficial for the case organization, it needs to provide some benefits over their current implicit EA and over... 

In order for the artifact to be considered valid, its demonstration needs to offer benefits over the case's as-is architecture and traditional EA suggestions. 

%TODO word that better

The peer production based framework differs from the as-is and centralized governance frameworks in a number of facets. In terms of decision rights allocation, the peer production based framework keeps the same operating principle as the as-is framework -- to push decision making as far down to the operational level as possible -- whereas a centralized framework would instead keep decision making in the upper levels of management. As most of the staff are operationally-focused (e.g. professors, researchers, and PhDs), this would likely be a sub-optimal situation. 

Budgeting in both the as-is and centralized frameworks is centralized, which is in conflict with the decentralized IT management. In the peer production based framework, a mix between  centralized and decentralized is instead proposed. Here, the budget is still allocated centrally, however the department then has complete control over their allocated amount. This should remove the conflict as the department would then have the necessary control over their own budget to operate in a manner decentralized from the rest of the organization. 



%Decision rights
% C --> A minimum of decision rights are pushed down to the operational level.
% D --> same as as-is

%Budgeting 
% C --> same as as-is
% D --> remove that ability of the central to make decisions for the dept

Advisory group
 C --> made up of upper management
 D --> same

Strategy and vision
 C --> same
 D --> members control over setting

IT 

Move to completely centralized IT

IT Approval
 C --> decided on by university IT
 D --> projects run by IT, approval granted collab (instead of from above)

Essential Systems 
 C --> same as as-is
 D --> Departments determine what is essential collaboratively. All departments pay for and use these systems.

Non-essential
 C --> decision made by the university
 D --> Opt-in (only pay for opting in)

discuss whether the proposed decentralized EA brings advantages with respect to the ``as-is'' and the centralized ones. 

\subsection{Limitations of the case}

implicit versus explicit EA

no chance for eval/applying suggestions




  