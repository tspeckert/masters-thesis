\subsection{Fulfilment of initial requirements}

In keeping with the principles of design science, it is important that the artifact described in section \ref{sec:design_iteration2} and demonstrated in section \ref{sec:demo} fulfils the requirements outlined in table \ref{table:requirements}. 

\paragraph{Lateral Coordination}

The first requirement is that the artifact supports the lateral coordination characteristic of a decentralized organization, and the decentralized authority structures and heterogeneous goals that go along with it. The three guidelines of the artifact, being based on peer production, inherently support lateral coordination over vertical coordination. The core characteristic of peer production is that a number of peers work together without a central coordinating authority (as in vertical coordination), which supports lateral coordination. The artifact supports decentralized authority structures and decision making (requirement 1.1) by suggesting that individual operational units are responsible for their own decision making. Furthermore, it suggests that the individuals in such a department make decisions collaboratively by giving each individual a vote for major decisions. This further supports heterogeneous goals (requirement 1.2) as each individual (and their respective views and goals) is able to contribute to making decisions.

Requirement 2? --> wilful coordination + voting for approval?

%TODO reflect decisions in design "peer decision making"
%TODO maybe also suggest discussion groups/meetings instead of just voting
%TODO fix requirements table

Goals -> everyone has a voice

demonstrated in section \ref{sec:demo} addresses this requirement by proposing six changes to the coordination practices of the case organization. The first suggestion addresses how 
  
The   model  shall  support  lateral coordination
1.1 The model shall support decentralized authority structures  --> peer produced strategy
1.2 The model shall support heterogeneous goals

The model shall support governance activities --> IT does this
2.1 The model shall address architecture interoperability issues
2.2 The model shall support approving architectural artifacts for implementation
2.3 The model shall support accountability


\subsection{Advantages of the new EA compared to the old and Centralized ones}

In order for the artifact to be beneficial for the case organization, it needs to provide some benefits over their current implicit EA and over... 

In order for the artifact to be considered valid, its demonstration needs to offer benefits over the case's as-is architecture and traditional EA suggestions. This demonstration was done by outlining the as-is governance framework of the case, a proposed peer production based framework, as well as an example of how their framework could work in a centralized manner. This section will discuss the three frameworks that make up the artifact's demonstration and how the peer production framework can offer an advantage over the other two. 

%TODO word that better

The peer production based framework differs from the as-is and centralized governance frameworks in a number of facets. In terms of decision rights allocation, the peer production based framework keeps the same operating principle as the as-is framework -- to push decision making as far down to the operational level as possible -- whereas a centralized framework would instead keep decision making in the upper levels of management. As most of the staff are operationally-focused (e.g. professors, researchers, and PhDs), this would likely be a sub-optimal situation. 

Budgeting in both the as-is and centralized frameworks is centralized, which is in conflict with the decentralized IT management. In the peer production based framework, a mix between  centralized and decentralized is instead proposed. Here, the budget is still allocated centrally, however the department then has complete control over their allocated amount. This could potentially remove the conflict as the department would then have the necessary control over their own budget to operate in a manner decentralized from the rest of the organization. 

The setting of strategy and overall operating principles for the department is centralized in both the centralized and as-is frameworks. In the peer production framework, a mixed style of governance has been proposed in its place. The decentralized component is that the department members produce the operating principles and strategy using peer production while the centralized component is that approval is still needed from the faculty. This offers the advantage of allowing the department to set a strategy that reflects all aspects of the department (by having input from everyone instead of just the department board) that is still compatible with the rest of the university (as approval is needed by the faculty).

%Decision rights
% C --> A minimum of decision rights are pushed down to the operational level.
% D --> same as as-is

%Budgeting 
% C --> same as as-is
% D --> remove that ability of the central to make decisions for the dept

%%%%Advisory group
%%%% C --> made up of upper management
%%%% D --> same

%Strategy and vision
% C --> same
% D --> members control over setting

The suggested peer production framework is significantly different from the as-is and (potential) centralized frameworks. It seeks to maintain the departmental-independence prevalent in the as-is framework while addressing the incompatible architecture components this results in. This is primarily accomplished through the cooperative classification of essential and non-essential systems, and the difference in governance for the two types. For a system to be classified as essential, this must be collaboratively agreed upon by the departments. 

With completely centralized IT, the department IT would be a subordinate entity to the central IT (or even not exist at all). 

IT 

Move to completely centralized IT

IT Approval
 C --> decided on by university IT
 D --> projects run by IT, approval granted collab (instead of from above)

Essential Systems 
 C --> same as as-is
 D --> Departments determine what is essential collaboratively. All departments pay for and use these systems.

Non-essential
 C --> decision made by the university
 D --> Opt-in (only pay for opting in)

discuss whether the proposed decentralized EA brings advantages with respect to the ``as-is'' and the centralized ones. 

\subsection{Limitations of the case}

implicit versus explicit EA

no chance for eval/applying suggestions




  