The field of Enterprise Architecture (EA) emerged in order to combat two increasingly prevalent problems facing enterprises: system complexity and business-IT alignment~\cite{sessions2007}. As enterprises rely more and more on information systems of increasing complexity, these problems become even more important. The field of EA views the solution to these problems to be one of concurrent design. It is not enough simply try and fit IT to the business; business and IT aspects should be designed concurrently.
%
%....EA is an enabler of collaboration, which as show earlier is an important concept to decentralized enterprises...however 
%   
%As shown in the previous section, collaboration important to decentralized enterprises: need way to collaborate
%    
%\textbf{The Problem: }EA could be that solution, however it has shortcomings for Decentralized Enterprises. 

IF YOU HAVE ENOUGH INFORMATION, YOU CAN SHOW THE "CENTRALIZED -ORIENTED" and "DECENTRALIZED ORIENTED" characteristics of the considered EA frameworks in a TABLE: 
Columns: TOGAF, ZACHMAN, FEA
Rows: METHOD, Description, Engine

IF NOT - TEXT IS OK.

\subsection{TOGAF}

\subsubsection{Concepts supporting a centralized organization}

TOGAF suggests the creation of an Architecture Board of 4-10 members ``to oversee the implementation of the [architecture] strategy''~\cite{togaf9.1}. This board has an important role in Architecture Governance, such as ``[p]roviding the basis for all decision-making with regard to the architectures''~\cite{togaf9.1} and enforcing architecture compliance. Having a single, central board with responsibility over the entire architecture process is not compatible with a decentralized organization. [WHY]

Throughout TOGAF, references are made to the existence of a bureaucratic or hierarchical centralized structure in place. For example, the strategic level of the Architecture Landscape "allows for direction setting at an executive level"~\cite{togaf9.1}.





Approval

Assumes formal bureaucratic processes

Assumes the usage of a dedicated EA team for all EA--> not collaborative


Concepts Supporting Decentralization:

\subsection{Zachman}
Limited perspectives?

\subsection{FEA}
Standards for everyone

SUMMARY TABLE;
(see my handwritten comments in pdf)
