In this study we have analyzed the problem of non-fit between emerging decentralized organizational environments, and established EA methodologies aimed to model organizations and specifically their linkages between business and IT. 

We have argued that modern organizations show strong tendencies toward decentralization in their organizational structures and thereby IT governance by following different patterns, having in common fostering of entirely new relationships between business processes and IT, and how IT resources are managed. The classification of organizational forms of IT presented in Section 3, we have used to assess if the dominant EA methodologies can support them.

Current EA, providing methods to set  up organizations' IT architecture, management and evolution, fail to solve this major concern in decentralized environments by helping organizations  to address  the
challenges related  to changes required  for, or issued by IT. We have surveyed Zachman Framework, TOGAF and FEA, and concluded that the first is unable to support any significant aspect of decentralization, while the latter two provide some basic flexibility in TOGAF, it is mainly facilitated by the ability to have different architecture for organizational units and by providing  space for new methods for the architecture development; in FEA, the conclusions are similar,  while the top- level organization  standards need to be obeyed by all units. Consequently, implementations of these methodologies are heavily limited to promote, or even support new decentralized organization patterns fostered by virtual organizations, collaborative networks, coopetitions, and others.

The aim of this research is to contribute to a state-of the-art on enterprise modeling methodologies by analyzing the decentralization of organizations and supporting business patterns and technologies, and thereby the consequences of this trend to the requirements  for new approaches  to IT resources,  namely, their use  and management. Regarding future work, our next steps involves contrasting the presented theories and argumentations empirically, i.e. by mapping them to EA of different organizations.  Such an ongoing study concerns an organization in the public sector of Sweden, exposing many of decentralized behavior as discussed in this paper.