The findings and shortcomings of this thesis can be used to outline a number of concrete suggestions for future research in the area of decentralization in EA. An ideal case study that would address the shortcomings of this thesis would involve one or more case organizations that meet the following criteria:

\begin{itemize}
\item Has a formal EA specification that has also been implemented
\item Exhibits some significant attributes of a decentralized organization (outlined in table \ref{table:org_characteristics})
\item Has problems with their EA relating to their decentralized characteristics
\item Is interested in implementing changes to resolve these issues and monitoring the results of these changes
\end{itemize}

Together, these four criteria would allow for direct comparison to modern EA frameworks (due to the formal EA specification) and formal evaluation (due to the ability to implement and evaluate any developed artifacts). Concepts from action research could also be combined with the case study and design science approaches in order to incorporate regular feedback into the design process, thus potentially improving upon the results. 

In practice, however, finding a case that meets these four criteria may not be practical. 

ACTION RESEARCH

Ideal studies

Not-so-ideal studies that are more realistic

Future Work:

apply model to case and analyze results (ideal, but not feasible)

how peer production (my suggestions) can be implemented in practice (e.g. with something for voting)

find new case meeting some base requirements:
 - exhibits some significant attributes of a decentralized organization (refer to Table \ref{table:org_characteristics} and to \ref{fig:taxonomy})
 - has an EXPLICIT EA

extension requirements
 a) Explicit EA is incompatible with reality (wrt decentralization)
 b) don't have EA, but need to develop
 c) has EA that supports decentralization