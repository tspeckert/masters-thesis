The findings and shortcomings of this thesis can be used to outline a number of concrete suggestions for future research in the area of decentralization in EA. 

The generalizability and extensibility of the designed artifact are two topics that would benefit from future research. A natural topic for extension of the artifact would be how to extend it to include concrete guidelines on how to apply the artifact to a decentralized organization. This would improve the usefulness of the artifact greatly. Research into the generalizability could look to apply it to a variety of organizations for the purpose of determining if it can be considered suitable for use in all organizations with high degrees of decentralization.

An ideal case study that would address the credibility limitations of this thesis would involve one or more case organizations that meet the following criteria:

\begin{enumerate}
\item Has a formal EA specification that has also been implemented
\item Exhibits some significant attributes of a decentralized organization (outlined in table \ref{table:org_characteristics})
\item Has problems with their EA relating to their decentralized characteristics
\item Is interested in implementing changes to resolve these issues and monitoring the results of these changes
\end{enumerate}

Together, these four criteria would allow for direct comparison to modern EA frameworks (due to the formal EA specification) and formal evaluation (due to the ability to implement and evaluate any developed artifacts). Concepts from action research could also be combined with the case study and design science approaches in order to incorporate regular feedback into the design process, thus potentially improving upon the results. 

In practice, however, finding a case that meets these four criteria may be difficult. The fourth criteria would be particularly challenging to fulfil, however it might possible as part of a long-term research project. A case that only fulfils the first three criteria would still be quite beneficial as it would allow for the identification of concrete problems which could be mapped back to their source in an EA framework. 

Another approach could be to design the EA for a decentralized organization who currently does not have one. This could be used to, at a minimum, obtain feedback on the practical feasibility of such a decentralized EA, or ideally, to implement it and evaluate the results for refining it. 

A final suggestion would be to find a decentralized organization that has already implemented a kind of decentralized EA. In this case, this EA could be analyzed to determine its; a) effectiveness, and b) how it tackles the challenges associated with decentralization. It could further be compared to existing EA frameworks such as TOGAF, FEA and Zachman to highlight the major differences and areas where they could be improved by incorporating aspects from the decentralized EA. 