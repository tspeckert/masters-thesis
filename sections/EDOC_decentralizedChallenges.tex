\subsection{The Importance of Collaboration}
The idea that the whole needs to work together to operate in a way that benefits everyone --> still separate entities with independence doing their own thing at the same time as working together when appropriate for mutual benefit.

Caruso, Rogers and Bazerman~\cite{caruso2008boundaries} highlight the importance of information sharing and coordination for decentralized organizations. 

"In what will be an even faster changing world than the one we now  know,  businesses of all sizes  will need the ability to adapt to the dynamics of the exter­ nal environment. Automated information and com­ munication networks will  support the sharing of information throughout a large, widely  dispersed, complex company. The systems will form the organi­ zation's infrastructure and change the role of formal reporting procedures. Even in large  corporations, each  individual will  be able to communicate with any  other-just as if he or she  worked in  a small company."\cite{applegate1988}

\subsection{Challenges in Decentralized Organizations}

NOTE: Not sure how this fits in with the flow

As decentralized organizations function in a significantly different manner than centralized organizations, they offer a different set of challenges that need to be faced: "~...~the novel and dynamic pressures that create the demand for decentralization in the first place can place organization leaders in considerably less certain, and consequently less commanding, positions."~\cite{caruso2008boundaries} In order to effectively meet these challenges, it is important to first understand what they are. [LIST WHAT I WILL TALK ABOUT]

A paper by Caruso, Rogers and Bazerman~\cite{caruso2008boundaries} highlights the importance of information sharing and coordination for these organizations. In order to succeed at these aspects, they outline three barriers that decentralized organizations need to overcome. The first barrier is intergroup bias; the tendency to treat one's own group better than other groups. The second barrier is group territoriality; the tendency for a group to protect their territory (physical or informational). The third barrier is poor negotiation strategies used by different groups when interacting with one another. 

Intergroup bias is direct result of having separate, autonomous groups within an enterprise~\cite{caruso2008boundaries}. The individual groups have a tendency to promote their own group over other groups, especially in situations where they are competing for a resource, such as a portion of the budget. A certain level of competition can be beneficial, however if it leads to hostility or distrust between groups, this can have a detrimental effect on their ability to share information and collaborate. This can prevent the groups from taking advantage of situations where they have to ability to work together for the benefit of everyone. 

The second barrier identified by Caruso et al. is group territoriality~\cite{caruso2008boundaries}. Group territoriality is characterized by group members taking action in order to protect their perceived territory. This can include physical territory such as space or tangible resources, as well as intangible territory, such as roles or information. Group territoriality is supported by a group's need to maintain its identity, its reputation of competence and sense of value, and a group's need for a stable home within the organization from which they interact with the rest of it.

Group territoriality encourages "a sense of psychological ownership"~\cite{caruso2008boundaries} for a group's members which can enforce the belief that they are the sole responsible party for a role or specific knowledge. This "inward-looking" behavior works against collaboration and information sharing. On the other hand, group territoriality can be beneficial; it can foster a sense of security in its members that "facilitates planning and execution of activities"~\cite{caruso2008boundaries}. 

The third barrier identified by Caruso et al. in decentralized organizations is related to negotiations between groups, and how these negotiations are often conducted using "poor negotiation  strategies"~\cite{caruso2008boundaries}. These poor strategies are the result of three common errors made while negotiating. The first error is a false belief in a "fixed pie" of value that is to be divided when negotiating. This prevents negotiating parties from recognizing situations where they are able to help each other, and therefore increase the size of the figurative pie. The second error is a failure to properly consider the other group's perspective. Understanding the other group's decision process, valuing process, and interests is key to discovering opportunities for helping one another, and the organization as a whole. The third error is when groups fail to even recognize they are in the process of negotiating. Instead, they see it as a competitive or hostile behavior where, again, they only see a fixed pie that is to be split up. This also prevents groups from taking advantage of opportunities to increase the size of the pie.    

% STRUCTURAL DILEMMAS from Reframing...

%\subsection{Case Study: Challenges in SweU}
%
%Budget: Better to use exactly rather than go under
%
%SweU already views collaboration as a way to increase the size of the pie
%
%more