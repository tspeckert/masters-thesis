\subsection{The Importance of Collaboration}
The idea that the whole needs to work together to operate in a way that benefits everyone --> still separate entities with independence doing their own thing at the same time as working together when appropriate for mutual benefit.

Caruso, Rogers and Bazerman~\cite{caruso2008boundaries} highlight the importance of information sharing and coordination for decentralized organizations. 

"In what will be an even faster changing world than the one we now  know,  businesses of all sizes  will need the ability to adapt to the dynamics of the exter­ nal environment. Automated information and com­ munication networks will  support the sharing of information throughout a large, widely  dispersed, complex company. The systems will form the organi­ zation's infrastructure and change the role of formal reporting procedures. Even in large  corporations, each  individual will  be able to communicate with any  other-just as if he or she  worked in  a small company."\cite{applegate1988}


\subsection{Challenges}
As decentralized organizations function in a significantly different manner than centralized organizations, they offer a different set of challenges that need to be faced: "~...~the novel and dynamic pressures that create the demand for decentralization in the first place can place organization leaders in considerably less certain, and consequently less commanding, positions."~\cite{caruso2008boundaries} In order to effectively meet these challenges, it is important to first understand what they are. Caruso et al. outline three barriers to collaboration; intergroup bias, group territoriality, and poor negotiation  strategies. Other challenges include.... (communication, planning, maintenance, decision making, change management, budgeting)

\subsection{Case Study: Challenges in SweU}

Budget: Better to use exactly rather than go under

SweU already views collaboration as a way to increase the size of the pie

more