\subsection{IT Governance}

Rockart, Earl and Ross~\cite{Rockart1996} describe a continuum of governance styles ranging from centralized to decentralized, with federalism in the middle. This style of Federal IT aims to have the strengths of both centralized and decentralized while eliminating their respective weaknesses. Some key weaknesses of centralized IT to eliminate are slow responsiveness and having systems that do not fit the needs of individual business units. Decentralized IT on the other hand lacks "synergy and integration"~\cite{Rockart1996} due to a lack of standardization. Federal IT would aim to balance these through a combination of central IT and IT in the business units. A primary task of the central IT would be to maintain standards for the entire enterprise. The business units would still have ownership of many of their own systems, allowing them to implement them as they deem best. This allows for systems that meet the individual business units needs, as well as enabling interoperability throughout the enterprise. 

Weill~\cite{Weill2004} also describes a federal style of IT governance as part of a larger description of IT governance archetypes and the major types of IT decisions that they apply to. In addition to the federal archetype, he describes a business monarchy, IT monarchy, feudal, IT duopoly and anarchy. In a business monarchy all IT related decisions are made in a centralized manner by the top-level executives (e.g. the CxOs). In an IT monarchy, a group of IT professionals are responsible for making the decisions. This is also highly centralized as the authority resides with this group. An IT duopoly is characterized by two groups, one of IT executives and the other of business executives, coming to agreements in order to make decisions. This is more centralized than the federal form, as the decisions are only made by the two groups, rather than each individual business unit having input. The feudal is much less centralized. It is where individual organizational units are responsible for their own decisions. Anarchy is a highly decentralized style of governance. It is similar to the feudal archetype, however the size of the units is much smaller. Instead of being an entire business unit, small teams or even individuals are responsible for their own decisions.

Weill then proposes five major IT decision domains that these decisions apply to: IT principles, IT architecture, IT infrastructure strategies, business application needs, and IT investment~\cite{Weill2004}. IT principle are high level statements about the use of IT in the enterprise. IT architecture decisions relate to the policies and rules that describe how IT is to be sued, as well as the roadmap for implementing them. Decisions on IT infrastructure strategies relate to the foundation IT services (e.g. network, help desk) that exist throughout the entire enterprise.  Decisions on business application needs are about determining what needs IS will fulfill. IT investment decisions are related to financing and justification of IT projects.

% not certain here if these references should go to Weill or to his referenced papers. 


\subsection{Drivers for Decentralization}

Flexibility \& adaptability

Customization to local needs

Promote creativity

Globalization

Driver for inter-organizational networks: "fast-moving fields ... where knowledge is so complex and widely dispersed"~\cite{Bolman2008}

strengths of centralized i.e. hierarchies
define dynamic, uncertain, rapidly changing, stable... environments

"novel and dynamic pressures that create the demand for decentralization in the first place can place organization leaders in considerably less certain, and consequently less commanding, positions"

page 233 pearlson


\subsection{Case Study: An Institution of Higher Education in Sweden}

DSV pushes decision making power as close to the operational level as possible

Usage of meetings/seminars to bring everyone together and plan major change (Bologna)

Fluid, project-oriented teams; research projects

Informal communication lines \& ability to collaborate whenever they see fit (and is self-managed): DSV IT collaborates with other departments, collaboration for education programs

Task forces: reference groups of appropriate knowledgeable people for major (non-research) projects

Standardization exists, but is frequently not forced

Minimal formal metrics for performance measurement: where government mandated (re: education), but not much else.  


++++++++++++++++++++++++++++++++++++++++++++++



\subsection{The Importance of Collaboration}
The idea that the whole needs to work together to operate in a way that benefits everyone --> still separate entities with independence doing their own thing at the same time as working together when appropriate for mutual benefit.

Caruso, Rogers and Bazerman~\cite{caruso2008boundaries} highlight the importance of information sharing and coordination for decentralized organizations. 
        
\subsection{Challenges}
As decentralized organizations function in a significantly different manner than centralized organizations, they offer a different set of challenges that need to be faced: "~...~the novel and dynamic pressures that create the demand for decentralization in the first place can place organization leaders in considerably less certain, and consequently less commanding, positions."~\cite{caruso2008boundaries} In order to effectively meet these challenges, it is important to first understand what they are. Caruso et al. outline three barriers to collaboration; intergroup bias, group territoriality, and poor negotiation  strategies. Other challenges include.... (communication, planning, maintenance, decision making, change management, budgeting)

\subsection{Case Study: Challenges in SweU}

Budget: Better to use exactly rather than go under

SweU already views collaboration as a way to increase the size of the pie

more