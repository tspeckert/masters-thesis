The practice of EA is just one potential solution to the problems of business-IT alignment and enterprise integration. However, other potential solutions do exist, and this section will briefly outline a few of them, namely, enterprise modeling, Service-Oriented Architecture, and . In comparison with EA which While EA takes a holistic approach to the problem, 


RELATED WORKS

Enterprise Modeling
  goal models
  value models
  process modeling
  

SOA
  its architecture
  Enterprise Service Bus


Other Approaches to IT-Business Alignment
  
  The Strategy Triangle
  
\subsection{Enterprise Modeling}

``Enterprise Modeling is the art of externalizing enterprise knowledge which adds value to the enterprise or needs to be shared. It consists in making models of the structure, behavior and organization of the enterprise''~\cite{Vernadat200215}. Through modeling, practitioners aim to better understand the current or future organization and function an enterprise. To this end, common enterprise categories of enterprise modeling include goal, process, and value modeling. 

\subsubsection{Goal Modeling}

Goal modeling aims to describe the goals of an enterprise, their interrelations, means for achieving these goals, and additional factors that impact them. A specific example of a goal modeling technique is the Business Motivation Model (BMM)~\cite{bmm2010}. In BMM, means, ends and influencers of an organization are modeled and assessed. Here, the focus is on understanding what an organization wants to achieve (i.e. goals). The relationship between an organizations goals and its means is described, the specifics of the means are not. This is in contrast with EA which takes a holistic approach in describing an entire enterprise, not only its goals.

\subsubsection{Business Process Modeling}

According to Roshen, "[a] business process is a collection of related, structured activities or tasks that produces a specific product"~\cite{roshen2009soa}. Business processes are a complicated matter, and as such, process modeling is used to describe them in a detailed and accurate manner, in order to understand how an organization creates some output. Many different modeling languages exist, for example, Business Process Model and Notation (BPMN)~\cite{model2011notation}, Event-driven Process Chain (EPC)~\cite[Ch. 6]{scheer2005process}, and UML Activity Diagrams. Processes can exist within an organization or they can interact with. 


Weske~\cite{weske2012business}
"The primary focus of intraorganizational business processes is the streamlining
of the internal processes by eliminating activities that do not provide
value."

\subsubsection{Value Exchange Modeling}



\subsection{Enterprise Integration}

``Enterprise Integration (El) consists in breaking down organizational barriers to improve synergy within the enterprise so that business goals are achieved in a more productive and efficient way.''~\cite{Vernadat200215}
