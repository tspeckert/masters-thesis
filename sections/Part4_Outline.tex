%6.4 Guidelines for Outline Artefact and Define Requirements
%
%- Specify what artefact to build! Specify the artefact type (construct,
%model, method, instantiation) and its general characteristics.
%
%- Formulate each requirement clearly! Describe each requirement
%in a precise, concise and easily understandable way.
%
%- justify each requirement! For each requirement, explain why
%it is needed and relate it to the problem.
%
%- Be realistic but also original! Ensure that it is realistic to develop
%an artefact fulfilling the requirements but also try to be
%original.
%
%- Specify the sources of the requirements! Describe the literature
%and the stakeholders that have contributed to defining
%the requirements.
%
%- Describe how you have defined the requirements! Explain
%what you have done to define the requirements, in particular
%how you have reviewed the stakeholders and research literature.
%
%
%\subsection{Iteration 1}
%
%Not performed. 

%TODO Update method to match. Not doing 2 iterations here. Decide on which part of EA we do here. 
\subsection{Iteration 2}

\subsubsection*{Outline Artifact}

A key issue identified in the case is with decision making, where a mismatch exists between their integration- and centralization-focused architecture and an organizational structure with some highly decentralized aspects. This mismatch leads to problems in the ongoing success of the university's current architecture -- which is a problem addressed by the EA engine. Furthermore, as this issue lies in how decisions are made and enforced (i.e. the decision to have university-wide integration through use of a common system that cannot be enforced in practice), a governance framework which resolves these mismatches shall be developed. 

As identified in Section \ref{sec:exproblem}, an important property of a decentralized business environment that needs to be supported by EA is horizontal coordination. However, current EA frameworks primarily support vertical coordination in their governance styles, as identified in Section \ref{sec:exproblem} and Table \ref{table:summary}. Consequently, a different approach to EA governance is necessary for decentralized environments.

This thesis project will create an \textit{informal method artifact}~\cite[Ch. 2.4]{johannessonPerjons2012} in the form of a set of \textit{guidelines} to address this issue of governance in EA. These guidelines will be high-level Architecture Principles for a governance framework supporting decentralization.

%maybe not model, maybe something else

%For example, if IT systems
%need to be integrated to make a process more efficient, a solution
%can be a method for integrating IT systems, a model of an integration
%architecture, or an instantiation in the form of an integration tool.
%When the artefact type has been chosen, the artefact is to be described
%on an overview level.

%
%model or method
%
%However, models can also be used to describe
%potential solutions to practical problems, e.g. a drawing for a new
%type of vehicle or a proposal for an architecture of a mobile operating
%system. Such prescriptive models work as descriptions of possible
%future solutions and help to build artefacts that can solve practical
%problems.
%
%Methods express prescriptive knowledge by defining guidelines
%and processes for how to solve problems and achieve goals. In particular,
%they can prescribe how to create artefacts. Methods can be
%highly formalized like algorithms, but they can also be informal such
%as rules of thumb or best practices. Some examples are methods for
%database design, change management initiatives, or web service development.


\subsubsection*{Define Requirements}

The requirements for the artifact are outlined in Table \ref{table:requirements}.

\begin{table}
\caption{Requirements for the solution artifact}
\label{table:requirements}
%%\begin{tabular}{l p{0.28\textwidth}}
\begin{tabular}{ | p{0.5\textwidth} | p{0.5\textwidth} |}
%
\hline
%
\textbf{Requirement} & 
\textbf{Source}  \\
%
\hline
%
\textbf{1 The artifact shall support lateral coordination} & \\
%From Table \ref{table:org_characteristics} \\
%
%\hline
%
1.1 The artifact shall support decentralized authority structures & 
\cite{Weill2004,pearlson2009,robbins1997,Camarinha-Matos2005} \\
 %
%\hline
%
%1.2 The model shall support fluid roles &
%\cite{valveHandbook} \\
%
%\hline
%
1.3 The artifact shall support heterogeneous goals & 
\cite{Camarinha-Matos2005} \\
%
%\hline  
%
%The model shall support conflict resolution of conflicting goals  & 
%Implied from the previous requirement \\
%
\hline
%
\textbf{2 The artifact shall support governance activities}  & 
 \\


2.1 The artifact shall address architecture interoperability and integration issues &
The primary purpose of governance in FEA \cite[Sec. 2]{FEA_PMO2007} and TOGAF \cite[Ch. 50]{togaf9.1} is to ensure architecture components work well with one another for meeting an organization's goals \\

\hline
%
\end{tabular}
\end{table}

 
