%6.4 Guidelines for Outline Artefact and Define Requirements
%- Specify what artefact to build! Specify the artefact type (construct,
%model, method, instantiation) and its general characteristics.
%Formulate each requirement clearly! Describe each requirement
%in a precise, concise and easily understandable way.
%
%- justify each requirement! For each requirement, explain why
%it is needed and relate it to the problem.
%Be realistic but also original! Ensure that it is realistic to develop
%an artefact fulfilling the requirements but also try to be
%original.
%
%- Specify the sources of the requirements! Describe the literature
%and the stakeholders that have contributed to defining
%the requirements.
%
%Describe how you have defined the requirements! Explain
%what you have done to define the requirements, in particular
%how you have reviewed the stakeholders and research literature.
%

\subsection{Iteration 1}

\subsubsection*{Outline Artifact}

As identified in Section \ref{sec:exproblem}, an important property of a decentralized business environment that needs to be supported by EA is horizontal coordination. However, current EA frameworks primarily support vertical coordination in their governance styles, as identified in Section \ref{sec:exproblem} and Table \ref{table:summary}. Consequently, a possible solution to the problem of decentralization in EA could be \textit{model} of an EA governance framework supporting decentralization. 

%TODO since we're deciding on artifact here, need to reflect on that in method (e.g. selecting based on case study)

%Governance: Board, Principles, Compliance

%For example, if IT systems
%need to be integrated to make a process more efficient, a solution
%can be a method for integrating IT systems, a model of an integration
%architecture, or an instantiation in the form of an integration tool.
%When the artefact type has been chosen, the artefact is to be described
%on an overview level.

%
%construct, a model, a method or an instantiation.
%
%Methods express prescriptive knowledge by defining guidelines
%and processes for how to solve problems and achieve goals. In particular,
%they can prescribe how to create artefacts. Methods can be
%highly formalized like algorithms, but they can also be informal such
%as rules of thumb or best practices. Some examples are methods for
%database design, change management initiatives, or web service development.
%
%Models are used to depict or represent other objects. A model can
%represent an existing situation, which can be used for describing and
%analyzing problem situations. Such a descriptive model may work as a
%pedagogical tool for representing a current situation and explaining
%why it is challenging. However, models can also be used to describe
%potential solutions to practical problems, e.g. a drawing for a new
%type of vehicle or a proposal for an architecture of a mobile operating
%system. Such prescriptive models work as descriptions of possible
%future solutions and help to build artefacts that can solve practical
%problems. There are also predictive models that can be used to forecast
%the behavior of objects and systems. Thus, models can express
%descriptive as well as predictive and prescriptive knowledge. In design
%science, the focus is on prescriptive models. Typical examples
%are business process models, systems architectures, domain ontologies,
%and user models.



\subsubsection*{Define Requirements}

Discuss.... See \ref{table:requirements}.

\begin{table}
\caption{Requirements for a Decentralized EA Governance Model}
\label{table:requirements}
%%\begin{tabular}{l p{0.28\textwidth}}
\begin{tabular}{ | p{0.5\textwidth} | p{0.5\textwidth} |}
%
\hline
%
\textbf{Requirement} & 
\textbf{Source}  \\
%
\hline
%
\textbf{The model shall support lateral coordination} & \\
%From Table \ref{org_characteristics} \\
%
%\hline
%
The model shall support decentralized authority structures & 
\cite{Weill2004,pearlson2009,robbins1997,Camarinha-Matos2005} \\
 %
%\hline
%
The model shall support fluid roles &
\cite{valveHandbook} \\
%
%\hline
%
The model shall support heterogeneous goals & 
\cite{Camarinha-Matos2005} \\
%
%\hline
%
The model shall support conflict resolution of conflicting goals  & 
Implied from the previous requirement \\
%
\hline
%
The model shall support Architecture Principles & 
 \\
%
\hline
%
The model shall support a compliance process & 
 \\
%
\hline
%
 & 
 \\
%
\hline
%
 & 
 \\
%
\hline
%
 & 
 \\
%
\hline
%
\end{tabular}
\end{table}

%TODO Requirements from EA literature? Reflect in method. 

% **********************************************************************
% **********************************************************************
% **********************************************************************
% **********************************************************************
% **********************************************************************
% **********************************************************************

\subsection{Iteration 2}

\subsubsection*{Outline Artifact}


\subsubsection*{Define Requirements}


