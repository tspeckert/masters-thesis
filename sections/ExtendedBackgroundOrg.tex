\subsection{Overview}
start with different styles of organization structure ranging from centralized to decentralized
definition a decentralized organization
barriers to decentralization
principals used by decentralized organizations


\subsection{Why Have Decentralization in Organizations?}
strengths of centralized i.e. hierarchies
define dynamic, uncertain, rapidly changing, stable... environments

"novel and dynamic pressures that create the demand for decentralization in the first place can place organization leaders in considerably less certain, and consequently less commanding, positions"

page 233 pearlson

\subsection{Specific Forms of Organizational Structure}
\label{org:form}

Organization have traditionally utilized more centralized forms of organizational structure. Of the centralized structures, a hierarchical organization structure is perhaps the most well-known and most utilized. As shown in figure[HIERARCHICAL FIGURE REF], it is characterized by a a hierarchy of positions. Except for the top level position, each position has one supervisor and zero or more subordinates. Decision rights and communication lines are strictly defined and work their way down from the top (i.e. the centre): the scope of a position is specialized and strictly defined by your superior and one works who they assigned to work with. The primary benefit of a hierarchy is that the high levels of management have strict governance and control of everything that goes on in the company. This allows them to easily direct the company how they deem best. [MIGHT NEED SOURCES ABOUT BENEFITS OF HIERARCHY]. Hierarchical organizations generally either divide their labor in terms of function, a grouping of common activities, or in terms of division, a grouping based on output. Due to this, According to Pearlson and Saunders, hierarchical organization structures are suited for stable, certain environments. ~\cite{pearlson2009}

[DRAWBACKS]

According to Pearlson and Saunders~\cite{pearlson2009}, another structure that is highly centralized is the flat type. A flat organizational structure is characterized by having a single (or small number) of people, frequently owners, at the top. The rest of the employees are all below the top level and are equal to one another. This kind of flat structure is effectively a hierarchical structure with only two levels. A common structure for new companies, Pearlson and Saunders state that is a centralized from of organizational structure as all the power and decision making authority typically is controlled by the person (or small number of people) at the top. They then tell the rest of the employees what to do. [MAYBE MORE AOBUT FLAT FROM PEARLSON] However, this is not always true for flat organizations, it depends on how they operate. For example, Valve Corporation, a software company in the video game industry released their handbook in 2012~\cite{valveHandbook}. In it, they describe their structure as being flat, but a very different style of flat where individual employees have complete freedom despite there being a president/founder at the top. Unlike the style of flat organization described by Pearlson and Saunders, at Valve it is a highly decentralized style. Nobody reports to anyone, and everyone is free to work on whatever they want to. Valve states that the company is "yours to steer"~\cite{valveHandbook}, meaning that everyone the power to alter the direction of the company. This difference in what a "flat" organization demonstrates is that it is important to take into account more than simply the structure of a organization, how that structure is implemented is equally important. 
[SUITABILITY]

[FLAT DIAGRAMS]

Another popular style of organization structure is the matrix organization structure~\cite{pearlson2009}. In this style, individuals are assigned two or more supervisors covering different dimensions of the enterprise. The aim here is to integrate these different dimensions. Pearlson and Saunders state that matrix organization structures are suited for dynamic environments with lots of uncertainty, presumably because their authority structure allows them to cover multiple aspects when making decisions. However, like a hierarchical structure, a matrix structure is a rigid construct with strictly defined roles, communication lines and decision rights. While there is a more distributed authority structure at the lowest level of management, the upper levels are structure in hierarchical fashion [MAYBE NEED A SOURCE] and remain highly centralized. This has many of the drawbacks of a hierarchical organization, and as a result, this rigidity may prevent this type of organization from effectively adapting to rapidly changing environments. 

[MATRIX DIAGRAMS]

In recent years a new type of organizational structure has emerged, called the networked organization structure~\cite{pearlson2009}. As depicted in figure [NETWORKED FIGURE], a networked structure aims to discard traditional hierarchies in favor of decentralized decision rights and flexible communication lines connecting the entire enterprise~\cite{applegate1988,pearlson2009}. This enables an organization where many (or all) employees are able to easily share knowledge and provide input into the overall decision making for the organization~\cite{pearlson2009}. An important effect of this is a flexible enterprise that promotes creativity. Pearlson and Saunders state that this type of organizational is suitable for dynamic and uncertain environments. This stands to reason, much more so than for matrix  organization structures, as a high level of creativity and flexibility should allow an organization to adapt quickly to changes in its environment. 

Network structure are not just within an organization, also between \cite{Bolman2008} .... in fast - moving fields
[NETWORKED FIGURE]

Adhocaracy\cite{Bolman2008}"
Adhocracy is a loose, fl exible, self - renewing organic form tied together mostly through lateral means (see Exhibit 4.6 ). Usually found in a diverse, freewheeling environment, adhocracy functions as an “ organizational tent, ” exploiting benefits that structural designers traditionally regarded as liabilities: “ Ambiguous authority structures, unclear objectives, and contradictory assignments of responsibility can legitimize controversies and challenge traditions. Incoherence and indecision can foster exploration, self- evaluation, and learning ” (Hedberg, Nystrom, and Starbuck, 1976, p. 45). Inconsistencies and contradictions in an adhocracy become paradoxes where a balance between opposites protects an organization from falling into an either - or trap.

Ad hoc structures are most often found in conditions of turbulence and rapid change. Examples are advertising agencies, think - tank consulting fi rms, and the recording industry. In the 1970s and 1980s, Digital Equipment was a well - known pioneer of adhocracy: “ In many ways [DEC] is a big company in small company clothes. It doesn ’ t believe much in hierarchy, rule books, dress codes, company cars, executive dining rooms, lofty titles, country club memberships, or most trappings of ‘ corporacy. ’ It doesn ’ t even have assigned parking spots. Only the top half - dozen executives have sizable offi ces. Everyone else at the company headquarters in Maynard, Mass., makes do with dinky doorless cubicles ” (Machan, 1987, p. 154).

Digital ’ s structural arrangements helped it become the world leader in minicomputers. But the structural design became a problem when the market shifted toward personal computers, where aggressive new competitors like Compaq and Dell were dominant. “ They fl ew so high and crashed so hard, ” said one observer, because “ at DEC, the internal mattered so much. They spent their lives playing with each other ” (Johnson, 1996, p. F11). The strength of Digital ’ s adhocracy, a flowering of local creativity, became a liability when the company needed a timely organization - wide change in direction.
"

Adhocracy~\cite{Applegate1988a}
Rapidly changing set of project oriented groups

\subsection{What is a Decentralized Organization?}

As demonstrated in section~\ref{org:form}, whether an enterprise is centralized or decentralized depends on more than simply its structure. Furthermore, enterprises will have elements of both centralization and decentralization in them, meaning that would be an oversimplification to classify an enterprise as just one or another. Consequently, organizational structure is best viewed as being on an organizational continuum, with decentralization on one end, centralization on the other end, and federalism in the middle~\cite{pearlson2009}. This section will describe a number of organizational characteristics that can be used to determine to what degree an organization is centralized or decentralized. 

The first characteristic are the allocation of decision rights in an enterprise, specifically who has the right to make what kinds of decisions in running and planning the enterprise.~\cite{pearlson2009} Decision rights are what controls the overall direction of an enterprise. They are needed in order to make change. In a completely centralized enterprise, all decision making authority would reside with a single, top-level authority. More decentralized enterprises would allow for participation in the decision making process by members throughout the enterprise. In a completely decentralized enterprise all members would have equal decision making rights. 

A second characteristic is the structure of communication lines in an enterprise. A centralized enterprise will have rigid hierarchies of communication, i.e. it is strictly defined who you work with and who you report to. Decentralized enterprises instead have less formalized communication lines~\cite{pearlson2009}, and more fluid, project oriented teams.~\cite{Applegate1988a}

The third characteristic is the choice of forms of coordination in an enterprise. More centralized enterprises lean towards primarily vertical style of coordination~\cite{Bolman2008}, which is characterized by formal authority, standardization and rules in operations and in IT, and planning and control systems. [ELABORATE] Decentralized organizations lean towards more lateral styles of coordination. Lateral coordination is characterized by meetings, task forces, coordinating roles, matrix structures, and networks~~\cite{Bolman2008}[ELABORATE AND MAYBE NOT USE EXACTLY] It should be noted that enterprises will generally have a mix of lateral and vertical coordination, it is the tendency of an enterprise to focus on one more than the other that is an indicator for decentralization. 

\subsection{Challenges in Decentralized Organizations}

As decentralized organizations function in a significantly different manner than centralized organizations, they offer a different set of challenges that need to be faced: "~...~the novel and dynamic pressures that create the demand for decentralization in the first place can place organization leaders in considerably less certain, and consequently less commanding, positions."~\cite{caruso2008boundaries} In order to effectively meet these challenges, it is important to first understand what they are. [LIST WHAT I WILL TALK ABOUT]

A paper by Caruso, Rogers and Bazerman~\cite{caruso2008boundaries} highlights the importance of information sharing and coordination for these organizations. In order to succeed at these aspects, they outline three barriers that decentralized organizations need to overcome. The first barrier is intergroup bias; the tendency to treat one's own group better than other groups. The second barrier is group territoriality; the tendency for a group to protect their territory (physical or informational). The third barrier is poor negotiation strategies used by different groups when interacting with one another. 

Intergroup bias is direct result of having separate, autonomous groups within an enterprise~\cite{caruso2008boundaries}. The individual groups have a tendency to promote their own group over other groups, especially in situations where they are competing for a resource, such as a portion of the budget. A certain level of competition can be beneficial, however if it leads to hostility or distrust between groups, this can have a detrimental effect on their ability to share information and collaborate. This can prevent the groups from taking advantage of situations where they have to ability to work together for the benefit of everyone. 

The second barrier identified by Caruso et al. is group territoriality~\cite{caruso2008boundaries}. Group territoriality is characterized by group members taking action in order to protect their perceived territory. This can include physical territory such as space or tangible resources, as well as intangible territory, such as roles or information. Group territoriality is supported by a group's need to maintain its identity, its reputation of competence and sense of value, and a group's need for a stable home within the organization from which they interact with the rest of it.

Group territoriality encourages "a sense of psychological ownership"~\cite{caruso2008boundaries} for a group's members which can enforce the belief that they are the sole responsible party for a role or specific knowledge. This "inward-looking" behavior works against collaboration and information sharing. On the other hand, group territoriality can be beneficial; it can foster a sense of security in its members that "facilitates planning and execution of activities"~\cite{caruso2008boundaries}. 

The third barrier identified by Caruso et al. in decentralized organizations is related to negotiations between groups, and how these negotiations are often conducted using "poor negotiation  strategies"~\cite{caruso2008boundaries}. These poor strategies are the result of three common errors made while negotiating. The first error is a false belief in a "fixed pie" of value that is to be divided when negotiating. This prevents negotiating parties from recognizing situations where they are able to help each other, and therefore increase the size of the figurative pie. The second error is a failure to properly consider the other group's perspective. Understanding the other group's decision process, valuing process, and interests is key to discovering opportunities for helping one another, and the organization as a whole. The third error is when groups fail to even recognize they are in the process of negotiating. Instead, they see it as a competitive or hostile behavior where, again, they only see a fixed pie that is to be split up. This also prevents groups from taking advantage of opportunities to increase the size of the pie.    

% STRUCTURAL DILEMMAS from Reframing...

\subsection{Principles of Existing Decentralized Organizations}

\subsubsection{Co-Design}
Smart Cities

\subsubsection{Coopetition}

\subsubsection{Virtual Organizations}

\subsubsection{Valve}