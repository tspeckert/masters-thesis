\subsection{Overview}
start with different styles of organization structure ranging from centralized to decentralized
definition a decentralized organization
barriers to decentralization
principals used by decentralized organizations


\subsection{Why Have Decentralization in Organizations?}
strengths of centralized i.e. hierarchies
define dynamic, uncertain, rapidly changing, stable... environments

"novel and dynamic pressures that create the demand for decentralization in the first place can place organization leaders in considerably less certain, and consequently less commanding, positions"

\subsection{Different Forms of Organizational Structure}
\label{org:form}

Organizations have traditionally utilized more centralized forms of organizational structure. Of the centralized structures, a hierarchical organization structure is perhaps the most well-known and most utilized. As shown in figure[HIERARCHICAL FIGURE REF], it is characterized by a a hierarchy of positions. Except for the top level position, each position has one supervisor and zero or more subordinates. Decision rights and communication lines are strictly defined and work their way down from the top (i.e. the centre): the scope of a position is specialized and strictly defined by your superior and one works who they assigned to work with. The primary benefit of a hierarchy is that the high levels of management have strict governance and control of everything that goes on in the company. This allows them to easily direct the company how they deem best. [MIGHT NEED SOURCES ABOUT BENEFITS OF HIERARCHY]. Hierarchical organizations generally either divide their labor in terms of function, a grouping of common activities, or in terms of division, a grouping based on output. Due to this, According to Pearlson and Saunders, hierarchical organization structures are suited for stable, certain environments. ~\cite{pearlson2009}

[DRAWBAKCS]

According to Pearlson and Saunders~\cite{pearlson2009}, another structure that is highly centralized is the flat type. A flat organizational structure is characterized by having a single (or small number) of people, frequently owners, at the top. The rest of the employees are all below the top level and are equal to one another. This kind of flat structure is effectively a hierarchical structure with only two levels. A common structure for new companies, Pearlson and Saunders state that is a centralized from of organizational structure as all the power and decision making authority typically is controlled by the person (or small number of people) at the top. They then tell the rest of the employees what to do. [MAYBE MORE AOBUT FLAT FROM PEARLSON] However, this is not always true for flat organizations, it depends on how they are run. For example, Valve Corporation, a software company in the video game industry released their handbook in 2012~\cite{valveHandbook}. In it, they describe their structure as being flat, but a very different style of flat compared to what is described by Pearlson and Saunders. where nobody reports to anyone even though there is a president/founder at the top. Unlike the style of flat organization described by Pearlson and Saunders, at Valve it is a highly decentralized style. Nobody reports to anyone, and everyone is free to work on whatever they want to. Valve states that the company is "yours to steer", meaning that everyone the power to alter the direction of the company. This difference in what a "flat" organization demonstrates is that it is important to take into account more than simply the structure of a organization, how that structure is implemented is equally important. 
[SUITABILITY]

[FLAT DIAGRAMS]

Another popular style of organization structure is the matrix organization structure~\cite{pearlson2009}. In this style, individuals are assigned two or more supervisors covering different dimensions of the enterprise. The aim here is to integrate these different dimensions. Pearlson and Saunders state that matrix organization structures are suited for dynamic environments with lots of uncertainty, presumably because their authority structure allows them to cover multiple aspects when making decisions. However, like a hierarchical structure, a matrix structure is a rigid stricture with strictly defined roles, communication lines and decision rights. While there is a more distributed authority structure at the lowest level of management, the upper levels are structure in hierarchical fashion [MAYBE NEED A SOURCE] and remain highly centralized. This has many of the drawbacks of a hierarchical organization, and as a result, this rigidity may prevent this type of organization from effectively adapting to rapidly changing environments. 

[MATRIX DIAGRAMS]

In recent years a new type of organizational structure has emerged, called the networked organization structure~\cite{pearlson2009}. As depicted in figure [DECENTRALIZED FIGURE], a networked structure aims to discard traditional hierarchies in favor of decentralized decision rights and flexible communication networks connecting the entire company~\cite{applegate1988,pearlson2009}. This enables an organization where many (or all) employees are able to easily share knowledge and provide input into the overall decision making for the organization~\cite{pearlson2009}. An important effect of this is a flexible enterprise that promotes creativity. Pearlson and Saunders state that this type of organizational is suitable for dynamic and uncertain environments. This stands to reason, much more so than for matrix  organization structures, as a high level of creativity and flexibility should allow an organization to adapt quickly to changes in its environment. 

[DECENTRALIZED FIGURE]

\subsection{What is a Decentralized Organization?}

As demonstrated in section~\ref{org:form}, whether an enterprise is centralized or decentralized depends on more than simply its structure. Furthermore, enterprises will have elements of both centralization and decentralization in them, meaning that would be an oversimplification to classify an enterprise as just one or another. Consequently, this section will describe a number of organizational characteristics that can be used to determine to what degree an organization is centralized or decentralized. 


definition of decentralized org
    - decision rights/making
        - ability to make change
    - reporting structure
    - communication lines
    - standardization (processes and IT)
    - Dependencies - common (transactions)
    - fluid teams, change at will
    - why the different structures are at different degrees of centralization
    

\subsection{Challenges in Decentralized Organizations}

As decentralized organizations function in a significantly different manner than centralized organizations, they offer a different set of challenges that need to be faced: "~...~the novel and dynamic pressures that create the demand for decentralization in the first place can place organization leaders in considerably less certain, and consequently less commanding, positions."~\cite{caruso2008boundaries} In order to effectively meet these challenges, it is important to first understand what they are. [LIST WHAT I WILL TALK ABOUT]

A paper by Caruso, Rogers and Bazerman~\cite{caruso2008boundaries} highlights the importance of information sharing and coordination for these organizations. In order to succeed at these aspects, they outline three barriers that decentralized organizations need to overcome. The first barrier is intergroup bias; the tendency to treat one's own group better than other groups. The second barrier is group territoriality; the tendency for a group to protect their territory (physical or informational). The third barrier is poor negotiation strategies used by different groups when interacting with one another. 

Intergroup bias is direct result of having separate, autonomous groups within an enterprise. The individual groups have a tendency to promote their own group over other groups, especially in situations where they are competing for a resource, such as a portion of the budget. A certain level of competition can be beneficial, however if it leads to hostility or distrust between groups, this can have a detrimental effect on their ability to share information and collaborate. This can prevent the groups from taking advantage of situations where they have to ability to work together for the benefit of everyone. 

The second barrier identified by Caruso et al. is group territoriality. Group territoriality is characterized by group members 
    
    2 Group Territoriality
    
    
group territoriality – actions undertaken by a group or by individuals on behalf of their group which are designed to reflect, communicate, preserve, or restore the group's psychological ownership of its territory.

This territory will generally include physical space and other tangible objects, as well as any number of intangible objects like activities, roles, issues, ideas, and information.

negative implications for cross-boundary collaboration, as it affords group members a sense of psychological ownership – claims to, or feelings of possessiveness and attachment toward, territorial objects.Groups may begin to see themselves as the sole rightful performers of certain tasks or possessors of certain knowledge, and then hold themselves to those expectations by restricting their activities and information exchange to ingroup members.

inward-looking behavior; it stems from the need to respect and reaffirm the identity, efficacy, and security of the group within the organization. Nevertheless, group territoriality can constitute a significant barrier to emergent intergroup collaboration and information exchange.

group territoriality may manifest as the institution of barriers that discourage outgroup members from even attempting to access group information

Even more forceful territorial actions can arise when a group feels the need to defend or restore its territory.  In such situations, groups seek to actually restrict territorial  access to group members alone and repair breaches to that restricted access.

Initially, groups are likely to enact anticipatory defenses – actions taken prior to any territorial infringement with the purpose of thwarting attempts (e.g., storing or encrypting its information on a password-protected computer). 

If outgroup members overcome anticipatory defenses, groups are  likely to enact reactionary defenses – reactions to territorial infringement intended to undermine the infringement and restore the territory to the group (e.g., discrediting the outgroup’s understanding of the information, and acquiring new, higher-quality information).

Territorial behavior in decentralized organizations is sustained by three important and universal group needs.  
    First, territoriality serves a group’s need to establish, develop, and safeguard a  concept of itself within the organization – a group identity.  Group territory can be useful here  because people can simply look for those who have access to the territory as an indicator of group membership.
    The second group need that undergirds group territoriality is the need to establish and maintain a sense of group efficacy in organizationally-relevant domains. This form of efficacy refers to a group’s belief in its collective ability to organize and perform the activities necessary to achieve desired goals.14 At the broadest level, identification and protection of group territory helps groups to identify the goals they should aspire to achieve. Moreover, when a group’s territory is widely recognized by others, such recognition can serve as an implicit endorsement of the group’s efficacy in related domains.
    The last need group territoriality serves is the need for security within the organizational environment – the need for a relatively stable and familiar “place” from which to solicit, access, and interact with the rest of the organization.  When a group feels secure in its environment, it can more easily develop expectations of and predictions about its environment, facilitating the planning and execution of activities. Moreover, it can relax or eradicate any fears about expulsion from the environment, so its members can make longer-term investments in their work and set more ambitious goals.



    3 Poor Negotiations across the Organization
    
The final barrier to effective cross-boundary information sharing we discuss involves the poor strategies members of different organizational divisions use when they negotiate with one another.

Nevertheless, both parties commonly focus only on the claiming aspect, destroying value for themselves and their organization.

3 main errors:
    False belief of a "Fixed Pie" of value
        Rare that there actually is a fixed pie
        Identify broad set of interests and issues --> make mutually beneficial trade-offs
    the failure to carefully consider the decision processes of one’s negotiation partner
        poorer negotiation outcomes
        to identify mutually beneficial trades:
            understand how (and in what domains) their negotiation partners value issues differently than themselves
            integrate valuing differences into trades --> >negotiations
            if do not consider or inquire about the underlying interests of the other division’s negotiator, it becomes more likely that mutually beneficial trades will never be discovered
                and that the organization will consequently suffer.

            
    the failure to recognize opportunities to negotiate in the first place
        i.e. don't realize they are negotiating
        actually stigmatize it as hostile (competitive?) behavior
        fixed pie --> people don't approach the interaction as a negotiation
    
    
    

\subsection{Principles of Existing Decentralized Organizations}

\subsubsection{Co-Design}
Smart Cities

\subsubsection{Coopetition}

\subsubsection{Virtual Organizations}

\subsubsection{Valve}