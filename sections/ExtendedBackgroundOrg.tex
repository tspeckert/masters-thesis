\subsection{Overview}
start with different styles of organization structure ranging from centralized to decentralized
definition a decentralized organization
barriers to decentralization
principals used by decentralized organizations


\subsection{Why Have Decentralization in Organizations?}
strengths of centralized i.e. hierarchies
define dynamic, uncertain, rapidly changing, stable... environments

\subsection{Different Forms of Organizational Structure}
\label{org:form}

Organizations have traditionally utilized more centralized forms of organizational structure. Of the centralized structures, a hierarchical organization structure is perhaps the most well-known and most utilized. As shown in figure[HIERARCHICAL FIGURE REF], it is characterized by a a hierarchy of positions. Except for the top level position, each position has one supervisor and zero or more subordinates. Decision rights and communication lines are strictly defined and work their way down from the top (i.e. the centre): the scope of a position is specialized and strictly defined by your superior and one works who they assigned to work with. The primary benefit of a hierarchy is that the high levels of management have strict governance and control of everything that goes on in the company. This allows them to easily direct the company how they deem best. [MIGHT NEED SOURCES ABOUT BENEFITS OF HIERARCHY]. Hierarchical organizations generally either divide their labor in terms of function, a grouping of common activities, or in terms of division, a grouping based on output. Due to this, According to Pearlson and Saunders, hierarchical organization structures are suited for stable, certain environments. ~\cite{pearlson2009}

[DRAWBAKCS]

According to Pearlson and Saunders~\cite{pearlson2009}, another structure that is highly centralized is the flat type. A flat organizational structure is characterized by having a single (or small number) of people, frequently owners, at the top. The rest of the employees are all below the top level and are equal to one another. This kind of flat structure is effectively a hierarchical structure with only two levels. A common structure for new companies, Pearlson and Saunders state that is a centralized from of organizational structure as all the power and decision making authority typically is controlled by the person (or small number of people) at the top. They then tell the rest of the employees what to do. [MAYBE MORE AOBUT FLAT FROM PEARLSON] However, this is not always true for flat organizations, it depends on how they are run. For example, Valve Corporation, a software company in the video game industry released their handbook in 2012~\cite{valveHandbook}. In it, they describe their structure as being flat, but a very different style of flat compared to what is described by Pearlson and Saunders. where nobody reports to anyone even though there is a president/founder at the top. Unlike the style of flat organization described by Pearlson and Saunders, at Valve it is a highly decentralized style. Nobody reports to anyone, and everyone is free to work on whatever they want to. Valve states that the company is "yours to steer", meaning that everyone the power to alter the direction of the company. This difference in what a "flat" organization demonstrates is that it is important to take into account more than simply the structure of a organization, how that structure is implemented is equally important. 
[SUITABILITY]

[FLAT DIAGRAMS]

Another popular style of organization structure is the matrix organization structure~\cite{pearlson2009}. In this style, individuals are assigned two or more supervisors covering different dimensions of the enterprise. The aim here is to integrate these different dimensions. Pearlson and Saunders state that matrix organization structures are suited for dynamic environments with lots of uncertainty, presumably because their authority structure allows them to cover multiple aspects when making decisions. However, like a hierarchical structure, a matrix structure is a rigid stricture with strictly defined roles, communication lines and decision rights. While there is a more distributed authority structure at the lowest level of management, the upper levels are structure in hierarchical fashion [MAYBE NEED A SOURCE] and remain highly centralized. This has many of the drawbacks of a hierarchical organization, and as a result, this rigidity may prevent this type of organization from effectively adapting to rapidly changing environments. 

[MATRIX DIAGRAMS]

In recent years a new type of organizational structure has emerged, called the networked organization structure~\cite{pearlson2009}. As depicted in figure [DECENTRALIZED FIGURE], a networked structure aims to discard traditional hierarchies in favor of decentralized decision rights and flexible communication networks connecting the entire company~\cite{applegate1988,pearlson2009}. This enables an organization where many (or all) employees are able to easily share knowledge and provide input into the overall decision making for the organization~\cite{pearlson2009}. An important effect of this is a flexible enterprise that promotes creativity. Pearlson and Saunders state that this type of organizational is suitable for dynamic and uncertain environments. This stands to reason, much more so than for matrix  organization structures, as a high level of creativity and flexibility should allow an organization to adapt quickly to changes in its environment. 

[DECENTRALIZED FIGURE]

\subsection{What is a Decentralized Organization?}

As demonstrated in section~\ref{org:form}, whether an enterprise is centralized or decentralized depends on more than simply its structure. Furthermore, enterprises will have elements of both centralization and decentralization in them, meaning that would be an oversimplification to classify an enterprise as just one or another. Consequently, this section will describe a number of organizational characteristics that can be used to determine to what degree an organization is centralized or decentralized. 


definition of decentralized org
    - decision rights/making
        - ability to make change
    - reporting structure
    - communication lines
    - standardization (processes and IT)
    - Dependencies - common (transactions)
    - fluid teams, change at will
    - why the different structures are at different degrees of centralization
    

\subsection{Barriers in Decentralized Organizations}

barriers
    1 Intergroup Bias
    2 Group Territoriality
    3 Poor Negotiations across the Organization

\subsection{Principles Used in Existing Decentralized Organizations}

CoDesign, Smart Cities

Coopetition

VOs

Valve