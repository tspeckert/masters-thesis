Emerged as technical (~\cite{saroiu2001measurement} )... spread to other domains...



One such effort is peer-to-peer architecture. According to Saroiu, Gummadi and Gribble, peer-to-peer systems ``...typically lack dedicated, centralized infrastructure, but rather depend on the voluntary participation of peers to contribute resources out of which the infrastructure is constructed. Membership in a peer-to-peer system is ad-hoc and dynamic...''~\cite{saroiu2001measurement}. 

%Multiple parallels can be drawn between the problem decentralization for peer-to-peer systems and the problem of decentralization for EA.

We argue that peer-to-peer is a relevant concept to decentralization in EA for two reasons. First, individuals in highly decentralized organization are able to contribute to the enterprise in a manner that is completely up to them. This is similar to peers in a peer-to-peer system, where the peers participate in a completely voluntary manner. Second, the challenge that peer-to-peer systems overcome is similar to the main challenge faced by decentralized organizations. Saroiu et al. state that the challenge of peer-to-peer systems is to ``to figure out a mechanism and architecture for organizing the peers in such a way so that they can cooperate to provide a useful service to the community of users''~\cite{saroiu2001measurement}. This is similar to the main challenge facing decentralized organizations--a lack of interaction and communication, or in other words, cooperation--which was identified in Section \ref{sec:challenge}. 

With EA being a potential solution to this challenge of decentralization in organizations and the parallels between the domains of peer-to-peer systems and decentralized organizations, we propose that peer-to-peer may be a potential source of principles that could form the basis for evolving current centralization-focused EA frameworks into ones that are supportive of decentralization. This section will briefly present and discuss two relevant principles from peer-to-peer.

%
% Theory: 
%

\subsection{Peer production}

Benkler defines peer production as ``...production systems that depend on individual action that is self-selected and decentralized, rather than hierarchically assigned''~\cite{benkler2006wealth}. Here, individuals act according to their own will rather than being directed by a central figure. Peer production works on the idea of the individuals willingly coordinating with one another by expressing their own views while understanding the views of others.

%
%% Examples:
%

Peer production takes many different forms. One example are user-driven media sites such as Reddit\footnote{~\url{www.reddit.com}} and Slashdot\footnote{~\url{www.slashdot.org}}, which follow a peer-production model for producing ``relevance/accreditation''~\cite{benkler2006wealth} on user-submitted content. On these sites, the users have the ability to vote on the submitted content in order to decide on the content's relevance or credibility. Another example of relevance production are crowdfunding sites such as Kickstarter\footnote{~\url{www.kickstarter.com}} where individuals decide on the funding of user-submitted projects by giving their own money. Peer production is also used to produce content, such as in the case of Wikipedia\footnote{~\url{www.wikipedia.org}}, an online encyclopedia which provides a platform for user-driven content submission and change management of that content.

%
%% Why Appropriate:
%

\subsection{Trust management in peer-to-peer}
%
%% Definition:
%
Due to the fact that peers in peer-to-peer systems are able to operate in a completely independent manner, there exists the problem of knowing whether or not the contribution made by a peer is trustworthy or not. Consequently, some researchers have proposed various methods for determining trust in a peer-to-peer environment.
%
%% Examples:
%
For example, Aberer and Despotovic~\cite{aberer2001managing} have proposed determining whether a peer is trustworthy or not based on a peers history of interactions with other peers in the system. This assessment is performed by the individual peers, and as such, is appropriate for a peer-to-peer environment.
%
