% \subsection{Enterprise Structure}
Large enterprises have traditionally implemented formal, centralized forms of organizational structure~\cite{pearlson2009}, such as hierarchical or matrix structures. In these structures, communication patterns and decision rights are strictly defined and controlled by management.  This has the advantage of clarity, in that it is always simple to determine exactly who is responsible for what. This clarity comes at the expense of agility, it is difficult for these organizations to quickly adapt to a changing environment. While they were appropriate for the business environments of the past, modern business environments demand a high level of agility.

Common components of modern business environments include cooperation with different organizations,  rapidly changing business activities and processes, and a rapidly changing competitive landscape. In order to properly handle these components, a high level of enterprise agility is necessary. Instead of having strict hierarchies and formal processes that need to be followed, it is important for enterprises to have a more flexible, decentralized organizational structure that allows for members to make their own decisions and change the way they work. More decentralized structures, such as networked~\cite{pearlson2009} organizations are examples of this. It is important to note that a lack of rigidity and formal structure does not mean a lack of organization. It is still important for a decentralized organization to maintain a level of organization.

In the technical world, distributed computing applications are the solution for maintaining organization in a decentralized environment. These types of applications are becoming increasingly popular in modern computing. For example, one of the most common architectures for distributed computing applications is peer-to-peer: an architecture in which the peers communication directly to each other, without the need for a central server. Peer-to-peer applications are being used in many different areas, such as file sharing \cite{bittorrent}, content distribution \cite{blizzard}, revision control \cite{progit}, and even as a digital currency ~\cite{bitcoin2008}.

On the organizational level, Enterprise Architecture (or EA) is a discipline for creating an architecture for an enterprise. EA takes a holistic view of an enterprise in order to bring its many components (such as goals, strategies, information systems, and processes) into alignment with each other. Many different EA frameworks currently exist, for example The Open Group Architecture Framework (TOGAF)~\cite{togaf9.1} and the Zachman Framework~\cite{zachman}. 

All frameworks address one or more of the following three different aspects: the process of creating an enterprise architecture, how to describe enterprises architecture, and a description of how to actually implement the described architecture. Together, these three aspects form a solution to the problem of how to organize the components of an entire enterprise.

% \subsection{Goals}
% This thesis will have two primary goals. The first goal is to show that current EA techniques are inadequate for decentralized business environments. Assuming the first goal is met, the second goal is to determine principals from the field of distributed computing that can be used to form the basis for EA of the next generation. 

