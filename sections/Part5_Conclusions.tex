\subsection{Current EA Support of Decentralization}

This thesis sought to determine how existing EA frameworks support decentralization, as reflected in the first research question: \textit{What aspects of existing EA frameworks support decentralized organizations? What aspects do not?}. Answering this research question was based on an extensive literature review of three well-known EA frameworks (TOGAF, FEA, and Zachman) and of decentralized organizations (sections \ref{sec:ea} and \ref{sec:organizations}). To be able to then analyze EA for its support of decentralization, a classification of organizations and relevant properties to decentralization was necessary. Figure \ref{fig:taxonomy} describes a taxonomy of organizations ranging from \textit{High Centralization} to \textit{High Decentralization} with \textit{Federated} in the centre and classifies many types of organizations that are described in literature. In Figure \ref{table:org_characteristics}, three key organizational properties that differentiate between centralized and decentralized organizations are outlined: geographical dispersion, coordination (authority, decision rights, standards and regulations), and communication patterns. These properties were then used to qualitatively analyze the three EA frameworks for their support of both centralization and decentralization, summarized in \ref{table:summary}, which is the first of two main contributions of this thesis.

While the analysis uncovered some support for decentralization, the main conclusion drawn is that the EA frameworks of TOGAF, Zachman, and FEA are primarily supportive of centralized and federated organizational structures and therefore fail to address the demands of decentralized organizations. Zachman is unable to support any significant aspect of decentralization due to its reliance on traditional organizational roles and structures on the High Centralization end of the organizational taxonomy. TOGAF does provide some basic support through its ability to have a different architecture for organizational units and by providing space for new methods for the architecture development. It however still mainly relies on hierarchy and central roles responsible for overall coordination and approval. In FEA the conclusions are similar as it primarily supports federated organizational structures where individual units have their own architectures that are coordinated through centralized standards that must be followed.


\subsection{Principles for an EA supportive of Decentralization}

The second goal of this thesis was to design principles for an EA supportive of decentralization, reflected in the second research question: \textit{What are the principles of an EA framework that supports decentralized organizations?}.  This research question was answered through a combination of a short literature review on peer-to-peer architectures and a case study on an institute of higher education in Sweden. The answer to this research question in the thesis question is that the principle of \textit{peer production}, and specifically the following peer production based guidelines for governance \textit{wilful coordination}, \textit{decentralized authority structure}, and \textit{peer decision making}, is a principle for an EA supporting decentralized organizations. 

The relevance of the case was established in section \ref{sec:explicate_iteration2}, where the root causes of a mismatch between the case organization's implicit EA and its organizational structure were detailed. Here, it was found that the case organization's organizational structure exhibited many of the properties of a decentralized organization while the implicit EA remained highly centralized in many aspects, particularly with respect to decision making. 

As part of the ``Design and Develop Artifact'' step (section \ref{sec:design}, the general concept of peer-to-peer architectures was introduced and shown, on a high-level, to have some parallels with EA. Two specific principles of peer-to-peer architectures were then outlined, peer production and peer-to-peer trust management, and shown that they could be used as potential principles for an EA that is supportive of decentralization. 

The principle of \textit{peer production} was selected to form the basis of the solution artifact in section \ref{sec:design_iteration2}. This solution artifact was selected to address the issue that the governance aspects of TOGAF, Zachman and FEA tend to assume vertical coordination (as seen in centralized organizations) over horizontal coordination (as seen in decentralized organizations). The solution artifact, composed of three Architecture Principles for a governance framework based on peer production were then outlined. These principles suggested the replacement of formal compliance measures with \textit{wilful coordination}, to implement a ``decentralized authority structure'' by pushing decision-making authority down to the level of individual business units, and to enable ``peer decision making'', where decisions are made collaboratively by individuals instead of it residing with a central authority. The feasibility of these guidelines were then demonstrated in section~\ref{sec:framework} by outlining the current governance framework (``as-is EA'') and then replacing some of the centralized components with peer production based ones. This answers the second research question and is the second main conclusion and contribution of this thesis: \textbf{principles from peer-to-peer architectures, and specifically the principle of peer production, have the potential to be able to form the basis for an EA framework that is supportive of decentralization.} A weak form of evaluation, ``informed argument'', was performed on the demonstration of the solution artifact. Here, the solution artifact was shown to offer improved the (decentralized) case organization improved governance support and fulfil the initial requirements that the artifact shall support lateral coordination and that the artifact shall support governance activities in an organization.

\subsection{Ethical and Societal Consequences}

In keeping with the ethical issue of maintaining a neutral view of EA theory outlined in Section \ref{sec:ethics}, areas where EA supported and lacked support for decentralization were both identified (table \ref{table:summary}). Furthermore, in the demonstration of the peer production based EA governance framework, a reference demonstration of a centralized artifact based on traditional EA was also performed (tables \ref{table:centralGeneralGovernance} and \ref{table:centralITGovernance}) in order to ensure that traditional EA theory was not ignored. 

The conclusions of this thesis are of interest to three groups: the case organization, researchers in the field of EA, and, potentially, other organizations with decentralized structures interested in implementing some form of EA. 

For the case organization, the developed EA principle of peer production might be of interest, especially because the demonstration of its application was performed using their organization as a model. For them, the application of this principle as demonstrated in tables \ref{table:peerGeneralGovernance} and \ref{table:peerITGovernance} could offer some improvements to their governance structure. If not, the problem explication in section \ref{sec:explicate_iteration2} can be taken as third-party feedback into their organizational structure, which offers them insight from an external perspective. 

For researchers, this thesis work might be of interest as it highlights some potential issues with traditional EA knowledge while giving some initial insight into how they could be solved. These insights are certainly not conclusive; this research should be positioned as a starting point for future research in the topic of decentralization in EA. 

This thesis work might be of interest to organizations that have adopted, or are interested in adopting, a decentralized structure and are looking for insight into how governance can be done in this environment. This thesis would have more of a long-term effect for this group, however, as the ideas presented here likely need to be matured through further research.  