This section will first outline how the results answer the research questions \& the answers themselves. The design science iterations (1st for q1, 2nd for q2).



The section section will discuss the limitations of the thesis and how that impacted the results

%that the limitations in the composition of the thesis and their impact on the conclusions are discussed as well as how the results relate to previous research
%ethical and societal consequences of the conclusions <-- dont have any?

The final section will outline in detail suggestions for relevant future work in this research topic 



\section{Conclusions}
\label{sec:conclusions}

\subsection{EA Support of Decentralization}

This thesis sought to determine how existing EA frameworks support decentralization, as reflected in the first research question (in section \ref{sec:resq}): \textit{What aspects of existing EA frameworks are supportive of decentralized organizations? What aspects are not supportive?}. Answering this research question was based on an extensive literature review of three well-known EA frameworks (TOGAF, FEA, and Zachman) and of decentralized organizations (sections \ref{sec:ea} and \ref{sec:organizations}). To be able to then analyze EA for its support of decentralization, a classification of organizations and relevant properties to decentralization was necessary. Figure \ref{fig:taxonomy} describes a taxonomy of organizations ranging from \textit{High Centralization} to \textit{High Decentralization} with \textit{Federated} in the centre and classifies many types of organizations that are described in literature. In Figure \ref{table:org_characteristics}, three key organizational properties that differentiate between centralized and decentralized organizations are outlined: geographical dispersion, coordination (authority, decision rights, standards and regulations), and communication patterns. These properties were then used to qualitatively analyze the three EA frameworks for their support of both centralization and decentralization, summarized in \ref{table:summary}, which is the first of two main contributions of this thesis.

While the analysis uncovered some support for decentralization, the main conclusion drawn is that the EA frameworks of TOGAF, Zachman, and FEA are primarily supportive of centralized and federated organizational structures and therefore fail to address the demands of decentralized organizations. Zachman is unable to support any significant aspect of decentralization due to its reliance on traditional organizational roles and structures on the High Centralization end of the organizational taxonomy. TOGAF does provide some basic support through its ability to have a different architecture for organizational units and by providing space for new methods for the architecture development. It however still mainly relies on hierarchy and central roles responsible for overall coordination and approval. In FEA the conclusions are similar as it primarily supports federated organizational structures where individual units have their own architectures that are coordinated through centralized standards that must be followed.


\subsection{What are the principals of an EA framework that is supportive of decentralized organizations?}





FUNCTIONS OF A CONCLUSION

1.To summarize
 – What you researched 
 – Nature of your main arguments 
 – How you researched it 
 – What you discovered 
 – What pre-existing views were challenged
2.To provide an overview of 
•The new knowledge or information discovered
•The significance of your research (where is it new?)
•The limitations of your thesis (concepts, data)
•Speculation on the implications of these limitations
•Areas for further development and research(alternative data sets; links with other fields; different method applied to same data)


DOs

You must 
1.Make a clear and concise statement of the original contribution to knowledge found in your thesis.

Ideally you should aspire to
1.Show links between the key ideas spread across chapters
2.Show your commitment to and enthusiasm for academic research
3.Leave a positive impression with the examiner


DONTS

Avoid claiming findings that you have not proven throughout your thesis
Avoid introducing new data
Avoid hiding weaknesses or limitations in your thesis(make a virtue of showing strong analytical skills and self-critique by discussing the limitations--but don't go overboard on this!
Avoid making practical recommendations (e.g. for policy). If you must include them put them in an appendix.
Avoid being too long (repetitive) or too short (saying nothing of importance)


SAMPLE STRUCTURE

One paragraph stating what you researched and what your original contribution to the field is…then break into sections
One section on what you researched and how you did it
One section on what are the main findings were… showinglinks across chapters (this explains why you chose thestructure you did)
One section on possible areas for future research•Final section reminding readers of the original contributionand significance of your research to your field