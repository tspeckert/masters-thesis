\begin{enumerate}
\item What aspects of existing EA frameworks are supportive of decentralized organizations? What aspects are not supportive?
\label{req:1}
\item What are the principals of an EA framework that is supportive of decentralized organizations?
\label{req:2}
\end{enumerate}



FUNCTIONS OF A CONCLUSION

1.To summarize
 –What you researched –Nature of your main arguments –How you researched it –What you discovered –What pre-existing views were challenged
2.To provide an overview of 
•The new knowledge or information discovered•The significance of your research (where is it new?)•The limitations of your thesis (concepts, data)•Speculation on the implications of these limitations•Areas for further development and research(alternative data sets; links with other fields; differentmethod applied to same data)


DOs

You must 
1.Make a clear and concise statement of theoriginal contribution to knowledge found in your thesis.

Ideally you should aspire to
1.Show links between the key ideas spread acrosschapters
2.Show your commitment to and enthusiasm for academic research
3.Leave a positive impression with the examiner


DONTS

Avoid claiming findings that you have not proventhroughout your thesis
Avoid introducing new data
Avoid hiding weaknesses or limitations in your thesis(make a virtue of showing strong analytical skills and self-critique by discussing the limitations--but don’t gooverboard on this!
Avoid making practical recommendations (e.g. for policy).If you must include them put them in an appendix.
Avoid being too long (repetitive) or too short (sayingnothing of importance)


SAMPLE STRUCTURE

One paragraph stating what you researched and what your original contribution to the field is…then break into sections
One section on what you researched and how you did it
One section on what are the main findings were… showinglinks across chapters (this explains why you chose thestructure you did)
One section on possible areas for future research•Final section reminding readers of the original contributionand significance of your research to your field