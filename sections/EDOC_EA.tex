\subsection{Using EA for Decentralized Enterprises}
The field of Enterprise Architecture (EA) emerged in order to combat two increasingly prevalent problems facing enterprises: system complexity and business-IT alignment~\cite{sessions2007}. As enterprises rely more and more on information systems of increasing complexity, these problems become even more important. The field of EA views the solution to these problems to be one of concurrent design. It is not enough simply try and fit IT to the business; business and IT aspects should be designed concurrently.

....EA is an enabler of collaboration, which as show earlier is an important concept to decentralized enterprises...however 
   
As shown in the previous section, collaboration important to decentralized enterprises: need way to collaborate
    
\textbf{The Problem: }EA could be that solution, however it has shortcomings for Decentralized Enterprises. But first, we present a taxonomy for EA:

\subsection{A Common Perspective on Enterprise Architecture}
While there is no singular agreed-upon definition for EA, different definitions\cite{jungle2004,GartnerInc,ross2006,pearlson2009,lankhorst2009,sessions2007,togaf9.1} do have much in common. EA is a discipline that takes a holistic, design-oriented approach to transforming high-level business vision and goals into the integration of an enterprise's organizational structure, business processes, and information systems. This transformation involves identifying and implementing the necessary change for this to occur. In order to view different Enterprise Architectures from a common perspective, this paper will break the frameworks down into three separate components: the preliminary phase, the EA description phase, and the EA engine phase. 

The preliminary phase aims to lay the groundwork for the EA process. Typically, this involves setting up teams, ownership, responsibilities and gaining commitment. The EA description phase is involved with creating the actual architecture artifacts (eg. models, principles) themselves. This generally involves outlining an enterprises EA principles, its as-is EA description,  the planned to-be architecture, an analysis of the gap between the as-is and to-be architectures and a plan to cross that gap. The engine  phase involves setting up a support structure for ensuring the ongoing adoption of the to-be EA description. This can involve gaining commitment from stakeholders, setting up some compliance checking procedures, and deciding upon a prioritization of tasks to be completed. The remainder of this section will look at three different EA frameworks from the perspective of of these three phases: The Open Group Architecture Framework (TOGAF), the Zachman Framework, and the Federal Enterprise Architecture (FEA)

\subsection{TOGAF}
\subsection{Zachman}
\subsection{FEA}



    

