\subsection{Iteration 2}
\subsubsection*{Choose Evaluation Strategy}
\label{section:choose_eval}

Implementing the proposed solution artifact is not practical or even feasible in the course of a Master's thesis. The time constraints for such an implementation are better suited to a Doctoral thesis where the researcher is able to implement and evaluate the artifact over a longer period of time (i.e. over the course of several years). For this reason, an argumentative evaluation is performed instead of an ex post evaluation in order to determine if the solution artifact, the guidelines of ``willful coordination'', ``decentralized authority structure'', and ``peer decision making'', would solve the problem explicated in section~\ref{sec:exproblem}. The artifact's demonstration (the Peer Production Governance Framework presented in tables \ref{table:peerGeneralGovernance} and \ref{table:peerITGovernance}), however, can be compared to the As-Is Governance Framework (tables~\ref{table:as-isGeneralGovernance} and \ref{table:as-isITGovernance}) and the Centralized Governance Framework (tables~\ref{table:centralGeneralGovernance}  and \ref{table:centralITGovernance}), to argumentatively evaluate how well they support the governance activities of a decentralized organization. 

In keeping with the principles of design science, it is also important that the proposed Peer Production Governance Framework fulfills the requirements outlined in table \ref{table:requirements}. This is also argumentatively evaluated. 

If the artifact's demonstration fulfills these requirements and can be shown to support a decentralized organization (i.e. the case), then we will take this as a weak form of positive confirmation  that the artifact itself can be said to fulfill the prescribed requirements and support decentralization.

\subsubsection*{Carry Out Evaluation}

\paragraph*{Improved Governance Support of the Proposed Peer Production Governance Framework}

The peer production based framework differs from the as-is and centralized governance frameworks in a number of facets. In terms of decision rights allocation, the peer production based framework keeps the same operating principle as the as-is framework -- to push decision making as far down to the operational level as possible -- whereas a centralized framework would instead keep decision making in the upper levels of management. As most of the staff are operationally-focused (e.g. professors, researchers, and PhDs), this would likely be a sub-optimal situation. 

Budgeting in both the as-is and centralized frameworks is centralized, which is in conflict with the decentralized IT management. In the peer production based framework, a mix between  centralized and decentralized is instead proposed. Here, the budget is still allocated centrally, however the department then has complete control over their allocated amount. This could potentially remove the conflict as the department would then have the necessary control over their own budget to operate in a manner decentralized from the rest of the organization. 

The setting of strategy and overall operating principles for the department is centralized in both the centralized and as-is frameworks. In the peer production framework, a mixed style of governance has been proposed in its place. The decentralized component is that the department members produce the operating principles and strategy using peer production while the centralized component is that approval is still needed from the faculty. This offers the advantage of allowing the department to set a strategy that reflects all aspects of the department (by having input from everyone instead of just the department board) that is still compatible with the rest of the university (as approval is needed by the faculty).

The suggested peer production framework is significantly different from the as-is and centralized frameworks. It seeks to maintain the departmental-independence prevalent in the as-is framework while addressing the incompatible architecture components that this results in. This is primarily accomplished through the cooperative classification of essential and non-essential systems, and the difference in governance for the two types. This classification is done in a collaborative manner by the departments, for example by giving each department a vote. Systems classified as essential are required to be used or integrated with by the departments while departments have the option to choose if they want to utilize systems classified as non-essential. These changes are addressed at reconciling differences between the two architecture principles identified previously identified without actually changing the principles. Decision rights are still pushed down and systems are still integrated throughout the organization, but this change in governance addresses the conflict that can arise when a decision is made to implement a decentralized system that the rest of the organization is integrating (as it stands in the as-is framework). In comparison, shifting to centralized governance (as in the centralized governance framework) would require changing the principle of pushing decision making down to the operational level. As this is a core principle of the organization, this change would have a significant impact across the entire organization, with no guarantee of it being a positive one. 

\paragraph*{Fulfillment of Initial Requirements}

The first requirement is that the artifact supports the lateral coordination characteristic of a decentralized organization, and the decentralized authority structures and heterogeneous goals that go along with it. The three guidelines of the artifact -- being based on peer production -- inherently support lateral coordination over vertical coordination. The core characteristic of peer production is that a number of peers work together without a central coordinating authority (as it is in vertical coordination), which supports lateral coordination. The artifact supports decentralized authority structures and decision making (requirement 1.1) by suggesting that individual operational units are responsible for their own decision making. Furthermore, it suggests that the individuals in such a department make decisions collaboratively by giving each individual a vote for major decisions. This further supports heterogeneous goals (requirement 1.2) as each individual (and their respective views and goals) is able to contribute to making decisions.

TOGAF and FEA support governance activities through the use of compliance checking, standards and approval to ensure interoperability and integration of architecture artifacts. The proposed artifact instead suggests less formal style of governance which relies on individuals willfully coordinating, an authority structure that gives them the freedom to act as they feel is best, and decisions that are made as peers rather than by management.

Being general guidelines, the specifics of how the artifact ensures interoperability and integration of EA components depends on the specific case. An example that fulfills the requirement of supporting governance activities is in the demonstration of the artifact. Here it is proposed that the governance of systems (and therefore the EA artifacts that they result from) is accomplished by collaboratively classifying them as either essential and non-essential. The specific guidelines for interoperability are different for the two classifications. This is a style of governance that follows the guidelines of willful coordination, decentralized authority structure and peer decision making: individual business units come together to decide on the classification of essential systems, and the implementation of non-essential systems is completely up to discretion of the individual business units. 
