% \subsection{Enterprise Structure}
Large enterprises have traditionally implemented formal, centralized forms of organizational structure~\cite{pearlson2009}, such as hierarchical or matrix structures. In these structures, communication patterns, roles and decision rights are strictly defined. This allows for management to have a high degree of control over the enterprise and therefore enforce compliance with standards, procedures and policies which results in a highly stable enterprise. However, this comes at the expense of agility; it is difficult for these organizations to quickly adapt to a changing environment. While centralized structures were appropriate for the business environments of the past, modern business environments demand a high level of agility.

Common components of modern business environments include cooperation with different organizations,  rapidly changing business activities and processes, and a rapidly changing competitive landscape. In order to properly handle these components, a high level of enterprise agility is necessary. In centralized organizations, decisions need to be discussed at all levels of the hierarchy in order to obtain the appropriate justification and approval. This takes time; by the time a decision is made, it is often too late for it to be effective. In contrast, having decision making on the operational level allows for quick decisions that enables an organization to take advantage of opportunities quickly. More decentralized structures, such as networked organizations~\cite{pearlson2009}, are examples of this. It is important to note that a lack of rigidity and formal structure does not mean a lack of organization. It is still important for a decentralized enterprise to maintain order in its activities; this organization just needs to be based on an underlying decentralized structure instead of centralized one. Consequently, decentralized organizations need solutions to the same problems faced by centralized organizations -- such as business-IT alignment -- but the solutions need to be supportive of decentralization over centralization. 

On the organizational level, Enterprise Architecture (or EA) is a practice for creating an architecture for an enterprise. EA takes a holistic view of an enterprise in order to bring its many components (such as goals, strategies, information systems, and processes) into alignment with each other. Many different EA frameworks currently exist, for example The Open Group Architecture Framework (TOGAF)~\cite{togaf9.1} and the Zachman Framework~\cite{zachman}. 

All frameworks address one or more of the following three different aspects: the process of creating an enterprise architecture, how to describe enterprises architecture, and a description of how to actually implement the described architecture. Together, these three aspects form a solution to the problem of how to organize the components of an entire enterprise.

Enterprises are not alone in their trend towards decentralization. In the technical world, for example, peer-to-peer applications are becoming increasingly popular. Peer-to-peer applications follow an architecture in which peers communication directly to each other, without the need for a central server. These applications now exist in many different areas, such as file sharing \cite{bittorrent}, content distribution \cite{blizzard}, revision control \cite{progit}, and even as a digital currency ~\cite{bitcoin2008}. 



% \subsection{Goals}
% This thesis will have two primary goals. The first goal is to show that current EA techniques are inadequate for decentralized business environments. Assuming the first goal is met, the second goal is to determine principals from the field of distributed computing that can be used to form the basis for EA of the next generation. 

