Organizations with rigid centralized management style (linear or pyramidal hierarchies) fail to sustain the dynamic environments due to their inertia in decision making and lack of agility. Political, social and economic systems progressively transform to distributed networks (Michael Bowens, http://www.ctheory.net/articles.aspx?id=499) and novel organization forms are emerging. 
The term "liquid enterprise" coined recently, describes the nature of such organisations. Transparent or dynamically changing boundaries, agile processes\cite{agileproc}, virtual collaborations, coopetition \cite{coopetition} – all of those are technology-enabled capabilities of an organization of the future. 

Decentralization of organizations and subsequent change of their management and operation style requires major changes in organization processes and heavily involves  the IT. 

Ross et. al. \cite{ross2006} define a term foundation for execution to address \textit{\"the IT infrastructure and digitized business processes automating a company’s core capabilities\"}. While emerging technologies serve the main catalyst for organizational transformations, embracing the “right” technologies and evolving the existing foundation for execution accordingly - is primordial for organizations.  IF WE KEEP THIS TERM - WE NED TO USE  "FOUNDATION FOR EXECUTION" CONSISTENTLY IN THE REST OF THE PAPER. OTHERWISE THIS PARAGRAPH CAN BE DELETED

Traditionally, such questions are addressed by the enterprise architecture (EA) discipline. 
EA \textit{"defines the underlying principles, standards, and best practices according to which current and future activities of the enterprise should be conducted"} \cite{schekkerman2003}. EA methods and tools serve:
\begin{itemize}
\item to specify the current state of the company's foundation for execution (FFE)  - \textit{architecture as is};
\item to identify the target architecture – \textit{architecture to-be};
\item to analyze the gap and set up a master plan for achieving this target – \textit{architectural principles, architecture roadmap}.
\end{itemize}
These artifacts are addressed as EA description; the process that organization has to execute in order to obtain its EA description is called EA method. Traditional EA project, though, consists in implementing the EA method and producing the EA description. To assure that the organization will continuously follow the principles and achieve the designated goals after the termination of EA project – the third element has to be defined. We call this element EA engine, referring to its capacity to stir the company. \footnote{In \cite{ross2006}, this element is addressed as “engagement model”.}

In \cite{sachdeva1990}, organizational structure is defined as "... institutional arrangements and mechanisms for mobilizing human, physical, financial and information resources at all levels of the system..." Numerous taxonomies of organizational types are defined in the literature


Many modern organizations are trending towards decentralization; where organizational units and individuals are being given an increasing amount of autonomy and control over an enterprise's direction and operations. In order to classify these enterprises, this paper presents a taxonomy for categorizing these organizations on a spectrum ranging from centralized to decentralized. 

Decentralization transforms the role of company's authority and makes relationships inside and between different company's divisions much more complex. Planning and governance in different functional areas, including IT,  is not anymore ensured centrally. As a consequence, more efforts are required to prioritize initiatives, coordinate and communicate decisions, manage projects, and evaluate results. Moreover, the practices of management, coordination, communication and decision making are not the same as before: collaboration and information sharing gain extreme importance.

The de-facto EA methodologies rely on organizational properties such as centralized management, global company identity, etc.  that are getting obsolete with progressive decentralization.  Consequently, implementation of these methodologies in decentralized organization becomes difficult and inefficient and  the role of EA as a driver for IT transformations is getting compromised.

A company Y acquired a software system  with an objective of integrated support of XXX across divisions that costed them  XXX \$. Divisions  were not (only partially?) involved into decision making process and product evaluation (decision was made centrally) and eventually refused to shut down their local systems and switch to the global one (decentralized IT management). As a consequence, strategic initiative for integration failed; divisions managed to protect their interests (preserving local systems that are tailored for their needs) however  got charged for the acquired system they never used (centralized budgeting). This example demonstrates a "good" decision made by "wrong" people... Is it a sign of an inappropriate EA? [NEED TO WORK ON IT..]

%In order for these EA frameworks to continue to be suitable for the modern enterprise, some changes need to be made to take into account these new enterprise structures. 


%They are taking for granted some organization and ICT properties (e.g. culture) that do not exist any more. As a result, the attempts to implement the EA methods or to follow the traditional principles meet the hostility in the organization (parts) and often fail.

Therefore,  novel EA processes, principles and concepts are needed to both handle the ICT resources and to foster business/ICT co-evolution in decentralized environment. This paper presents the literature review that supports our claim. 

The main contribution of this paper is ........


The reminder of this article is organized as follows: In Section II, we discuss the role of EA in organization and provides an overview of three EA methodologies: TOGAF, Zachman and FEA;  in Section III, we discuss different forms of organizational structure and the trend towards decentralization. We conclude this section by highlighting the 4 (?) challenges related to  decentralization in IT. In Section IV we analyze the EA methodologies presented in Section II focusing on their support to decentralization. We identify (5?? - a concrete number may sound good..) In Section V, we propose ....... 

% 
% Related EDOC aspects
%     Enterprise architecture frameworks 
%     (Collaborative development and cooperative engineering issues)
%     Enterprise interoperability, collaboration and its architecture
%     ([Cross-enterprise collaboration] in a world of cloud, social and big data)
% 
% 
% 
% 
% 
% 
% 
% 
% 
% 
% 
% 
% 
% 
% 
% 