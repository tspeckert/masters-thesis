 Organizations with rigid centralized management style (linear or pyramidal hierarchies) fail to sustain the dynamic environments due to their inertia in decision making and lack of agility. Political, social and economic systems progressively transform to distributed networks (Michael Bowens, http://www.ctheory.net/articles.aspx?id=499) and novel organization forms are emerging. 
The term "liquid enterprise" coined recently, describes the nature of such organisations. Transparent or dynamically changing boundaries, agile processes\cite{agileproc}, virtual collaborations, coopetition \cite{coopetition} – all of those are technology-enabled capabilities of an organization of the future. 
Ross et. al. \cite{ross2006} define a term foundation for execution to address \textit{\"the IT infrastructure and digitized business processes automating a company’s core capabilities\"}. While emerging technologies serve the main catalyst for organizational transformations, embracing the “right” technologies and evolving the existing foundation for execution accordingly - is primordial for organizations. 

Traditionally, such questions are addressed by the enterprise architecture (EA) discipline. EA \textit{"defines the underlying principles, standards, and best practices according to which current and future activities of the enterprise should be conducted"} \cite{schekkerman2003}. EA methods and tools serve:
\begin{itemize}
\item to specify the state of the art of the company’s foundation for execution (FFE)  - \textit{architecture as is};
\item to identify the target architecture – \textit{architecture to-be};
\item to analyze the gap and set up a master plan for achieving this target – \textit{architectural principles, architecture roadmap}.
\end{itemize}
These artifacts are addressed as EA description; the process that organization has to execute in order to obtain its EA description is called EA method. Traditional EA project, though, consists in implementing the EA method and producing the EA description. To assure that the organization will continuously follow the principles and achieve the designated goals after the termination of EA project – the third element has to be defined. We call this element EA engine, referring to its capacity to stir the company. \footnote{In \cite{ross2006}, this element is addressed as “engagement model”.}

The de-facto EA standards such as Zachman, TOGAF, FEA are grounded on organizational structures (i.e. hierarchies; centralized management; global company identity) that are getting obsolete with progressive decentralization. They are taking for granted some organization and ICT properties (e.g. culture) that do not exist any more. As a result, the attempts to implement the EA methods or to follow the traditional principles meet the hostility in the organization (parts) and often fail.

We claim that the mission of EA in organizations of today is to assist their movement towards decentralization. Therefore, novel EA processes, principles and concepts are needed to both handle the ICT resources and to foster business/ICT co-evolution in decentralized environment.

This paper presents the literature review that supports our claim. 

xxxxxx


The list of related works to consider in details:

TOGAF \\
Zachman \\
FEA?? \\
organizational types, examples  \\
works on org. issues related to decentralization (negotiation, collaboration, etc) \\
Coopetition\\
Social Peer-to-Peer \\
Centralization-decentralization in networking: star (centralized), decentralized, distributed \\
Content-delivery network \\
Tools: \\
GitHub \\
decisionlens?\\
kiskstarter\\
