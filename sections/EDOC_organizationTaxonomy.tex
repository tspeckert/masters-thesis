\subsection{Overview}
There has been a lot of research on specific forms of organizational structure. Consequently, forms such as hierarchical, flat, matrix, networked and adhocracy have been very well defined. Less well defined is what exactly makes an organization centralized or decentralized. To answer this question, this section will first present the well-defined forms of organizational structure. Second, the (de)centralization of current styles of IT governance will be explored. Third, relevant principles from existing decentralized organizations and organizational science will be outlined. By then bringing together these three areas, this paper will then present a number of characteristics for defining decentralization in an enterprise. 

\subsection{Specific Forms of Organizational Structure}
\label{org:form}
%Organization have traditionally utilized more centralized forms of organizational structure. Of the centralized structures, a hierarchical organization structure is perhaps the most well-known and most utilized.

Perhaps the most well-known and traditional form of enterprise structure is the hierarchical form. As shown in figure[HIERARCHICAL FIGURE REF], it is characterized by a a hierarchy of positions. Pearlson and Saunders offer a thorough description of a pure hierarchical organization structure~\cite{pearlson2009}: Except for the top level position, each position has one superior and zero or more subordinates. Decision rights and communication lines are strictly defined and work their way down from the top (i.e. the centre). The scope of a position is specialized and strictly defined by your superior and one works in assigned teams. The primary benefit of a hierarchy is that the high levels of management have strict governance and control of everything that goes on in the company. This allows them to easily direct the company how they deem best. Hierarchical organizations generally either divide their labor in terms of function, a grouping of common activities, or in terms of division, a grouping based on output. Due to this, hierarchical organization structures are suited for stable, certain environments. 

[HIERARCHY DIAGRAM]

Another popular style of organization structure is the matrix organization structure~\cite{pearlson2009}. In this style, individuals are assigned two or more supervisors covering different dimensions of the enterprise. The aim here is to integrate these different dimensions. Pearlson and Saunders state that matrix organization structures are suited for dynamic environments with lots of uncertainty, presumably because their authority structure allows them to cover multiple aspects when making decisions. However, like a hierarchical structure, a matrix structure is a rigid construct with strictly defined roles, communication lines and decision rights. Authority still comes from the top in a centralized manner, even though it becomes more distributed among matrix managers at the lower levels~\cite{pearlson2009}. Consequently, matrix structures still may not be perfectly suited for uncertain, dynamic environments. 

[MATRIX DIAGRAM]

Applegate, Cash, and Mills~\cite{applegate1988} support this statement, as they describe hierarchical and matrix structures as rigidly structuring "communication, responsibility, and accountability to help reduce complexity and provide  stability". They furthermore state that both matrix and hierarchical structures have the effect of stifling creativity and preventing organizations from being able to adapt effectively to rapidly changing environments. 

[DRAWBACKS]

According to Pearlson and Saunders~\cite{pearlson2009}, another structure that is highly centralized is the flat type. A flat organizational structure is characterized by having a single (or small number) person at the top. The rest of the personnel are all below the top level and are equal to one another. This kind of flat structure is effectively a hierarchical structure with only two levels. A common structure for new companies, Pearlson and Saunders state that is a centralized from of organizational structure as all the power and decision making authority typically is controlled by the authority at the top. This is consistent with the simple structure outlined by Mintzberg where "Mom or pop constantly monitors what is going on and exercises total authority over daily operations"~\cite{Mintzberg1979}. However, this is not always true for flat organizations, it depends on how they operate. For example, Valve Corporation, a software company in the video game industry released their handbook in 2012~\cite{valveHandbook}. In it, they describe their structure as being flat, but a very different style of flat where individual employees have complete freedom despite there being a president/founder at the top. Unlike the style of flat organization described by Pearlson and Saunders, at Valve it is a highly decentralized style. Nobody reports to anyone, and everyone is free to work on whatever they want to. Valve states that the company is "yours to steer"~\cite{valveHandbook}, meaning that everyone the power to alter the direction of the company. This difference in what a "flat" organization demonstrates is that it is important to take into account more than simply the structure of a organization, how that structure is implemented is equally important. 

[FLAT DIAGRAM]

In recent years a new type of organizational structure has emerged, called the networked organization structure, or adhocracy~\cite{applegate1988,pearlson2009}. As depicted in figure [NETWORKED FIGURE], a networked structure aims to discard traditional hierarchies in favor of decentralized decision rights and flexible communication lines connecting the entire enterprise~\cite{applegate1988,pearlson2009}. Specifically, instead of hierarchies, an adhocracy has a rapidly changing set of project oriented groups. Mintzberg describes an adhocracy as "a loose, flexible, self-renewing organic form tied together mostly through lateral means"~\cite{Mintzberg1979}.  Regardless of the specifics of its definition, this form of organizational structure is clearly vastly different from a rigid, centralized structure. This enables an organization where many (or all) employees are able to easily share knowledge and provide input into the overall decision making for the organization~\cite{pearlson2009}. An important effect of this is a flexible enterprise that promotes creativity. Together, these characteristics make an organization that is suitable for dynamic and uncertain environments due to its ability to adapt quickly.

[NETWORKED FIGURE]

\subsection{IT Governance}

Rockart, Earl and Ross~\cite{Rockart1996} describe a continuum of governance styles ranging from centralized to decentralized, with federalism in the middle. This style of Federal IT aims to have the strengths of both centralized and decentralized while eliminating their respective weaknesses. Some key weaknesses of centralized IT to eliminate are slow responsiveness and having systems that do not fit the needs of individual business units. Decentralized IT on the other hand lacks "synergy and integration"~\cite{Rockart1996} due to a lack of standardization. Federal IT would aim to balance these through a combination of central IT and IT in the business units. A primary task of the central IT would be to maintain standards for the entire enterprise. The business units would still have ownership of many of their own systems, allowing them to implement them as they deem best. This allows for systems that meet the individual business units needs, as well as enabling interoperability throughout the enterprise. 

Weill~\cite{Weill2004} also describes a federal style of IT governance as part of a larger description of IT governance archetypes and the major types of IT decisions that they apply to. In addition to the federal archetype, he describes a business monarchy, IT monarchy, feudal, IT duopoly and anarchy. In a business monarchy all IT related decisions are made in a centralized manner by the top-level executives (e.g. the CxOs). In an IT monarchy, a group of IT professionals are responsible for making the decisions. This is also highly centralized as the authority resides with this group. An IT duopoly is characterized by two groups, one of IT executives and the other of business executives, coming to agreements in order to make decisions. This is more centralized than the federal form, as the decisions are only made by the two groups, rather than each individual business unit having input. The feudal is much less centralized. It is where individual organizational units are responsible for their own decisions. Anarchy is a highly decentralized style of governance. It is similar to the feudal archetype, however the size of the units is much smaller. Instead of being an entire business unit, small teams or even individuals are responsible for their own decisions.

Weill then proposes five major IT decision domains that these decisions apply to: IT principles, IT architecture, IT infrastructure strategies, business application needs, and IT investment~\cite{Weill2004}. IT principle are high level statements about the use of IT in the enterprise. IT architecture decisions relate to the policies and rules that describe how IT is to be sued, as well as the roadmap for implementing them. Decisions on IT infrastructure strategies relate to the foundation IT services (e.g. network, help desk) that exist throughout the entire enterprise.  Decisions on business application needs are about determining what needs IS will fulfill. IT investment decisions are related to financing and justification of IT projects.

% not certain here if these references should go to Weill or to his referenced papers. 



\subsection{Existing Decentralized Organizations}

\subsubsection{Smart Cities and the Principle of Co-Design}

Smart Cities is a collaborative project between a number of governments and universities seeking to achieve excellence in the field of e-services~\cite{Cities}. One of their publications explored a concept called co-design and how it contributes to the delivery of e-services. 

% explain how smart cities is decentralized

The general idea of co-design, according to Smart Cities, is to bring all stakeholders into the decision making process as equals. The ideal of co-design is to actively involve all stakeholders (including users) of a new system or service to participate in defining what it should do, the process to develop it, and to provide acceptance that the end result functions as it should. Co-design is seen as a way to tackle the many conflicting views and goals that stakeholders will have. 

There are number important characteristics to proper co-design as viewed by Smart Cities. First, it is a completely transparent collaborative process where all participants are able to contribute. Second, the focus of co-design is on both developing the system or service, and on and improving the development process. Third, co-design involves shifting "power to the process"~\cite{Cities} instead of only on a subset of stakeholders.

Smart Cities views co-design activities in three different dimensions; horizontal, vertical, and intensity. Horizontal co-design is done with partner organizations in order to deliver services. Vertical co-design involves stakeholders at various levels in the "service delivery chain"~\cite{Cities}, end-users or other departments. The third dimension is the intensity or degree in which the co-design participants can contribute. 

% may want to link to other literture on co-design

\subsubsection{Coopetition}

The term "coopetition" is a portmanteau of the terms "competition" and "cooperation", used to describe a relationship between parties where they both collaborate and compete. Bengsston and Kock describe coopetition as a complex relationship between firms where they simultaneously compete and collaborate and benefit from both~\cite{Bengtsson2000}. They explore the concept of coopetition in the context of competing firms that "produce and market the same product".~\cite{Bengtsson2000} In this context, they place an important limit on coopetition: their needs to be some kind of separation between the competing and cooperating aspects i.e. they can not "coopetate" in the same aspect. 

An advantage of coopetition is that it allows the participating organizations to take advantage of a heterogeneity of resources~\cite{Bengtsson2000}. Organizations may seek to create competitive advantage through a unique resource they own (e.g. skill). At the same time, it might be beneficial for them to cooperate with another organization that possesses a unique resource that is of value to them. Together, these two factors can lead to a relationship of coopetition that allows the participants to develop in new areas. 

An example of coopetition is Amazon.com's Marketplace; a platform provided by Amazon where any  competitor can list items for sale alongside Amazon's own sales, often of the same items~\cite{UnknownAskIrina,Amazon.com}. This allows sellers to take advantage of Amazon's platform while Amazon takes advantage of the increase in traffic. 
 
% relationship to decentralized:
% Coopetition is a way two entities interact with each other; no overseeing central auuthority


\subsubsection{Virtual Organizations}

Related to the idea of a networked organization structure is the concept of collaborative networks (CN). Camarinha-Matos and Afsarmanesh define collaborative networks as being composed of "a variety of entities (e.g., organizations and people) that are largely autonomous, geographically distributed, and heterogeneous in terms of their: operating environment, culture, social capital, and goals."~\cite{Camarinha-Matos2005} These entities then collaborate through the use of some sort of computer network in order to achieve common or complementary goals. The main driver behind CNs is that the goals the seek to achieve would be impossible or much more difficult to achieve without collaboration. The composition of autonomous entities makes CNs a very relevant concept to decentralization. 

Some examples of collaborative networks are virtual organizations, a group of independent organizations working together to achieve some goal(s); virtual communities, a community of individuals that interact with each other through the use of computer network-based technologies; and virtual breeding environments, a group of organizations that set up a framework for inter-operability in order to enable the potential for forming a virtual organization~\cite{Camarinha-Matos2005}. Three common characteristics in various CNs are autonomy in the individual entities, a drive towards meeting common or complementing goals, and the use of an agreed-upon framework for collaboration. 

% VO
% \cite{Kerschbaum2009}
% "Autonomy of participating organizations in the VO must be maintained."
% no global state in our control-flow enforcement in choreographies

% VE \cite{Camarinha-Matos1999}
% flexible/reactive systems are important

% Virtual Communities

\subsection{What is a Decentralized Organization?}

As demonstrated in section~\ref{org:form}, whether an enterprise is centralized or decentralized depends on more than simply its structure. Furthermore, enterprises will have elements of both centralization and decentralization in them, meaning that would be an oversimplification to classify an enterprise as just one or another. Consequently, organizational structure is best viewed as being on an organizational continuum, with decentralization on one end, centralization on the other end, and federalism in the middle~\cite{pearlson2009}. This section will describe a number of organizational characteristics that can be used to determine to what degree an organization is centralized or decentralized. 

The first characteristic are the allocation of decision rights in an enterprise, specifically who has the right to make what kinds of decisions in running and planning the enterprise.~\cite{pearlson2009} Decision rights are what controls the overall direction of an enterprise. They are needed in order to make change. In a completely centralized enterprise, all decision making authority would reside with a single, top-level authority. More decentralized enterprises would allow for participation in the decision making process by members throughout the enterprise. In a completely decentralized enterprise all members would have equal decision making rights. 

A second characteristic is the structure of communication lines in an enterprise. A centralized enterprise will have rigid hierarchies of communication, i.e. it is strictly defined who you work with and who you report to. Decentralized enterprises instead have less formalized communication lines~\cite{pearlson2009}, and more fluid, project oriented teams.~\cite{Applegate1988a}

The third characteristic is the choice of forms of coordination in an enterprise. More centralized enterprises lean towards primarily vertical style of coordination~\cite{Bolman2008}, which is characterized by formal authority, standardization and rules in operations and in IT, and planning and control systems. [ELABORATE] Decentralized organizations lean towards more lateral styles of coordination. Lateral coordination is characterized by meetings, task forces, coordinating roles, matrix structures, and networks~~\cite{Bolman2008}[ELABORATE AND MAYBE NOT USE EXACTLY] It should be noted that enterprises will generally have a mix of lateral and vertical coordination, it is the tendency of an enterprise to focus on one more than the other that is an indicator for decentralization. 

\subsection{Challenges in Decentralized Organizations}

As decentralized organizations function in a significantly different manner than centralized organizations, they offer a different set of challenges that need to be faced: "~...~the novel and dynamic pressures that create the demand for decentralization in the first place can place organization leaders in considerably less certain, and consequently less commanding, positions."~\cite{caruso2008boundaries} In order to effectively meet these challenges, it is important to first understand what they are. [LIST WHAT I WILL TALK ABOUT]

A paper by Caruso, Rogers and Bazerman~\cite{caruso2008boundaries} highlights the importance of information sharing and coordination for these organizations. In order to succeed at these aspects, they outline three barriers that decentralized organizations need to overcome. The first barrier is intergroup bias; the tendency to treat one's own group better than other groups. The second barrier is group territoriality; the tendency for a group to protect their territory (physical or informational). The third barrier is poor negotiation strategies used by different groups when interacting with one another. 

Intergroup bias is direct result of having separate, autonomous groups within an enterprise~\cite{caruso2008boundaries}. The individual groups have a tendency to promote their own group over other groups, especially in situations where they are competing for a resource, such as a portion of the budget. A certain level of competition can be beneficial, however if it leads to hostility or distrust between groups, this can have a detrimental effect on their ability to share information and collaborate. This can prevent the groups from taking advantage of situations where they have to ability to work together for the benefit of everyone. 

The second barrier identified by Caruso et al. is group territoriality~\cite{caruso2008boundaries}. Group territoriality is characterized by group members taking action in order to protect their perceived territory. This can include physical territory such as space or tangible resources, as well as intangible territory, such as roles or information. Group territoriality is supported by a group's need to maintain its identity, its reputation of competence and sense of value, and a group's need for a stable home within the organization from which they interact with the rest of it.

Group territoriality encourages "a sense of psychological ownership"~\cite{caruso2008boundaries} for a group's members which can enforce the belief that they are the sole responsible party for a role or specific knowledge. This "inward-looking" behavior works against collaboration and information sharing. On the other hand, group territoriality can be beneficial; it can foster a sense of security in its members that "facilitates planning and execution of activities"~\cite{caruso2008boundaries}. 

The third barrier identified by Caruso et al. in decentralized organizations is related to negotiations between groups, and how these negotiations are often conducted using "poor negotiation  strategies"~\cite{caruso2008boundaries}. These poor strategies are the result of three common errors made while negotiating. The first error is a false belief in a "fixed pie" of value that is to be divided when negotiating. This prevents negotiating parties from recognizing situations where they are able to help each other, and therefore increase the size of the figurative pie. The second error is a failure to properly consider the other group's perspective. Understanding the other group's decision process, valuing process, and interests is key to discovering opportunities for helping one another, and the organization as a whole. The third error is when groups fail to even recognize they are in the process of negotiating. Instead, they see it as a competitive or hostile behavior where, again, they only see a fixed pie that is to be split up. This also prevents groups from taking advantage of opportunities to increase the size of the pie.    

% STRUCTURAL DILEMMAS from Reframing...

\subsection{Drivers for Decentralization}

Flexibility \& adaptability

Customization to local needs

Promote creativity

Globalization

Driver for inter-organizational networks: "fast-moving fields ... where knowledge is so complex and widely dispersed"~\cite{Bolman2008}

strengths of centralized i.e. hierarchies
define dynamic, uncertain, rapidly changing, stable... environments

"novel and dynamic pressures that create the demand for decentralization in the first place can place organization leaders in considerably less certain, and consequently less commanding, positions"

page 233 pearlson

\subsection{Case Study: An Institution of Higher Education in Sweden}

DSV pushes decision making power as close to the operational level as possible

Usage of meetings/seminars to bring everyone together and plan major change (Bologna)

Fluid, project-oriented teams; research projects

Informal communication lines \& ability to collaborate whenever they see fit (and is self-managed): DSV IT collaborates with other departments, collaboration for education programs

Task forces: reference groups of appropriate knowledgeable people for major (non-research) projects

Standardization exists, but is frequently not forced

Minimal formal metrics for performance measurement: where government mandated (re: education), but not much else.  