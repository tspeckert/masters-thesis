\subsection{What Makes an Organization Decentralized?}
Whether an enterprise is centralized or decentralized depends on more than simply its structure. Furthermore, enterprises will have elements of both centralization and decentralization in them, meaning that would be an oversimplification to classify an enterprise as just one or another. Consequently, organizational structure is best viewed as being on an organizational continuum, with decentralization on one end, centralization on the other end, and federalism in the middle~\cite{pearlson2009}. This section will describe a number of organizational characteristics that can be used to determine to what degree an organization is centralized or decentralized. 

The first characteristic are the allocation of decision rights in an enterprise, specifically who has the right to make what kinds of decisions in running and planning the enterprise.~\cite{pearlson2009} Decision rights are what controls the overall direction of an enterprise. They are needed in order to make change. In a completely centralized enterprise, all decision making authority would reside with a single, top-level authority. More decentralized enterprises would allow for participation in the decision making process by members throughout the enterprise. In a completely decentralized enterprise all members would have equal decision making rights. 

A second characteristic is the structure of communication lines in an enterprise. A centralized enterprise will have rigid hierarchies of communication, i.e. it is strictly defined who you work with and who you report to. Decentralized enterprises instead have less formalized communication lines~\cite{pearlson2009}, and more fluid, project oriented teams.~\cite{applegate1988}

The third characteristic is the choice of forms of coordination in an enterprise. More centralized enterprises lean towards primarily vertical style of coordination~\cite{Bolman2008}, which is characterized by formal authority, standardization and rules in operations and in IT, and planning and control systems. [ELABORATE] Decentralized organizations lean towards more lateral styles of coordination. Lateral coordination is characterized by meetings, task forces, coordinating roles, matrix structures, and networks~~\cite{Bolman2008}[ELABORATE] It should be noted that enterprises will generally have a mix of lateral and vertical coordination, it is the tendency of an enterprise to focus on one more than the other that is an indicator for decentralization.

\subsection{Drivers for Decentralization}

Flexibility \& adaptability

Customization to local needs

Promote creativity

Globalization

Driver for inter-organizational networks: "fast-moving fields ... where knowledge is so complex and widely dispersed"~\cite{Bolman2008}

\subsection{Case Study: An Institution of Higher Education in Sweden}

DSV pushes decision making power as close to the operational level as possible

Usage of meetings/seminars to bring everyone together and plan major change (Bologna)

Fluid, project-oriented teams; research projects

Informal communication lines \& ability to collaborate whenever they see fit (and is self-managed): DSV IT collaborates with other departments, collaboration for education programs

Task forces: reference groups of appropriate knowledgeable people for major (non-research) projects

Standardization exists, but is frequently not forced

Minimal formal metrics for performance measurement: where government mandated (re: education), but not much else.  