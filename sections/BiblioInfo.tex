% Filling in this bibliographic information facilitates the
% processing of this document.

% Insert linebreaks if necessary

% Just leave the posistion blank for any information that doesn't
% apply for your thesis, e.g. if your thesis doesn't have a subtitle:
%
% \newcommand{\thesisSubtitle}{}




% Abstract and titelpage

\firstAuthorFirstName{Thomas}                % First author given name
\firstAuthorSurname{Speckert}                 % First author surname
\secondAuthorFirstName{}              % Second author given name
\secondAuthorSurname{}                % Second author given name

\firstAuthorEmail{thsp7525@student.su.se}              % First author's e-mail address
\secondAuthorEmail{}             % Second author's e-mail address

\thesisTitle{Enterprise Architecture for Decentralized Environments}                   % The title of the thesis
\thesisSubtitle{} % The subtitle of the thesis 

\thesisSubject{Computer and Systems Sciences}   % May be changed
                                                % to e.g. Human-Computer 
                                                % Interaction or other 
                                                % suitable for your thesis 
\thesisIsKind{Master}                           % Change to Bachelor if suitable
\theYear{2013}
\thesisCred{30}                                 % Change to 15 if Bachelor
\thesisAdvisor{Jelena Zdravkovic}
\thesisAssistantAdvisor{}                       % Name of Assistant Advisor,
                                                % if you have one
\thesisExternalAdvisor{Irina Rychkova}                        % Name of External Advisor,
                                                % if you have one

\thesisReviewer{Janis Stirna}
\semester{Spring}
\swedishTitle{Företagsarkitektur för Decentraliserade Miljöer}

                                               % The abstract text comes
                                               % here. Not more than 300
                                               % words. No empty lines.                                     
\abstracttext{The problem of nusiness-IT alignment is an important issue for modern organizations. Solving it allows all components of an organization to operate together in a collaborative manner for the purpose of maximizing overall benefit to the enterprise. Enterprise Architecture (EA) is a discipline that aims to solve this problem in a holistic manner from the ground up through proper design. \\ \\
This thesis addresses the problem of a suitable EA for decentralized organizations and seeks to answer two research questions: 1) What aspects of existing EA frameworks are supportive of decentralized organizations? What aspects are not supportive? and 2) What are the principles of an EA framework that is supportive of decentralized organizations?. A design science research strategy has been followed in combination with  interviews and document studies from a case study for data generation, and a qualitative approach to data analysis. The results of this were: specific shortcomings of the existing EA frameworks of Zachman, FEA, TOGAF with respect to organizational decentralization, based on a literature review; the detailing of mismatches between the case organization's implicit EA and their organizational structure; the elicitation of requirements for an EA artifact addressing these shortcomings; the development of a generic EA artifact for IT governance based on the peer-to-peer principle of \textit{peer production}; and the demonstration of this artifact on the case organization. The demonstration showed that the proposed artifact offered advantages over traditional EA suggestions as it is able to better match the decentralized components of the case organization's organizational structure.}

\keywords{enterprise architecture, EA, TOGAF, FEA, Zachman, decentralized organizations, peer production}
%
%
%
%Requirement for 1 point: that the
%abstract of the thesis describes the
%problem, the research question, the
%choice and application of the
%research methods, the result, and
%conclusions and that it can be read
%and understood separately from the
%thesis.
%
%The abstract may not contain
%references to sections of the thesis,
%references, or not well-known
%abbreviations. The abstract shall be
%concise but detailed enough that a
%reader, after reading the abstract, can
%determine if the thesis is of interest.
%The abstract shall follow the Template
%for the Summary. A student may even
%include a complementary abstract
%structured in another way if desired.
%A thesis written en Swedish must
%even contain an English version of
%the abstract (entitled abstract).
%
%\abstracttext{The problem of business-IT alignment is of vital significance for modern enterprises. Solving it allows all components of an enterprise to operate together in a collaborative manner for the purpose of maximizing overall benefit to the enterprise. Enterprise Architecture (EA) is a discipline that aims to solve this problem in a holistic manner from the ground up through proper design. \\ \\
%The decentralization of organizations and the subsequent change to their management and operation style requires major changes in business processes and heavily involves the IT. This thesis project uses a combination of a literature study and a case study to demonstrate that EA is primarily suited to centralized organizational structures, and as such has some shortcomings when being applied to decentralized organizations. Overcoming these deficiencies requires some new principles to be introduced and incorporated into EA knowledge. A relevant source for these new principles are peer-to-peer architectures, which tackle their own version of the problem of decentralization. This thesis project presents prevalent decentralization principles and applies them to EA in order to make a set of recommendations for existing EA to improve their support of decentralization. These recommendations are shown to be viable through the use of a case study.}
