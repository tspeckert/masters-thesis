% Filling in this bibliographic information facilitates the
% processing of this document.

% Insert linebreaks if necessary

% Just leave the posistion blank for any information that doesn't
% apply for your thesis, e.g. if your thesis doesn't have a subtitle:
%
% \newcommand{\thesisSubtitle}{}




% Abstract and titelpage

\firstAuthorFirstName{Thomas}                % First author given name
\firstAuthorSurname{Speckert}                 % First author surname
\secondAuthorFirstName{}              % Second author given name
\secondAuthorSurname{}                % Second author given name

\firstAuthorEmail{thsp7525@student.su.se}              % First author's e-mail address
\secondAuthorEmail{}             % Second author's e-mail address

\thesisTitle{Enterprise Architecture for Decentralized Environments}                   % The title of the thesis
\thesisSubtitle{} % The subtitle of the thesis 

\thesisSubject{Computer and Systems Sciences}   % May be changed
                                                % to e.g. Human-Computer 
                                                % Interaction or other 
                                                % suitable for your thesis 
\thesisIsKind{Master}                           % Change to Bachelor if suitable
\theYear{2013}
\thesisCred{30}                                 % Change to 15 if Bachelor
\thesisAdvisor{Jelena Zdravkovic}
\thesisAssistantAdvisor{}                       % Name of Assistant Advisor,
                                                % if you have one
\thesisExternalAdvisor{Irina Rychkova}                        % Name of External Advisor,
                                                % if you have one

\thesisReviewer{Janis Stirna}
\semester{Spring}
\swedishTitle{Företagsarkitektur för Decentraliserade Miljöer}

                                               % The abstract text comes
                                               % here. Not more than 300
                                               % words. No empty lines.                                     
\abstracttext{The problem of business-IT alignment is an important issue for modern organizations. Solving it allows all components of an organization to operate together in a collaborative manner for the purpose of maximizing overall benefit to the enterprise. Enterprise Architecture (EA) is a discipline that aims to solve this problem in a holistic manner from the ground up through proper design. \\ \\
This thesis addresses the problem of a suitable EA for decentralized organizations and seeks to answer the following research question: Do existing EA frameworks need to be extended in order to support decentralized organizations?. A design science research strategy has been followed in combination with  interviews and document studies from a case study for data generation, and a qualitative approach to data analysis. The results of this were: specific shortcomings of the existing EA frameworks of Zachman, FEA, TOGAF with respect to organizational decentralization, based on a literature review; the detailing of mismatches between the case organization's implicit EA and their organizational structure; the elicitation of requirements for a EA artifact addressing these shortcomings; the development of a generic EA artifact for IT governance based on the peer-to-peer principle of \textit{peer production}; and the demonstration of this artifact on the case organization. An evaluation of the demonstration showed that the proposed artifact offered advantages over traditional EA suggestions due to its improved support of the decentralized properties present in the case organization's organizational structure.}

\keywords{enterprise architecture, EA, TOGAF, FEA, Zachman Framework, decentralized organizations, peer production}
