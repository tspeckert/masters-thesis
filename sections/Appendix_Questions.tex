\section*{Interviews with the Head of Graduate Education \& the Vice-Head of the Department}
%Organization
\begin{itemize}
\item Does the department have a set of operating principles?
\item According to the organizational chart, units, centres and some administrative staff have some sort of ``decision making'' influence over the Head of the Department. What exactly do these rights entail? 
\item What level of autonomy do the various units and centres have? 
	\begin{itemize}
		\item Do they run themselves, or are they managed by the head of the department?
		\item Do they manage their own IT systems, or do they exclusively utilize the department’s systems (or some combination thereof)?
		\item Are there any control or performance measures that the centres and units are measured by?
	\end{itemize}
\item How autonomous is the department within the institution as a whole?
\item Current processes at the department:
  \begin{itemize}
		\item How is project management performed?
		\item How are systems managed?
		\item How does the department manage its portfolios of applications and information?
		\item What general process is behind system design and development?
		\item Does the department keep an inventory of its skills and capabilities?
	\end{itemize}
\item Does the department monitor some form of ``return on investment''?
\item Does the department have any high-level guidelines or rules for its interactions with outside organizations?
  \begin{itemize} 
    \item Do the department’s IT systems offer support for these interactions?
  \end{itemize}
\item Are there any key legal issues (e.g. laws or directives from Swedish/EU government) that have a significant effect on the department’s operations?
\item Does the department use IT systems to support their managerial or governance activities?
%Reseach
\item What are the primary research-related operations at the department? 
	\begin{itemize}
	  \item How are they managed?
  	\item Who is responsible for their governance?
  \end{itemize}
\item Does the department have any control or performance measures for research activities?
\item How does the department manage collaborations with other organizations (e.g. with other universities or companies)?
\item What IT systems are used by the department to support their research activities?
\item Does the department follow a set of operating principles for education activities?
%Education
\item What are the primary education-related operations at the department?
  \begin{itemize}
	  \item How are they managed?
	  \item Who is responsible for their governance?
	\end{itemize}
\item Does the department have any control or performance measures for educational activities?
\item How does the department manage agreements/interactions with other organizations (for example, with respect to admissions, exchanges, or teaching in conjunction with other institutions)?
\item What IT systems are used by the department to support their education/teaching activities?
\item Does the department follow a set of operating principles for education activities?
\end{itemize}

\section*{Interview with the Head of IT}
\begin{itemize}
	\item What different systems are under your control?
	  \begin{itemize}
	   \item Are these all separate systems, or are some of them simply front-ends to the same system.
	  \end{itemize}
	\item What are the primary operations of the IT department?
	\item How do you ``evolve/change'' your IT systems (for example, domain migration)?
	\begin{itemize}
		\item Do the directives come from above?
		\item Are you able to do as you see necessary without explicit approval?
		\item How are the changes managed?
	\end{itemize}
	\item Do you cooperate with other the rest of the institution on other IT projects?
	\begin{itemize}
		\item If yes, how is this managed?
		\item Does this lead to conflicts?
		\item Do these cooperations have lead to improvements for everyone?
	\end{itemize}
	\item How do you measure/evaluate your performance?
	\item How is the Institution budget distributed?
\item Does this department compete with other departments in any way?
\item Is it important for this department to maintain an identity that is separate from the rest of the institution?
\end{itemize}

\section*{Interview with the Head of Undergraduate Education}
\begin{itemize}
\item How do you ``evolve/change'' your education programs?
  \begin{itemize}
	  \item Do the directives come from above?
	  \item Are you able to do as you see necessary without explicit approval?
	\end{itemize}
\item Do you cooperate with other departments for any degree programs?
  \begin{itemize}
	  \item If yes, how is this managed?
	\end{itemize}
\item Do you cooperate with other faculties for any degree programs?
  \begin{itemize}
	  \item If yes, how is this managed?
	\end{itemize}
\item How do you measure/evaluate your performance (informal and formal)?
\item Are the different education programs organized by the different units?
\item How is the Institution budget distributed?
\item Does this department compete with other departments in any way?
\item Is it important for this department to maintain an identity that is separate from the rest of the institution?
\end{itemize}
%
%
%Miscellaneous re: three barriers (inter-group bias, group territoriality, poor negotiations across the organization)
%\item How is the Institution budget distributed?
%\item Does this department compete with other departments in any way?
%\item Is it important for this department to maintain an identity that is separate from the rest of the institution?
%\item Is there a dedicated Doctoral School at SU/the Faculty of Social Sciences? 
