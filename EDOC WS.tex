
%% bare_conf.tex
%% V1.3
%% 2007/01/11
%% by Michael Shell
%% See:
%% http://www.michaelshell.org/
%% for current contact information.
%%
%% This is a skeleton file demonstrating the use of IEEEtran.cls
%% (requires IEEEtran.cls version 1.7 or later) with an IEEE conference paper.
%%
%% Support sites:
%% http://www.michaelshell.org/tex/ieeetran/
%% http://www.ctan.org/tex-archive/macros/latex/contrib/IEEEtran/
%% and
%% http://www.ieee.org/

%%*************************************************************************
%% Legal Notice:
%% This code is offered as-is without any warranty either expressed or
%% implied; without even the implied warranty of MERCHANTABILITY or
%% FITNESS FOR A PARTICULAR PURPOSE! 
%% User assumes all risk.
%% In no event shall IEEE or any contributor to this code be liable for
%% any damages or losses, including, but not limited to, incidental,
%% consequential, or any other damages, resulting from the use or misuse
%% of any information contained here.
%%
%% All comments are the opinions of their respective authors and are not
%% necessarily endorsed by the IEEE.
%%
%% This work is distributed under the LaTeX Project Public License (LPPL)
%% ( http://www.latex-project.org/ ) version 1.3, and may be freely used,
%% distributed and modified. A copy of the LPPL, version 1.3, is included
%% in the base LaTeX documentation of all distributions of LaTeX released
%% 2003/12/01 or later.
%% Retain all contribution notices and credits.
%% ** Modified files should be clearly indicated as such, including  **
%% ** renaming them and changing author support contact information. **
%%
%% File list of work: IEEEtran.cls, IEEEtran_HOWTO.pdf, bare_adv.tex,
%%                    bare_conf.tex, bare_jrnl.tex, bare_jrnl_compsoc.tex
%%*************************************************************************

% *** Authors should verify (and, if needed, correct) their LaTeX system  ***
% *** with the testflow diagnostic prior to trusting their LaTeX platform ***
% *** with production work. IEEE's font choices can trigger bugs that do  ***
% *** not appear when using other class files.                            ***
% The testflow support page is at:
% http://www.michaelshell.org/tex/testflow/



% Note that the a4paper option is mainly intended so that authors in
% countries using A4 can easily print to A4 and see how their papers will
% look in print - the typesetting of the document will not typically be
% affected with changes in paper size (but the bottom and side margins will).
% Use the testflow package mentioned above to verify correct handling of
% both paper sizes by the user's LaTeX system.
%
% Also note that the "draftcls" or "draftclsnofoot", not "draft", option
% should be used if it is desired that the figures are to be displayed in
% draft mode.
%
\documentclass[10pt, conference, compsocconf]{IEEEtran}
% Add the compsocconf option for Computer Society conferences.
%
% If IEEEtran.cls has not been installed into the LaTeX system files,
% manually specify the path to it like:
% \documentclass[conference]{../sty/IEEEtran}





% Some very useful LaTeX packages include:
% (uncomment the ones you want to load)


% *** MISC UTILITY PACKAGES ***
%
%\usepackage{ifpdf}
% Heiko Oberdiek's ifpdf.sty is very useful if you need conditional
% compilation based on whether the output is pdf or dvi.
% usage:
% \ifpdf
%   % pdf code
% \else
%   % dvi code
% \fi
% The latest version of ifpdf.sty can be obtained from:
% http://www.ctan.org/tex-archive/macros/latex/contrib/oberdiek/
% Also, note that IEEEtran.cls V1.7 and later provides a builtin
% \ifCLASSINFOpdf conditional that works the same way.
% When switching from latex to pdflatex and vice-versa, the compiler may
% have to be run twice to clear warning/error messages.






% *** CITATION PACKAGES ***
%
%\usepackage{cite}
% cite.sty was written by Donald Arseneau
% V1.6 and later of IEEEtran pre-defines the format of the cite.sty package
% \cite{} output to follow that of IEEE. Loading the cite package will
% result in citation numbers being automatically sorted and properly
% "compressed/ranged". e.g., [1], [9], [2], [7], [5], [6] without using
% cite.sty will become [1], [2], [5]--[7], [9] using cite.sty. cite.sty's
% \cite will automatically add leading space, if needed. Use cite.sty's
% noadjust option (cite.sty V3.8 and later) if you want to turn this off.
% cite.sty is already installed on most LaTeX systems. Be sure and use
% version 4.0 (2003-05-27) and later if using hyperref.sty. cite.sty does
% not currently provide for hyperlinked citations.
% The latest version can be obtained at:
% http://www.ctan.org/tex-archive/macros/latex/contrib/cite/
% The documentation is contained in the cite.sty file itself.






% *** GRAPHICS RELATED PACKAGES ***
%
\ifCLASSINFOpdf
  % \usepackage[pdftex]{graphicx}
  % declare the path(s) where your graphic files are
  % \graphicspath{{../pdf/}{../jpeg/}}
  % and their extensions so you won't have to specify these with
  % every instance of \includegraphics
  % \DeclareGraphicsExtensions{.pdf,.jpeg,.png}
\else
  % or other class option (dvipsone, dvipdf, if not using dvips). graphicx
  % will default to the driver specified in the system graphics.cfg if no
  % driver is specified.
  % \usepackage[dvips]{graphicx}
  % declare the path(s) where your graphic files are
  % \graphicspath{{../eps/}}
  % and their extensions so you won't have to specify these with
  % every instance of \includegraphics
  % \DeclareGraphicsExtensions{.eps}
\fi
% graphicx was written by David Carlisle and Sebastian Rahtz. It is
% required if you want graphics, photos, etc. graphicx.sty is already
% installed on most LaTeX systems. The latest version and documentation can
% be obtained at: 
% http://www.ctan.org/tex-archive/macros/latex/required/graphics/
% Another good source of documentation is "Using Imported Graphics in
% LaTeX2e" by Keith Reckdahl which can be found as epslatex.ps or
% epslatex.pdf at: http://www.ctan.org/tex-archive/info/
%
% latex, and pdflatex in dvi mode, support graphics in encapsulated
% postscript (.eps) format. pdflatex in pdf mode supports graphics
% in .pdf, .jpeg, .png and .mps (metapost) formats. Users should ensure
% that all non-photo figures use a vector format (.eps, .pdf, .mps) and
% not a bitmapped formats (.jpeg, .png). IEEE frowns on bitmapped formats
% which can result in "jaggedy"/blurry rendering of lines and letters as
% well as large increases in file sizes.
%
% You can find documentation about the pdfTeX application at:
% http://www.tug.org/applications/pdftex





% *** MATH PACKAGES ***
%
%\usepackage[cmex10]{amsmath}
% A popular package from the American Mathematical Society that provides
% many useful and powerful commands for dealing with mathematics. If using
% it, be sure to load this package with the cmex10 option to ensure that
% only type 1 fonts will utilized at all point sizes. Without this option,
% it is possible that some math symbols, particularly those within
% footnotes, will be rendered in bitmap form which will result in a
% document that can not be IEEE Xplore compliant!
%
% Also, note that the amsmath package sets \interdisplaylinepenalty to 10000
% thus preventing page breaks from occurring within multiline equations. Use:
%\interdisplaylinepenalty=2500
% after loading amsmath to restore such page breaks as IEEEtran.cls normally
% does. amsmath.sty is already installed on most LaTeX systems. The latest
% version and documentation can be obtained at:
% http://www.ctan.org/tex-archive/macros/latex/required/amslatex/math/





% *** SPECIALIZED LIST PACKAGES ***
%
%\usepackage{algorithmic}
% algorithmic.sty was written by Peter Williams and Rogerio Brito.
% This package provides an algorithmic environment fo describing algorithms.
% You can use the algorithmic environment in-text or within a figure
% environment to provide for a floating algorithm. Do NOT use the algorithm
% floating environment provided by algorithm.sty (by the same authors) or
% algorithm2e.sty (by Christophe Fiorio) as IEEE does not use dedicated
% algorithm float types and packages that provide these will not provide
% correct IEEE style captions. The latest version and documentation of
% algorithmic.sty can be obtained at:
% http://www.ctan.org/tex-archive/macros/latex/contrib/algorithms/
% There is also a support site at:
% http://algorithms.berlios.de/index.html
% Also of interest may be the (relatively newer and more customizable)
% algorithmicx.sty package by Szasz Janos:
% http://www.ctan.org/tex-archive/macros/latex/contrib/algorithmicx/




% *** ALIGNMENT PACKAGES ***
%
%\usepackage{array}
% Frank Mittelbach's and David Carlisle's array.sty patches and improves
% the standard LaTeX2e array and tabular environments to provide better
% appearance and additional user controls. As the default LaTeX2e table
% generation code is lacking to the point of almost being broken with
% respect to the quality of the end results, all users are strongly
% advised to use an enhanced (at the very least that provided by array.sty)
% set of table tools. array.sty is already installed on most systems. The
% latest version and documentation can be obtained at:
% http://www.ctan.org/tex-archive/macros/latex/required/tools/


%\usepackage{mdwmath}
%\usepackage{mdwtab}
% Also highly recommended is Mark Wooding's extremely powerful MDW tools,
% especially mdwmath.sty and mdwtab.sty which are used to format equations
% and tables, respectively. The MDWtools set is already installed on most
% LaTeX systems. The lastest version and documentation is available at:
% http://www.ctan.org/tex-archive/macros/latex/contrib/mdwtools/


% IEEEtran contains the IEEEeqnarray family of commands that can be used to
% generate multiline equations as well as matrices, tables, etc., of high
% quality.


%\usepackage{eqparbox}
% Also of notable interest is Scott Pakin's eqparbox package for creating
% (automatically sized) equal width boxes - aka "natural width parboxes".
% Available at:
% http://www.ctan.org/tex-archive/macros/latex/contrib/eqparbox/





% *** SUBFIGURE PACKAGES ***
%\usepackage[tight,footnotesize]{subfigure}
% subfigure.sty was written by Steven Douglas Cochran. This package makes it
% easy to put subfigures in your figures. e.g., "Figure 1a and 1b". For IEEE
% work, it is a good idea to load it with the tight package option to reduce
% the amount of white space around the subfigures. subfigure.sty is already
% installed on most LaTeX systems. The latest version and documentation can
% be obtained at:
% http://www.ctan.org/tex-archive/obsolete/macros/latex/contrib/subfigure/
% subfigure.sty has been superceeded by subfig.sty.



%\usepackage[caption=false]{caption}
%\usepackage[font=footnotesize]{subfig}
% subfig.sty, also written by Steven Douglas Cochran, is the modern
% replacement for subfigure.sty. However, subfig.sty requires and
% automatically loads Axel Sommerfeldt's caption.sty which will override
% IEEEtran.cls handling of captions and this will result in nonIEEE style
% figure/table captions. To prevent this problem, be sure and preload
% caption.sty with its "caption=false" package option. This is will preserve
% IEEEtran.cls handing of captions. Version 1.3 (2005/06/28) and later 
% (recommended due to many improvements over 1.2) of subfig.sty supports
% the caption=false option directly:
%\usepackage[caption=false,font=footnotesize]{subfig}
%
% The latest version and documentation can be obtained at:
% http://www.ctan.org/tex-archive/macros/latex/contrib/subfig/
% The latest version and documentation of caption.sty can be obtained at:
% http://www.ctan.org/tex-archive/macros/latex/contrib/caption/




% *** FLOAT PACKAGES ***
%
%\usepackage{fixltx2e}
% fixltx2e, the successor to the earlier fix2col.sty, was written by
% Frank Mittelbach and David Carlisle. This package corrects a few problems
% in the LaTeX2e kernel, the most notable of which is that in current
% LaTeX2e releases, the ordering of single and double column floats is not
% guaranteed to be preserved. Thus, an unpatched LaTeX2e can allow a
% single column figure to be placed prior to an earlier double column
% figure. The latest version and documentation can be found at:
% http://www.ctan.org/tex-archive/macros/latex/base/



%\usepackage{stfloats}
% stfloats.sty was written by Sigitas Tolusis. This package gives LaTeX2e
% the ability to do double column floats at the bottom of the page as well
% as the top. (e.g., "\begin{figure*}[!b]" is not normally possible in
% LaTeX2e). It also provides a command:
%\fnbelowfloat
% to enable the placement of footnotes below bottom floats (the standard
% LaTeX2e kernel puts them above bottom floats). This is an invasive package
% which rewrites many portions of the LaTeX2e float routines. It may not work
% with other packages that modify the LaTeX2e float routines. The latest
% version and documentation can be obtained at:
% http://www.ctan.org/tex-archive/macros/latex/contrib/sttools/
% Documentation is contained in the stfloats.sty comments as well as in the
% presfull.pdf file. Do not use the stfloats baselinefloat ability as IEEE
% does not allow \baselineskip to stretch. Authors submitting work to the
% IEEE should note that IEEE rarely uses double column equations and
% that authors should try to avoid such use. Do not be tempted to use the
% cuted.sty or midfloat.sty packages (also by Sigitas Tolusis) as IEEE does
% not format its papers in such ways.





% *** PDF, URL AND HYPERLINK PACKAGES ***
%
%\usepackage{url}
% url.sty was written by Donald Arseneau. It provides better support for
% handling and breaking URLs. url.sty is already installed on most LaTeX
% systems. The latest version can be obtained at:
% http://www.ctan.org/tex-archive/macros/latex/contrib/misc/
% Read the url.sty source comments for usage information. Basically,
% \url{my_url_here}.





% *** Do not adjust lengths that control margins, column widths, etc. ***
% *** Do not use packages that alter fonts (such as pslatex).         ***
% There should be no need to do such things with IEEEtran.cls V1.6 and later.
% (Unless specifically asked to do so by the journal or conference you plan
% to submit to, of course. )

% correct bad hyphenation here
\hyphenation{under-specified}


\begin{document}
%
% paper title
% can use linebreaks \\ within to get better formatting as desired
\title{On the Changing Role of Enterprise Architecture in Decentralized Environment: State of the Art}


% author names and affiliations
% use a multiple column layout for up to three different
% affiliations
\author{\IEEEauthorblockN{Thomas Speckert}
\IEEEauthorblockA{DSV - SU\\
Stockholm, Sweden\\
Email: xxxxx}
\and
\IEEEauthorblockN{Irina Rychkova}
\IEEEauthorblockA{CRI\\
Universit\'{e} Paris 1 Panth\'{e}on - Sorbonne,\\ 90 rue Tolbiac, 75013 Paris, France\\
Email: irina.rychkova@univ-paris1.fr}
\and
\IEEEauthorblockN{Jelena Zdravkovic}
\IEEEauthorblockA{DSV - SU\\
Stockholm, Sweden\\
Email: xxxxx}
 }



% conference papers do not typically use \thanks and this command
% is locked out in conference mode. If really needed, such as for
% the acknowledgment of grants, issue a \IEEEoverridecommandlockouts
% after \documentclass

% for over three affiliations, or if they all won't fit within the width
% of the page, use this alternative format:
% 
%\author{\IEEEauthorblockN{Michael Shell\IEEEauthorrefmark{1},
%Homer Simpson\IEEEauthorrefmark{2},
%James Kirk\IEEEauthorrefmark{3}, 
%Montgomery Scott\IEEEauthorrefmark{3} and
%Eldon Tyrell\IEEEauthorrefmark{4}}
%\IEEEauthorblockA{\IEEEauthorrefmark{1}School of Electrical and Computer Engineering\\
%Georgia Institute of Technology,
%Atlanta, Georgia 30332--0250\\ Email: see http://www.michaelshell.org/contact.html}
%\IEEEauthorblockA{\IEEEauthorrefmark{2}Twentieth Century Fox, Springfield, USA\\
%Email: homer@thesimpsons.com}
%\IEEEauthorblockA{\IEEEauthorrefmark{3}Starfleet Academy, San Francisco, California 96678-2391\\
%Telephone: (800) 555--1212, Fax: (888) 555--1212}
%\IEEEauthorblockA{\IEEEauthorrefmark{4}Tyrell Inc., 123 Replicant Street, Los Angeles, California 90210--4321}}




% use for special paper notices
%\IEEEspecialpapernotice{(Invited Paper)}




% make the title area
\maketitle


\begin{abstract}
%\boldmath
The problem of business-IT alignment is an important one for all enterprises. Solving it allows all components of an enterprise to operate together in a collaborative manner for the purpose of maximizing overall benefit to the enterprise. Enterprise Architecture (EA) is a discipline that aims to solve this problem in a holistic manner from the ground up through proper design. This paper demonstrates, however, that EA is primarily aimed at centralized organizational structures, and has many shortcomings when being applied to decentralized organizations. Overcoming these shortcomings requires some new principles need to be introduced that can be incorporated into existing EA knowledge [Or into a new type of EA? Maybe better]. Potential sources for these new principles are existing decentralized organizations, peer-to-peer technologies, and organizational science. All of these areas have tackled different "flavors" of the problem of decentralization in some way, and this paper presents some of their principles that also have the potential to be applied to EA.
\end{abstract}

% IEEEtran.cls defaults to using nonbold math in the Abstract.
% This preserves the distinction between vectors and scalars. However,
% if the conference you are submitting to favors bold math in the abstract,
% then you can use LaTeX's standard command \boldmath at the very start
% of the abstract to achieve this. Many IEEE journals/conferences frown on
% math in the abstract anyway.

% no keywords
\begin{IEEEkeywords}
Enterprise Architecture, Decentralization, Peer-to-Peer
\end{IEEEkeywords}



% For peer review papers, you can put extra information on the cover
% page as needed:
% \ifCLASSOPTIONpeerreview
% \begin{center} \bfseries EDICS Category: 3-BBND \end{center}
% \fi
%
% For peerreview papers, this IEEEtran command inserts a page break and
% creates the second title. It will be ignored for other modes.
\IEEEpeerreviewmaketitle

\section{Introduction}
\label{intro}
Organizations with rigid centralized management style (linear or pyramidal hierarchies) fail to sustain the dynamic environments due to their inertia in decision making and lack of agility. Political, social and economic systems progressively transform to distributed networks (Michael Bowens, http://www.ctheory.net/articles.aspx?id=499) and novel organization forms are emerging. 
The term "liquid enterprise" coined recently, describes the nature of such organisations. Transparent or dynamically changing boundaries, agile processes\cite{agileproc}, virtual collaborations, coopetition \cite{coopetition} – all of those are technology-enabled capabilities of an organization of the future. 

Decentralization of organizations and subsequent change of their management and operation style requires major changes in organization processes and heavily involves  the IT. 

Ross et. al. \cite{ross2006} define a term foundation for execution to address \textit{\"the IT infrastructure and digitized business processes automating a company’s core capabilities\"}. While emerging technologies serve the main catalyst for organizational transformations, embracing the “right” technologies and evolving the existing foundation for execution accordingly - is primordial for organizations.  IF WE KEEP THIS TERM - WE NED TO USE  "FOUNDATION FOR EXECUTION" CONSISTENTLY IN THE REST OF THE PAPER. OTHERWISE THIS PARAGRAPH CAN BE DELETED

Traditionally, such questions are addressed by the enterprise architecture (EA) discipline. 
EA \textit{"defines the underlying principles, standards, and best practices according to which current and future activities of the enterprise should be conducted"} \cite{schekkerman2003}. EA methods and tools serve:
\begin{itemize}
\item to specify the current state of the company's foundation for execution (FFE)  - \textit{architecture as is};
\item to identify the target architecture – \textit{architecture to-be};
\item to analyze the gap and set up a master plan for achieving this target – \textit{architectural principles, architecture roadmap}.
\end{itemize}
These artifacts are addressed as EA description; the process that organization has to execute in order to obtain its EA description is called EA method. Traditional EA project, though, consists in implementing the EA method and producing the EA description. To assure that the organization will continuously follow the principles and achieve the designated goals after the termination of EA project – the third element has to be defined. We call this element EA engine, referring to its capacity to stir the company. \footnote{In \cite{ross2006}, this element is addressed as “engagement model”.}

In \cite{sachdeva1990}, organizational structure is defined as "... institutional arrangements and mechanisms for mobilizing human, physical, financial and information resources at all levels of the system..." Numerous taxonomies of organizational types are defined in the literature


Many modern organizations are trending towards decentralization; where organizational units and individuals are being given an increasing amount of autonomy and control over an enterprise's direction and operations. In order to classify these enterprises, this paper presents a taxonomy for categorizing these organizations on a spectrum ranging from centralized to decentralized. 

Decentralization transforms the role of company's authority and makes relationships inside and between different company's divisions much more complex. Planning and governance in different functional areas, including IT,  is not anymore ensured centrally. As a consequence, more efforts are required to prioritize initiatives, coordinate and communicate decisions, manage projects, and evaluate results. Moreover, the practices of management, coordination, communication and decision making are not the same as before: collaboration and information sharing gain extreme importance.

The de-facto EA methodologies rely on organizational properties such as centralized management, global company identity, etc.  that are getting obsolete with progressive decentralization.  Consequently, implementation of these methodologies in decentralized organization becomes difficult and inefficient and  the role of EA as a driver for IT transformations is getting compromised.

A company Y acquired a software system  with an objective of integrated support of XXX across divisions that costed them  XXX \$. Divisions  were not (only partially?) involved into decision making process and product evaluation (decision was made centrally) and eventually refused to shut down their local systems and switch to the global one (decentralized IT management). As a consequence, strategic initiative for integration failed; divisions managed to protect their interests (preserving local systems that are tailored for their needs) however  got charged for the acquired system they never used (centralized budgeting). This example demonstrates a "good" decision made by "wrong" people... Is it a sign of an inappropriate EA? [NEED TO WORK ON IT..]

%In order for these EA frameworks to continue to be suitable for the modern enterprise, some changes need to be made to take into account these new enterprise structures. 


%They are taking for granted some organization and ICT properties (e.g. culture) that do not exist any more. As a result, the attempts to implement the EA methods or to follow the traditional principles meet the hostility in the organization (parts) and often fail.

Therefore,  novel EA processes, principles and concepts are needed to both handle the ICT resources and to foster business/ICT co-evolution in decentralized environment. This paper presents the literature review that supports our claim. 

The main contribution of this paper is ........


The reminder of this article is organized as follows: In Section II, we discuss the role of EA in organization and provides an overview of three EA methodologies: TOGAF, Zachman and FEA;  in Section III, we discuss different forms of organizational structure and the trend towards decentralization. We conclude this section by highlighting the 4 (?) challenges related to  decentralization in IT. In Section IV we analyze the EA methodologies presented in Section II focusing on their support to decentralization. We identify (5?? - a concrete number may sound good..) In Section V, we propose ....... 

% 
% Related EDOC aspects
%     Enterprise architecture frameworks 
%     (Collaborative development and cooperative engineering issues)
%     Enterprise interoperability, collaboration and its architecture
%     ([Cross-enterprise collaboration] in a world of cloud, social and big data)
% 
% 
% 
% 
% 
% 
% 
% 
% 
% 
% 
% 
% 
% 
% 
% 

\section{Enterprise Architecture}
\label{EA}
\subsection{A Common Perspective on Enterprise Architecture}
While there is no singular agreed-upon definition for EA, different definitions\cite{jungle2004,GartnerInc,ross2006,pearlson2009,lankhorst2009,sessions2007,togaf9.1} do have much in common. EA is a discipline that takes a holistic, design-oriented approach to transforming high-level business vision and goals into the integration of an enterprise's organizational structure, business processes, and information systems. This transformation involves identifying and implementing the necessary change for this to occur. In order to view different Enterprise Architectures from a common perspective, this paper will break the frameworks down into three separate components: the EA method, the EA description, and the EA engine. 

The Method aims to lay the groundwork for the EA project. Typically, this involves setting up teams, ownership, responsibilities and gaining commitment. Also it defines the overall process of collecting, validating and approving the EA artifacts  (eg. descriptions As-Is, To-Be, gap analysis,  principles) that will form the second component - The EA description.  The Engine involves setting up a support structure for ensuring the ongoing adoption of the to-be EA description. This can involve gaining commitment from stakeholders, setting up some compliance checking procedures, and deciding upon a prioritization of tasks to be completed. The remainder of this section will look at three different EA frameworks from the perspective of of these three phases: The Open Group Architecture Framework (TOGAF), the Zachman Framework, and the Federal Enterprise Architecture (FEA)

\subsection{TOGAF}
The Open Group Architecture Framework, more commonly know as TOGAF, is a freely available EA framework created by The Open Group~\cite{togaf9.1}, a consortium of IT organizations. TOGAF is comprised of a number of different aspects, mainly: the Architecture Development Method (ADM), "a method for developing and managing the lifecycle of an enterprise architecture"~cite{togaf9.1}; the Architecture Content Framework, a companion to the ADM which describes the content of the products of the ADM; and the Enterprise Continuum, which provides a means to organize the produced architectures. 
\subsubsection{EA Method}
The TOGAF ADM falls under our EA Method component of EA. The TOGAF ADM is made up of a preliminary phase, six core phases (labeled A-H), and a requirements management component. 

[XXXX TOGAF ADM Diagram XXXX]

In TOGAF, the preliminary phase lays the groundwork for the rest of the EA process. Some important aspects are to set up a governance structure and EA team for the EA process and to establish a repository for storing all architectural information.~\cite{togaf9.1}

Phase A of the TOGAF Process, the architectural vision phase, is aimed at setting a clear vision for the enterprises future architecture. This involves creating the initial as-is architecture as well as setting clear, management approved goals and requirements, and transforming them into a high-level vision of the enterprises to-be architecture~\cite{togaf9.1}.

At this point, TOGAF suggests that the outputs of the preliminary phase and phase A be organized into a "Statement of Architecture Work". This document is to be approved by project sponsors and can be used to form the basis of a contract between the architecture provider and the client~\cite{togaf9.1}.

The next three phases, B-D are concerned with creating the as-is and to-be business architecture, information systems architecture, and the technology architecture. TOGAF suggests two different approaches to creating the architectures: baseline or target-first~\cite{togaf9.1}. Baseline-first involves analyzing the as-is architecture for areas where improvements can be made. Target-first aims at creating a detailed target architecture and then mapping it back to the as-is architecture in order figure out what needs to change. The main aspects of these phases are to develop the as-is and to-be architectures, analyze the gap between them, and create an initial road-map of the steps needed to cross the gap.

Phase E and F, Opportunities and Solutions and Migration Planning, are concerned with organizing the work to be done into projects, and then creating a schedule for executing the projects. 

The final ADM phase, phase G, is concerned with the implementation and setting up a framework for its governance and its compliance to the target architecture~\cite{togaf9.1}.

\subsubsection{EA Description}
TOGAF views architecture from the perspective of four different architecture domains~\cite{sessions2007}: business, application, data, and technical. Business architecture is concerned with processes and functions used to meet business goals, application architecture is concerned with the design of specific applications and their interactions, data architecture is concerned with managing enterprise data, and the technical architecture is concerned with the infrastructure (hardware and software) used to support the applications. The architectures in these four domains are created through the ADM phases B (Business Architecture Phase), C (Information Systems Architectures Phase) and D (Technology Architecture).

The various architectural artifacts in TOGAF are organized across an Architectural Landscape~\cite{togaf9.1} of three dimensions: breadth, level, and time. Breadth refers to the area of subject matter for an architecture. Levels refer to the level of detail of an architecture. TOGAF specifies three levels of detail: strategic, for overall direction setting at the executive level; segment, for architectures at the level of a program or portfolio; and capability, for architectures concerned with how the architecture process is itself enabled and governed. The time dimension of the landscape keeps the state of architectures as they evolve over time. Additionally, the Architecture Landscape can be partitioned into independent partitions for supporting different organizational units. 

%Breadth: The breadth (subject matter) area is generally the primar y organizing character istic for descr ibing an Architecture Landscape. Architectures are functionally decomposed into a hierarchy of specific subject areas or segments.

%Capability Architecture provides an organizing framework for change activity and the development of effective architecture roadmaps realizing capability increments.


At each level of the Architecture Landscape, architectures are further organized through the Enterprise Continuum which provides a way to organize the architectures from generic to organization-specific~\cite{togaf9.1}. The most generic are called Foundation Architectures, which are applicable to all enterprises. A core aspect of a Foundation Architecture is to provide a high-level taxonomy which can provide a basis for the more specific architectures.~\cite{togaf9.1} TOGAF includes a Foundation Architecture which can be used, called the Technical Reference Model(TRM). The second set of architectures in the continuum are called the Common Systems Architectures. These architectures are specific to a generic problem domain (e.g. security management), and are thus applicable to a wide range (but not all) of enterprises. TOGAF includes a Common System Architecture for the domain of information integration, called the Integrated Information Infrastructure Reference Model (III-RM). The third set of architectures in the continuum are called Industry Architectures. These architectures are applicable to a specific problem within a specific industry. They are thus useful to many members of that industry, but not necessarily outside of it. The most specific level in the continuum are Organization-Specific  architectures. As the name implies, they are relevant only to a specific enterprise. These outline the architectural solution for a particular enterprise and provide "a means to communicate and manage business operations across all four architectural domains"~\cite{togaf9.1}.

%The content of TOGAF architecture artifacts is defined in the Content Framework... [Still don't really understand this part] [maybe content is actually determined through reference models which are placed on the continuum?]

% architecture landscape
% architecture partitioning
% architecture capability
% interop of partitions "federated architectures � independently developed, maintained, and managed architectures that are subsequently integrated within an integration framework � are typical



\subsubsection{EA Engine}

TOGAF outlines an ADM phase concerned with the ongoing change management process for the architecture of an enterprise.  It is concerned with managing changes to the architecture throughout its lifecycle~\cite{togaf9.1}. In this phase, a governance body sets criteria for determining if a change requires an architecture update if a new cycle of the ADM needs to be started. An important aspect of this process is to deploy tools for monitoring for business and technological changes and measuring performance indicators. 

TOGAF describes a formal review process for determining compliance. The main goal of this process is to "[f]irst and foremost, catch errors in the project architecture early, and thereby reduce the cost and risk of changes required later in the lifecycle"~\cite{togaf9.1}.

[Process Diagram from Figure 48-2 Architecture Compliance Review Process p. 565 of~\cite{togaf9.1}]

%Some other goals of this process are to "Ensure the application of best practices to architecture work" "Provide an overview of the compliance of an architecture to mandated enterpr ise standards." "Identify where the standards themselves may require modification." "Communicate to management the status of technical readiness of the project." "Identify and communicate significant architectural gaps to product and service providers"

TOGAF outlines a formal approach to architecture governance which involves the setting up of an "Architecture Board"~\cite{togaf9.1}. "to oversee the implementation of the [architecture] strategy"~\cite{togaf9.1}. This board has an important role in Architecture Governance, such as "[p]roviding the basis for all decision-making with regard to the architectures"~\cite{togaf9.1} and enforcing architecture compliance. The TOGAF Architecture Governance Framework suggests guidelines for developing a formal governance structure for the Enterprise Continuum (and thus, all the architectural artifacts) and architecture processes. 


%governance p 590

%[Repository]
%
%[requirements management]

\subsection{Zachman}
The Zachman Framework was the first EA, first introduced by John Zachman in 1987~\cite{sessions2007,zachman}. It consists only of a taxonomy, and as such only fits into the EA Description aspect of EA. 

\subsubsection{EA Description}



The Zachman Framework breaks down EA into a grid of perspectives. Each perspective is characterized by two things; its target audience and the issue is aimed at. ZF covers six issues: What (data and entities), How (functional), Where (locations and interconnections/networks), Who (people relationships), When (events and performance criteria), Why (motivations and goals)~\cite{jungle2004}. For each issue, it views it from five different perspectives: executive, business management, architect, engineer, and technician. 

The executive perspective is meant for executives or planners and needs to provide an estimate of a system's functionality and cost~\cite{jungle2004}. The business management perspective is a business view of how an owner thinks the business operates~\cite{Zachman2000}. The architect perspective takes a systems viewpoint and describes the operations and interactions of the variety of systems in an enterprise. The engineer perspective views describes the physical technology and design of the individual systems. The technician perspective takes the perspective of a "sub-contractor" who is implementing a specific system and the high, out-of-context level of detail associated with that.
 
[XXXX Zachman Diagram XXXX]

[expand on each perspective, table?]

\subsection{FEA}
The Federal Enterprise Architecture (FEA)is an effort by the federal government of the United States to create an EA for the entire government. The FEA is a complete EA framework, covering all three components of EA. The Federal Enterprise Architecture Program Management Office describes FEA as "...a common language and framework to describe and analyze IT investments, enhance collaboration and ultimately transform the Federal government into a citizen-centered, results-oriented, and market-based organization as set forth in the President's Management Agenda."~\cite{FederalEnterpriseArchitectureProgramManagementOffice} FEA takes takes an approach where individual organizational units develop their own architectures that fit into an overall framework of common standards and interoperability.

FEA is composed of six core elements~\cite{sessions2007}:
\begin{itemize}
    \item The enterprise is broken-down into different segments of varying scopes, and architecture is developed for each segment
    \item A set of five reference models which are used as a basis to describe the important elements of the FEA in a consistent manner
    \item A process for creating each segment EA
    \item A transitional process for moving from the current state of the enterprise to the visioned state
    \item A taxonomy for organizing the various assets of the FEA
    \item Guidelines for measuring the degree of success of the FEA
\end{itemize}

\subsubsection{EA Method}

% Performance Improvement Lifecyle seems to be the overall FEA process

FEA develops architecture for segments and enterprise services. A segment is a "major line-of-business functionality"~\cite{sessions2007} for an individual organizational unit (such as an agency or department). Two types of segments exist, core mission-area segments and business service segments~\cite{FederalEnterpriseArchitectureProgramManagementOffice2007}. Core mission-area segments are at the scope of a single organizational unit (though they may be shared by different units) and are essential to its purpose~\cite{sessions2007,FederalEnterpriseArchitectureProgramManagementOffice2007}. Business service segments are also at the scope of an individual organizational unit, however these segments exist in all organizational units and are defined for the entire enterprise. Like business service segments, enterprise services are defined organization-wide. However, they are different in that they also function at the enterprise level, e.g. a single security management service that is shared by the entire enterprise. 

%"single agency contains both core mission area segments and business service segments. Enterprise services are those cross-cutting services spanning multiple segments."\cite{FederalEnterpriseArchitectureProgramManagementOffice2007}
%"By contrast, segment architecture defines a simple roadmap for a core mission area, business service or enterprise service"~\cite{FederalEnterpriseArchitectureProgramManagementOffice2007}
%
% ~\cite{sessions2007}
%
%Segments 
% - segment is a major line-of-business functionality, e.g. HR
%     - core mission-area segments, central to mission/purpose of organization
%     - business-services segment, foundational to all organizations
% - function at agency level, defined at enterprise level

% Just as enterprises are themselves hierarchically organized, so are the different views provided by each type of architecture.
 
FEA defines a four step iterative process for creating architectures for each segment and service~\cite{FederalEnterpriseArchitectureProgramManagementOffice2007}:
\begin{enumerate}
    \item Architectural analysis
    \item Architectural definition
    \item Investment and funding strategy
    \item Program management plan and execute projects
\end{enumerate}

%New and revised business and information requirements are continuously identified as the segment moves though each lifecycle phase, and as business and information management solutions are funded and developed to meet stakeholder requirements. Consequently, segment architecture work products must be maintained to reflect these inputs.

The first step, architectural analysis, is concerned with defining the scope of the segment, its baseline architecture, current problems in the segment, and a high-level vision of the desired final state for the segment~\cite{FederalEnterpriseArchitectureProgramManagementOffice2007}.

The second step, architectural definition, is concerned with defining the detailed target architecture of the segment~\cite{FederalEnterpriseArchitectureProgramManagementOffice2007}. Aside for the architecture itself, it is also necessary to define a roadmap of projects to get there, the segment transition strategy, and the performance goals of the architecture. 
  
The third step, the investment and funding strategy, is concerned with specifying how the projects identified in the segment transition strategy are to be funded~\cite{FederalEnterpriseArchitectureProgramManagementOffice2007}. 

The fourth step, program management plan and execute strategies, is concerned with making detailed plans for the individual projects, executing the plans, and defining performance measurements for the initiative~\cite{FederalEnterpriseArchitectureProgramManagementOffice2007}.

[Insert Figure 3-2: Segment Architecture Development and Maintenance from ~\cite{FederalEnterpriseArchitectureProgramManagementOffice2007}]

% is there an EA agency???
    
\subsubsection{EA Definition}
% ~\cite{sessions2007}

%Segment identification is a continuous, iterative process. Enterprise assets including systems and IT investments are mapped to the agency-level reference models to create a segment-oriented view of the enterprise (see Figure 2-2). 
%
%"The FEA consists of a set of interrelated "reference models" designed to facilitate cross-agency analysis and the identification of duplicative investments, gaps, and opportunities for collaboration within and across agencies. Collectively, the reference models [compose] a framework for describing important elements of the FEA in a common and consistent way."~\cite{FederalEnterpriseArchitectureProgramManagementOffice}
%
%"This, in a nutshell, is the goal of the five FEA reference models: to give standard terms and definitions for the domains of enterprise architecture and, thereby, facilitate collaboration and sharing across the federal government."


In order to have a common language for describing the enterprises assets, FEA describes five reference models for mapping assets to segments and enterprise services~\cite{FederalEnterpriseArchitectureProgramManagementOffice}. The five reference models are the performance reference model, the business reference model, the service component reference model, the technical reference model, and the data reference model. 

[Insert Segment Identification Figure (2.2 on page 20 of ~\cite{FederalEnterpriseArchitectureProgramManagementOffice})]

The performance reference model provides a framework for developing consistent measurement. The business reference model provides a framework for developing a functional view of the enterprises line of business. The service component reference model provides a framework for describing how the services offered by IT systems support business functionality.  The data reference model provides a framework for describing data in a consistent way that enables enterprise-wide sharing. 

\subsubsection{EA Engine}
%~\cite{sessions2007}

FEA describes an "engine" to maintain the architecture in order ensure that it stays relevant over time. FEA calls this engine an activity it calls "segment architecture maintenance"~\cite{FederalEnterpriseArchitectureProgramManagementOffice}. In this activity, it is important to monitor for, list and prioritize new architectural change drivers as they appear. The impact of these drivers needs to be defined. 

% term is "architectural change drivers.", maybe define

% Performance Improvement Lifecycle

In addition to the segment architecture maintenance activity, the Office of Management and Budget (OMB) describes an "Enterprise Architecture Assessment Framework" for the continuous assessment of each agencies performance in their EA practice~\cite{OfficeofManagementandBudget}. This framework assesses the maturity of the enterprises adoption of EA in three dimensions of KPIs: completion, use, and results. The "completion" dimension aims to measure the completeness of an agencies target EA and transition plan, i.e. how well it is "positioned to serve  as the agency's blueprint that describes its future state from a performance, business, service, data, and technology standpoint"~\cite{OfficeofManagementandBudget}. The "use" dimension measures how well the architecture is used to "drive decision making"~\cite{sessions2007}. The results dimension measures the direct benefits of using the architecture~\cite{sessions2007}, such as measurable improvements in the performance of programs or the direct benefits to decision makers~\cite{OfficeofManagementandBudget}. 

[Insert Figure 2-1: Information and IT-Enabled Performance Improvement Lifecycle from~\cite{OfficeofManagementandBudget}]



\section{Organizational Structure and its Relation to Decentralization}
\label{organizations}
\subsection{Overview}
There has been a lot of research on specific forms of organizational structure. Consequently, forms such as hierarchical, flat, matrix, networked and adhocracy have been very well defined. Less well defined is what exactly makes an organization centralized or decentralized. To answer this question, this section will first present the well-defined forms of organizational structure. Second, the (de)centralization of current styles of IT governance will be explored. Third, relevant principles from existing decentralized organizations and organizational science will be outlined. By then bringing together these three areas, this paper will then present a number of characteristics for defining decentralization in an enterprise. 

\subsection{TODO}
Relate each section to generic definition of decentralized

Expand on "What is a Decentralized Organization?" section to relate back to previous subsections

\subsection{Generic Definition of "Decentralized"}

% to have a starting point, then show how each of the things in the section match up to being decentralzed

% can then use the decentralized characteristics for a "definition" of a decentralized organzation

Centralized is defined as "Having power concentrated in a single, central authority" [REF]

Decentralized is defined by Merriam-Webster "the dispersion or distribution of functions and powers; specifically : the delegation of power from a central authority to regional and local authorities" [REF]



\subsection{Specific Forms of Organizational Structure}
\label{org:form}
%Organization have traditionally utilized more centralized forms of organizational structure. Of the centralized structures, a hierarchical organization structure is perhaps the most well-known and most utilized.

Perhaps the most well-known and traditional form of enterprise structure is the hierarchical form. As shown in figure[HIERARCHICAL FIGURE REF], it is characterized by a a hierarchy of positions. Pearlson and Saunders offer a thorough description of a pure hierarchical organization structure~\cite{pearlson2009}: Except for the top level position, each position has one superior and zero or more subordinates. Decision rights and communication lines are strictly defined and work their way down from the top (i.e. the centre). The scope of a position is specialized and strictly defined by your superior and one works in assigned teams. The primary benefit of a hierarchy is that the high levels of management have strict governance and control of everything that goes on in the company. This allows them to easily direct the company how they deem best. Hierarchical organizations generally either divide their labor in terms of function, a grouping of common activities, or in terms of division, a grouping based on output. Due to this, hierarchical organization structures are suited for stable, certain environments. 

[HIERARCHY DIAGRAM]

Another popular style of organization structure is the matrix organization structure~\cite{pearlson2009}. In this style, individuals are assigned two or more supervisors covering different dimensions of the enterprise. The aim here is to integrate these different dimensions. Pearlson and Saunders state that matrix organization structures are suited for dynamic environments with lots of uncertainty, presumably because their authority structure allows them to cover multiple aspects when making decisions. However, like a hierarchical structure, a matrix structure is a rigid construct with strictly defined roles, communication lines and decision rights. Authority still comes from the top in a centralized manner, even though it becomes more distributed among matrix managers at the lower levels~\cite{pearlson2009}. Consequently, matrix structures still may not be perfectly suited for uncertain, dynamic environments. 

[MATRIX DIAGRAM]

Applegate, Cash, and Mills~\cite{applegate1988} support this statement, as they describe hierarchical and matrix structures as rigidly structuring "communication, responsibility, and accountability to help reduce complexity and provide  stability". They furthermore state that both matrix and hierarchical structures have the effect of stifling creativity and preventing organizations from being able to adapt effectively to rapidly changing environments. 

[DRAWBACKS]

According to Pearlson and Saunders~\cite{pearlson2009}, another structure that is highly centralized is the flat type. A flat organizational structure is characterized by having a single (or small number) person at the top. The rest of the personnel are all below the top level and are equal to one another. This kind of flat structure is effectively a hierarchical structure with only two levels. A common structure for new companies, Pearlson and Saunders state that is a centralized from of organizational structure as all the power and decision making authority typically is controlled by the authority at the top. This is consistent with the simple structure outlined by Mintzberg where "Mom or pop constantly monitors what is going on and exercises total authority over daily operations"~\cite{Mintzberg1979}. However, this is not always true for flat organizations, it depends on how they operate. For example, Valve Corporation, a software company in the video game industry released their handbook in 2012~\cite{valveHandbook}. In it, they describe their structure as being flat, but a very different style of flat where individual employees have complete freedom despite there being a president/founder at the top. Unlike the style of flat organization described by Pearlson and Saunders, at Valve it is a highly decentralized style. Nobody reports to anyone, and everyone is free to work on whatever they want to. Valve states that the company is "yours to steer"~\cite{valveHandbook}, meaning that everyone the power to alter the direction of the company. This difference in what a "flat" organization demonstrates is that it is important to take into account more than simply the structure of a organization, how that structure is implemented is equally important. 

[FLAT DIAGRAM]

In recent years a new type of organizational structure has emerged, called the networked organization structure, or adhocracy~\cite{applegate1988,pearlson2009}. As depicted in figure [NETWORKED FIGURE], a networked structure aims to discard traditional hierarchies in favor of decentralized decision rights and flexible communication lines connecting the entire enterprise~\cite{applegate1988,pearlson2009}. Specifically, instead of hierarchies, an adhocracy has a rapidly changing set of project oriented groups. Mintzberg describes an adhocracy as "a loose, flexible, self-renewing organic form tied together mostly through lateral means"~\cite{Mintzberg1979}.  Regardless of the specifics of its definition, this form of organizational structure is clearly vastly different from a rigid, centralized structure. This enables an organization where many (or all) employees are able to easily share knowledge and provide input into the overall decision making for the organization~\cite{pearlson2009}. An important effect of this is a flexible enterprise that promotes creativity. Together, these characteristics make an organization that is suitable for dynamic and uncertain environments due to its ability to adapt quickly.

[NETWORKED FIGURE]

\subsection{Existing Decentralized Organizations}

\subsubsection{Smart Cities and the Principle of Co-Design}

Smart Cities is a collaborative project between a number of governments and universities seeking to achieve excellence in the field of e-services~\cite{Cities}. One of their publications explored a concept called co-design and how it contributes to the delivery of e-services. 

% explain how smart cities is decentralized

The general idea of co-design, according to Smart Cities, is to bring all stakeholders into the decision making process as equals. The ideal of co-design is to actively involve all stakeholders (including users) of a new system or service to participate in defining what it should do, the process to develop it, and to provide acceptance that the end result functions as it should. Co-design is seen as a way to tackle the many conflicting views and goals that stakeholders will have. 

There are number important characteristics to proper co-design as viewed by Smart Cities. First, it is a completely transparent collaborative process where all participants are able to contribute. Second, the focus of co-design is on both developing the system or service, and on and improving the development process. Third, co-design involves shifting "power to the process"~\cite{Cities} instead of only on a subset of stakeholders.

Smart Cities views co-design activities in three different dimensions; horizontal, vertical, and intensity. Horizontal co-design is done with partner organizations in order to deliver services. Vertical co-design involves stakeholders at various levels in the "service delivery chain"~\cite{Cities}, end-users or other departments. The third dimension is the intensity or degree in which the co-design participants can contribute. 

% may want to link to other literture on co-design

%\subsection{Coopetition}
%
%The term "coopetition" is a portmanteau of the terms "competition" and "cooperation", used to describe a relationship between parties where they both collaborate and compete. Bengsston and Kock describe coopetition as a complex relationship between firms where they simultaneously compete and collaborate and benefit from both~\cite{Bengtsson2000}. They explore the concept of coopetition in the context of competing firms that "produce and market the same product".~\cite{Bengtsson2000} In this context, they place an important limit on coopetition: their needs to be some kind of separation between the competing and cooperating aspects i.e. they can not "coopetate" in the same aspect. 
%
%An advantage of coopetition is that it allows the participating organizations to take advantage of a heterogeneity of resources~\cite{Bengtsson2000}. Organizations may seek to create competitive advantage through a unique resource they own (e.g. skill). At the same time, it might be beneficial for them to cooperate with another organization that possesses a unique resource that is of value to them. Together, these two factors can lead to a relationship of coopetition that allows the participants to develop in new areas. 
%
%An example of coopetition is Amazon.com's Marketplace; a platform provided by Amazon where any  competitor can list items for sale alongside Amazon's own sales, often of the same items~\cite{UnknownAskIrina,Amazon.com}. This allows sellers to take advantage of Amazon's platform while Amazon takes advantage of the increase in traffic. 
 
% relationship to decentralized:
% Coopetition is a way two entities interact with each other; no overseeing central auuthority


\subsubsection{Collaborative Networks}

NOTE: Not sure how this fits in with the flow

Related to the idea of a networked organization structure is the concept of collaborative networks (CN). Camarinha-Matos and Afsarmanesh define collaborative networks as being composed of "a variety of entities (e.g., organizations and people) that are largely autonomous, geographically distributed, and heterogeneous in terms of their: operating environment, culture, social capital, and goals."~\cite{Camarinha-Matos2005} These entities then collaborate through the use of some sort of computer network in order to achieve common or complementary goals. The main driver behind CNs is that the goals the seek to achieve would be impossible or much more difficult to achieve without collaboration. The composition of autonomous entities makes CNs a very relevant concept to decentralization. 

Some examples of collaborative networks are virtual organizations, a group of independent organizations working together to achieve some goal(s); virtual communities, a community of individuals that interact with each other through the use of computer network-based technologies; and virtual breeding environments, a group of organizations that set up a framework for inter-operability in order to enable the potential for forming a virtual organization~\cite{Camarinha-Matos2005}. Three common characteristics in various CNs are autonomy in the individual entities, a drive towards meeting common or complementing goals, and the use of an agreed-upon framework for collaboration. 

% I think it is necessary to have properties of each some collaborative networks in order to match up with what 

% VO
% \cite{Kerschbaum2009}
% "Autonomy of participating organizations in the VO must be maintained."
% no global state in our control-flow enforcement in choreographies

% VE \cite{Camarinha-Matos1999}
% flexible/reactive systems are important

% Virtual Communities

\subsection{What is a Decentralized Organization?}

As demonstrated in section~\ref{org:form}, whether an enterprise is centralized or decentralized depends on more than simply its structure. Furthermore, enterprises will have elements of both centralization and decentralization in them, meaning that would be an oversimplification to classify an enterprise as just one or another. Consequently, organizational structure is best viewed as being on an organizational continuum, with decentralization on one end, centralization on the other end, and federalism in the middle~\cite{pearlson2009}. This section will describe a number of organizational characteristics that can be used to determine to what degree an organization is centralized or decentralized. 

The first characteristic are the allocation of decision rights in an enterprise, specifically who has the right to make what kinds of decisions in running and planning the enterprise.~\cite{pearlson2009} Decision rights are what controls the overall direction of an enterprise. They are needed in order to make change. In a completely centralized enterprise, all decision making authority would reside with a single, top-level authority. More decentralized enterprises would allow for participation in the decision making process by members throughout the enterprise. In a completely decentralized enterprise all members would have equal decision making rights. 

A second characteristic is the structure of communication lines in an enterprise. A centralized enterprise will have rigid hierarchies of communication, i.e. it is strictly defined who you work with and who you report to. Decentralized enterprises instead have less formalized communication lines~\cite{pearlson2009}, and more fluid, project oriented teams.~\cite{Applegate1988a}

The third characteristic is the choice of forms of coordination in an enterprise. More centralized enterprises lean towards primarily vertical style of coordination~\cite{Bolman2008}, which is characterized by formal authority, standardization and rules in operations and in IT, and planning and control systems. [ELABORATE] Decentralized organizations lean towards more lateral styles of coordination. Lateral coordination is characterized by meetings, task forces, coordinating roles, matrix structures, and networks~~\cite{Bolman2008}[ELABORATE AND MAYBE NOT USE EXACTLY] It should be noted that enterprises will generally have a mix of lateral and vertical coordination, it is the tendency of an enterprise to focus on one more than the other that is an indicator for decentralization. 

%\subsection{The Importance of Collaboration}
%The idea that the whole needs to work together to operate in a way that benefits everyone --> still separate entities with independence doing their own thing at the same time as working together when appropriate for mutual benefit.
%
%Caruso, Rogers and Bazerman~\cite{caruso2008boundaries} highlight the importance of information sharing and coordination for decentralized organizations. 
%
%"In what will be an even faster changing world than the one we now  know,  businesses of all sizes  will need the ability to adapt to the dynamics of the exter­ nal environment. Automated information and com­ munication networks will  support the sharing of information throughout a large, widely  dispersed, complex company. The systems will form the organi­ zation's infrastructure and change the role of formal reporting procedures. Even in large  corporations, each  individual will  be able to communicate with any  other-just as if he or she  worked in  a small company."\cite{applegate1988}

\subsection{Challenges in Decentralized Organizations}

NOTE: Not sure how this fits in with the flow

As decentralized organizations function in a significantly different manner than centralized organizations, they offer a different set of challenges that need to be faced: "~...~the novel and dynamic pressures that create the demand for decentralization in the first place can place organization leaders in considerably less certain, and consequently less commanding, positions."~\cite{caruso2008boundaries} In order to effectively meet these challenges, it is important to first understand what they are. [LIST WHAT I WILL TALK ABOUT]

A paper by Caruso, Rogers and Bazerman~\cite{caruso2008boundaries} highlights the importance of information sharing and coordination for these organizations. In order to succeed at these aspects, they outline three barriers that decentralized organizations need to overcome. The first barrier is intergroup bias; the tendency to treat one's own group better than other groups. The second barrier is group territoriality; the tendency for a group to protect their territory (physical or informational). The third barrier is poor negotiation strategies used by different groups when interacting with one another. 

Intergroup bias is direct result of having separate, autonomous groups within an enterprise~\cite{caruso2008boundaries}. The individual groups have a tendency to promote their own group over other groups, especially in situations where they are competing for a resource, such as a portion of the budget. A certain level of competition can be beneficial, however if it leads to hostility or distrust between groups, this can have a detrimental effect on their ability to share information and collaborate. This can prevent the groups from taking advantage of situations where they have to ability to work together for the benefit of everyone. 

The second barrier identified by Caruso et al. is group territoriality~\cite{caruso2008boundaries}. Group territoriality is characterized by group members taking action in order to protect their perceived territory. This can include physical territory such as space or tangible resources, as well as intangible territory, such as roles or information. Group territoriality is supported by a group's need to maintain its identity, its reputation of competence and sense of value, and a group's need for a stable home within the organization from which they interact with the rest of it.

Group territoriality encourages "a sense of psychological ownership"~\cite{caruso2008boundaries} for a group's members which can enforce the belief that they are the sole responsible party for a role or specific knowledge. This "inward-looking" behavior works against collaboration and information sharing. On the other hand, group territoriality can be beneficial; it can foster a sense of security in its members that "facilitates planning and execution of activities"~\cite{caruso2008boundaries}. 

The third barrier identified by Caruso et al. in decentralized organizations is related to negotiations between groups, and how these negotiations are often conducted using "poor negotiation  strategies"~\cite{caruso2008boundaries}. These poor strategies are the result of three common errors made while negotiating. The first error is a false belief in a "fixed pie" of value that is to be divided when negotiating. This prevents negotiating parties from recognizing situations where they are able to help each other, and therefore increase the size of the figurative pie. The second error is a failure to properly consider the other group's perspective. Understanding the other group's decision process, valuing process, and interests is key to discovering opportunities for helping one another, and the organization as a whole. The third error is when groups fail to even recognize they are in the process of negotiating. Instead, they see it as a competitive or hostile behavior where, again, they only see a fixed pie that is to be split up. This also prevents groups from taking advantage of opportunities to increase the size of the pie.    

% STRUCTURAL DILEMMAS from Reframing...

%\subsection{Case Study: Challenges in SweU}
%
%Budget: Better to use exactly rather than go under
%
%SweU already views collaboration as a way to increase the size of the pie
%
%more


%\subsection{Drivers for Decentralization}
%
%Flexibility \& adaptability
%
%Customization to local needs
%
%Promote creativity
%
%Globalization
%
%Driver for inter-organizational networks: "fast-moving fields ... where knowledge is so complex and widely dispersed"~\cite{Bolman2008}
%
%strengths of centralized i.e. hierarchies
%define dynamic, uncertain, rapidly changing, stable... environments
%
%"novel and dynamic pressures that create the demand for decentralization in the first place can place organization leaders in considerably less certain, and consequently less commanding, positions"
%
%page 233 pearlson
%
%\subsection{Case Study: An Institution of Higher Education in Sweden}
%
%DSV pushes decision making power as close to the operational level as possible
%
%Usage of meetings/seminars to bring everyone together and plan major change (Bologna)
%
%Fluid, project-oriented teams; research projects
%
%Informal communication lines \& ability to collaborate whenever they see fit (and is self-managed): DSV IT collaborates with other departments, collaboration for education programs
%
%Task forces: reference groups of appropriate knowledgeable people for major (non-research) projects
%
%Standardization exists, but is frequently not forced
%
%Minimal formal metrics for performance measurement: where government mandated (re: education), but not much else.  

%\section{Challenges in Collaboration for Decentralized Organizations}
%\label{challenges}
%\subsection{The Importance of Collaboration}
The idea that the whole needs to work together to operate in a way that benefits everyone --> still separate entities with independence doing their own thing at the same time as working together when appropriate for mutual benefit.

Caruso, Rogers and Bazerman~\cite{caruso2008boundaries} highlight the importance of information sharing and coordination for decentralized organizations. 
        
\subsection{Challenges}
As decentralized organizations function in a significantly different manner than centralized organizations, they offer a different set of challenges that need to be faced: "~...~the novel and dynamic pressures that create the demand for decentralization in the first place can place organization leaders in considerably less certain, and consequently less commanding, positions."~\cite{caruso2008boundaries} In order to effectively meet these challenges, it is important to first understand what they are. Caruso et al. outline three barriers to collaboration; intergroup bias, group territoriality, and poor negotiation  strategies. Other challenges include.... (communication, planning, maintenance, decision making, change management, budgeting)

\subsection{Case Study: Challenges in SweU}

Budget: Better to use exactly rather than go under

SweU already views collaboration as a way to increase the size of the pie

more
%
%\section{Relation Between IT Governance and Decentralized Organizations}
%\label{governance}
%\subsection{IT Governance}

Rockart, Earl and Ross~\cite{Rockart1996} describe a continuum of governance styles ranging from centralized to decentralized, with federalism in the middle. This style of Federal IT aims to have the strengths of both centralized and decentralized while eliminating their respective weaknesses. Some key weaknesses of centralized IT to eliminate are slow responsiveness and having systems that do not fit the needs of individual business units. Decentralized IT on the other hand lacks "synergy and integration"~\cite{Rockart1996} due to a lack of standardization. Federal IT would aim to balance these through a combination of central IT and IT in the business units. A primary task of the central IT would be to maintain standards for the entire enterprise. The business units would still have ownership of many of their own systems, allowing them to implement them as they deem best. This allows for systems that meet the individual business units needs, as well as enabling interoperability throughout the enterprise. 

Weill~\cite{Weill2004} also describes a federal style of IT governance as part of a larger description of IT governance archetypes and the major types of IT decisions that they apply to. In addition to the federal archetype, he describes a business monarchy, IT monarchy, feudal, IT duopoly and anarchy. In a business monarchy all IT related decisions are made in a centralized manner by the top-level executives (e.g. the CxOs). In an IT monarchy, a group of IT professionals are responsible for making the decisions. This is also highly centralized as the authority resides with this group. An IT duopoly is characterized by two groups, one of IT executives and the other of business executives, coming to agreements in order to make decisions. This is more centralized than the federal form, as the decisions are only made by the two groups, rather than each individual business unit having input. The feudal is much less centralized. It is where individual organizational units are responsible for their own decisions. Anarchy is a highly decentralized style of governance. It is similar to the feudal archetype, however the size of the units is much smaller. Instead of being an entire business unit, small teams or even individuals are responsible for their own decisions.

Weill then proposes five major IT decision domains that these decisions apply to: IT principles, IT architecture, IT infrastructure strategies, business application needs, and IT investment~\cite{Weill2004}. IT principle are high level statements about the use of IT in the enterprise. IT architecture decisions relate to the policies and rules that describe how IT is to be sued, as well as the roadmap for implementing them. Decisions on IT infrastructure strategies relate to the foundation IT services (e.g. network, help desk) that exist throughout the entire enterprise.  Decisions on business application needs are about determining what needs IS will fulfill. IT investment decisions are related to financing and justification of IT projects.

% not certain here if these references should go to Weill or to his referenced papers. 

\section{Shortcomings of using EA for Decentralized Organizations}
\label{shortcomings}
The field of Enterprise Architecture (EA) emerged in order to combat two increasingly prevalent problems facing enterprises: system complexity and business-IT alignment~\cite{sessions2007}. As enterprises rely more and more on information systems of increasing complexity, these problems become even more important. The field of EA views the solution to these problems to be one of concurrent design. It is not enough simply try and fit IT to the business; business and IT aspects should be designed concurrently.
%
%....EA is an enabler of collaboration, which as show earlier is an important concept to decentralized enterprises...however 
%   
%As shown in the previous section, collaboration important to decentralized enterprises: need way to collaborate
%    
%\textbf{The Problem: }EA could be that solution, however it has shortcomings for Decentralized Enterprises. 

%IF YOU HAVE ENOUGH INFORMATION, YOU CAN SHOW THE "CENTRALIZED -ORIENTED" and "DECENTRALIZED ORIENTED" characteristics of the considered EA frameworks in a TABLE: 
%Columns: TOGAF, ZACHMAN, FEA
%Rows: METHOD, Description, Engine
%
%IF NOT - TEXT IS OK.

\subsection{TOGAF}

\subsubsection{Concepts supporting a centralized organization}

%TOGAF outlines a formal approach to architecture governance which involves the setting up of an "Architecture Board"~\cite{togaf9.1}. "to oversee the implementation of the [architecture] strategy"~\cite{togaf9.1}. This board has an important role in Architecture Governance, such as "[p]roviding the basis for all decision-making with regard to the architectures"~\cite{togaf9.1} and enforcing architecture compliance. The TOGAF Architecture Governance Framework suggests guidelines for developing a formal governance structure for the Enterprise Continuum (and thus, all the architectural artifacts) and architecture processes.

The Architecture Board suggested by TOGAF is a centralized concept. Having a single entity responsible for high-level decision making fits in with the concept of a centralized organization discussed in section [LINK]. TOGAF does suggest that the board has enterprise-wide representation~\cite{togaf9.1} which may support some level of decentralization, however it suggests the representation comes in the form of "senior managers"; a concept primarily from traditional organization structures. 

Throughout TOGAF, references are made to the existence of a bureaucratic or hierarchical centralized structure in place. For example, an important part of the preliminary phase is to set up a formal governance framework for all architectural material, a concept that is related to the rigid forms of traditional organizational structure. A second example of this is after the completion of "Phase A: Architecture Vision", TOGAF requires approval of the current vision of the architecture. This requirement of approval assumes the existence of someone with a higher level of decision-making authority to give approval. A third example is an entire set of architectures at the strategic level of the Architecture Landscape which is meant for the "executive level"~\cite{togaf9.1}.

TOGAF suggests the development of architecture principles that "...define the underlying general rules and guidelines for the use and
deployment of all IT resources and assets across the enterprise"~\cite{togaf9.1}. Furthermore, TOGAF suggests principles can be organized into a hierarchy of principles. Having a central set of principles that is to be applied to an entire organization supports centralization.

TOGAF includes the concept of an Architecture Repository, which is to hold the entirety of the Architecture Landscape in addition to other architecture-related information. The idea of a single place to store all information is highly supportive of centralization. 

\subsubsection{Concepts supporting decentralization}

TOGAF primarily supports some level of decentralization through the concept of partitions. It suggests partitioning the Architecture Landscape into separate partitions in order to support multiple architecture teams working concurrently and conflicting architectures in different organizational units. This enables "federated architectures — independently developed, maintained, and managed architectures that are subsequently integrated within an integration framework — are typical. Federated architectures typically are used in governments and conglomerates, where the separate organizational units need separate architectures"~\cite{togaf9.1}. This supports the idea of different organizational units developing their own individual architectures, at least on the level of a federated organizational structure. 

%"modular architecture segments that can be taken and incorporated into broader architectures and solutions"~\cite{togaf9.1} 

TOGAF additionally indirectly supports decentralization through the suggestion that the entire TOGAF process be tailored to fit the needs of the enterprise. This is done in the preliminary phase of the ADM. In theory, this would allow TOGAF to support any kind of enterprise. The guidelines provided for this, however, are minimal. 


\subsection{Zachman}
%Limited perspectives?
The Zachman Framework aims to model a complete enterprise in a single, "periodic table of elements"~\cite{Bente2012}. It attempts to break down an enterprise into exactly 30 different views. This lack of flexibility is perhaps the Zachman Frameworks main shortcoming with respect to decentralization: a primary aspect of decentralized organizations is their high level of flexibility. Furthermore, the perspectives of Zachman Framework line up with a bureaucratic organizational structure (e.g. executive and business management perspectives). For a traditional, centralized organization these perspectives make a lot of sense. For a decentralized organization where these roles are not as defined, however, it is not clear how to use the Zachman Framework. 

\subsection{FEA}

Through the use of a common set of reference models, FEA prescribes standards that are to be followed throughout the organization. This limits the flexibility that the individual organizational units have, as they must follow this set of standards. On the other hand, in FEA, individual organizational units have the freedom to develop their own architecture as long as it fits in to the set standards. This supports some level of decentralization as it allows the individual units to do what they want, to a degree. This is similar to the federal style of governance outlined by Weill~\cite{ross2006}, where individual units have input into decisions. 

Further support for decentralization exists in FEA in the development of segment architectures, which is described as a collaborative process between EA architects and other staff members~\cite{FederalEnterpriseArchitectureProgramManagementOffice2007}, thus giving them some degree of input into the process. 

SUMMARY TABLE;
(see my handwritten comments in pdf)


%\section{Potential Principles for a Decentralized EA}    
%\label{principles}
%Other domains (e.g. technical) have tackled their version of the problem of decentralization
    \begin{itemize}
    \item Borrow principles from them
    \end{itemize}
    
    Organizations have been operating in a decentralized manner without a concrete "EA"
    \begin{itemize}
    \item Borrow principles from them
    \end{itemize}
    
    "Social Peer-to-Peer"; Kickstarter/Crowdfunding, Reddit/Voting. This is related to some EDOC topics:
    \begin{itemize}
    \item Cross-enterprise collaboration in a world of cloud, \textbf{social} and big data
    \item Social information and innovation networks, social media impact on the enterprise
    \end{itemize}

\subsection{Principles from Peer-to-Peer}

Why P2P

Treat EA description as a distribute resource

Distributed responsibility 

\subsection{Principles from Existing Decentralized Organizations}

Co-design, from SmartCities

Social networks

\subsection{Principles from Organizational Science}

Co-opetition


\section{Conclusion}
\label{conclusion}
In this study we have analyzed the problem of non-fit between emerging decentralized organizational environments, and established EA methodologies aimed to model organizations and specifically their linkages between business and IT. 

We have argued that modern organizations show strong tendencies toward decentralization in their organizational structures and thereby IT governance by following different patterns, having in common fostering of entirely new relationships between business processes and IT, and how IT resources are managed. The classification of organizational forms of IT presented in Section 3, we have used to assess if the dominant EA methodologies can support them.

Current EA, providing methods to set  up organizations' IT architecture, management and evolution, fail to solve this major concern in decentralized environments by helping organizations  to address  the
challenges related  to changes required  for, or issued by IT. We have surveyed Zachman Framework, TOGAF and FEA, and concluded that the first is unable to support any significant aspect of decentralization, while the latter two provide some basic flexibility in TOGAF, it is mainly facilitated by the ability to have different architecture for organizational units and by providing  space for new methods for the architecture development; in FEA, the conclusions are similar,  while the top- level organization  standards need to be obeyed by all units. Consequently, implementations of these methodologies are heavily limited to promote, or even support new decentralized organization patterns fostered by virtual organizations, collaborative networks, coopetitions, and others.

The aim of this research is to contribute to a state-of the-art on enterprise modeling methodologies by analyzing the decentralization of organizations and supporting business patterns and technologies, and thereby the consequences of this trend to the requirements  for new approaches  to IT resources,  namely, their use  and management. Regarding future work, our next steps involves contrasting the presented theories and argumentations empirically, i.e. by mapping them to EA of different organizations.  Such an ongoing study concerns an organization in the public sector of Sweden, exposing many of decentralized behavior as discussed in this paper.

% TODO
%Integrate Jelena's comments
%Integrate Irina's Comments
%Need some taxonomy of EA terms


\renewcommand{\baselinestretch}{0.98}
\bibliographystyle{IEEEtran}
{\small
\bibliography{biblioDecentralizedEA}
}
\renewcommand{\baselinestretch}{1}

%
\end{document}
